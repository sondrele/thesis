% !TEX root = ../main.tex

\section{Rust Embedded Modules}
\label{sec:rust-embedded-modules}

This section describes a separate project, which we refer to as \gls{rem}.
\gls{rem} have been developed alongside the various binding libraries, and it contains a couple of higher-level modules for the different peripheral bindings that are part of the {\rg} platform.

We have looked to other projects like {\zinc}, \texttt{arduino}, and ARM \texttt{mbed} for inspiration to this library.
The peripheral abstractions that have been implemented as part of \gls{rem} are still in very early development, and most of them are not yet general enough to be adapted to all new projects.

\subsection{USART}
\label{ssub:usart}

The \gls{usart} have many different use-cases.
It is a peripheral that is used for transferring of data, but it is also a very convenient tool to use for simple debugging of programs.
It can be used to send single strings of text between a PC and the \gls{mcu}, which is convenient for ``println''-debugging, and it is a good tool for defining \gls{cli} programs.

The {\gecko} has a total of three different \glspl{usart} which can be configured to run on a total of eleven different locations (i.e. \gls{gpio} Ports and Pins).
If the \gls{gpio} configuration for a \gls{usart}  is not specified correctly, the peripheral will not function correctly either.
The goal of the \gls{usart} abstraction was to make it easy to initialize the peripheral, as well as providing simple methods to read and write strings, and transfer data between two end-points.

The \texttt{Usart} module has an initialization procedure that takes care of initializing its required \gls{gpio} pins based on a specified location.
This is similar to the approach made by {\zinc}, but instead of failing with a compilation-error if it is incorrectly configured, as in {\zinc}, it will fail at run-time.
A minimal example that shows how to initialize a \gls{usart}, and send and receive strings, is shown in \autoref{lst:usart_abstraction}.
Note that the example is simplified slightly, we have trimmed away some \code{extern} and \code{use} statements to make the important parts clearer.
The point of this example is to demonstrate a program that initializes and uses a \gls{usart} with only four lines of code.


\begin{listing}[H]
  \begin{rustcode}
extern crate emlib;   // Include `emlib' bindings
extern crate modules; // Include `REM'
use modules::Usart;   // The Usart module

fn main() {
  // Acquire a USART with default configuration...
  let mut usart: Usart = Default::default();
  // ... and initialize its GPIO.
  usart.init_async();

  loop {
    // Perform a blocking read operation...
    let name = usart.read_line();
    // ... and echo back with a nice message.
    usart.write_line(&format!("Thank you, {}!", name));
  }
}
  \end{rustcode}
  \caption{Example usage of \gls{rem}'s \gls{usart} module}
  \label{lst:usart_abstraction}
\end{listing}

\subsection{GPIO} % (fold)
\label{ssub:gpio}

The \gls{gpio} peripheral are used extensively as a dependency for many other peripherals throughout {\emlib}.
The microcontroller's \gls{cpu} pins are configurable as \gls{gpio}, all these pins are ordered into ten ports (Port A - Port H) with up to 16 pins each (Pin 0 - Pin 15).
The pins can be configured individually to be used as input, output, or both, to the \gls{mcu}.

The \gls{gpio} module in \gls{rem} consists of a \code{GpioPin} structure, and two traits for \code{Button}'s and \code{Led}'s, respectively.
The \code{GpioPin} is an abstraction on top of {\emlib}'s \gls{gpio} definition, and the two traits implements a few convenient methods that abstracts away the underlying \code{GpioPin}.
An example that shows how to initialize and use a button and a LED on the {\STK} is shown in \autoref{lst:gpio_abstraction}.
Notice from the code that we only ever interfere directly with the two traits, and that we do not care about the underlying \code{GpioPin}, apart from when we define the button and the LED.

It is important to note the \emph{current} limitations of this module.
The two implemented traits provides an intuitive abstraction layer for their purposes, but \gls{gpio} in general, is so much more than just buttons and LEDs.
The module was first developed for buttons and LEDs because they are easy to interface with, but it will require some alteration to be more `general purpose'.

\begin{listing}[H]
  \begin{rustcode}
extern crate emlib;    // Include `emlib' bindings
extern crate modules;  // Include `REM'
use emlib::gpio::Port;
use modules::{GpioPin, Button, Led};
// Define a button and a LED. The 'static lifetime tells
// us that BTN and LED are alive for the whole program
const BTN: &'static Button = &GpioPin { port: Port::B, pin: 9 };
const LED: &'static Led    = &GpioPin { port: Port::E, pin: 2 };

fn main() {
  // Initialize the underlying GPIO pins
  BTN.init();
  LED.init();
  // Register a callback function for the button
  BTN.on_click(blink_led);
  loop {}
}
// This function gets called when the button is pressed
fn blink_led(_pin: u8) {
  LED.toggle();
}
  \end{rustcode}
  \caption{Example usage of \gls{rem}'s \gls{gpio} module}
  \label{lst:gpio_abstraction}
\end{listing}

\subsection{DMA}

The \gls{dma} peripheral on the {\gecko} is used for transferring data from one location to another, without intervention from the \gls{cpu}.
The interface to the \gls{dma} provided by {\emlib} is a low-level API, which deals with hardware descriptors and raw pointers for controlling the \gls{dma}.
Here we look at a higher level abstraction over the \gls{dma} module, which considers devices and buffers and the specification of the flow between these devices.

The core of the approach defines two types of endpoints which can be interacted upon with \gls{dma}, \emph{readable} and \emph{writable}.
In this classification, an \gls{adc} is readable as it produces samples and, the \gls{dac} is writable as it can consume samples.
A \gls{usart} fits both descriptions; this is because data can both be written to and read from the peripheral.
Both descriptions do also apply to memory allocated buffers.

These abstractions are modeled by providing the two traits, \code{Readable} and \code{Writable}, as shown in \autoref{lst:dma:traits}.
It is important to note that, even though the two traits define the \emph{same} methods, they have a \emph{semantic} difference.
% It is important to note that the two traits are there for a \emph{semantic} difference even though they define the same methods.

\begin{listing}[H]
  \begin{rustcode}
trait Readable {
  fn as_ptr(&self) -> *mut c_void; // Base pointer to device
  fn inc_size(&self) -> DataInc;   // Increment in bytes
  fn size(&self) -> DataSize;      // Element size in bytes
  fn n(&self) -> Option<u32>;      // Number of elements to transfer
}
trait Writable {
  fn as_ptr(&self) -> *mut c_void;
  fn inc_size(&self) -> DataInc;
  fn size(&self) -> DataSize;
  fn n(&self) -> Option<u32>;
}
  \end{rustcode}
  \caption{Traits used for \gls{dma} transfers}
  \label{lst:dma:traits}
\end{listing}

Both traits defined in \autoref{lst:dma:traits} requires the same set of methods.
These methods are required by the underlying \gls{dma} implementation and are described in the {\emlib} documentation for \gls{dma} \cite{Dma2004}.
As mentioned, both traits can be applicable for some peripherals, and this demonstrates {\rust}'s facility to do method resolution when a type implements multiple traits that can result in function name collisions.

\begin{listing}[H]
  \begin{rustcode}
// Create a static RAM buffer
static mut BUFFER: [u8, 4] = [0; 4];

fn main() {
  // Initialization omitted. Assume instead
  // that `adc0' and `dma0' hold references to the peripherals

  // Start a DMA transfer from the ADC to RAM
  dma0.start_basic(
    &adc0,                  // The `adc0' implements `Readable'
    unsafe { &mut BUFFER }, // The RAM buffer implements `Writable'
    AdcSingle               // Reference to the interrupt signal
  );
  loop {}
}

// The signature for the start_basic is included for the discussion
impl Dma {
  pub fn start_basic(&mut self,
                     src: &Readable, dst: &Writable, on: Signal);
}
  \end{rustcode}
  \caption{DMA transfer utilizing the trait abstractions}
  \label{lst:using:dma}
\end{listing}

\autoref{lst:using:dma} shows an example of using the \gls{dma} abstraction to transfer samples from the \gls{adc} to \gls{ram}.
The \func{start\_basic} function relies on the two traits to set up the hardware specifiers with the low-level \gls{api}.
Both the \code{Readable} and \code{Writable} traits are implemented for the memory buffer.
In this use case, the type of the \code{dst} parameter in the \func{start\_basic} function ensures that the correct implementation is chosen.

The \gls{dma} abstraction described here is only implemented for the simplest \gls{dma} transfers from the {\emlib} \gls{dma} \gls{api}.
The {\emlib} \gls{api} provides more complex facilities (e.g. scatter-gather) which can cause the interfaces to change if they are to be supported.
However, there exists facilities for registering closures to handle reactivation of long running \gls{dma} transfers; the closure will be called on the interrupt signal given by the \gls{dma} controller when the transfer is finished.
This facilitates a programming model which is similar to the one presented in \autoref{sec:irq-closures}.
