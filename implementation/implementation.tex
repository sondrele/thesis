%\chapter{Implementation}
%\label{chap:impl}

%The implementation chapter describes various software components both developed and fitted to create a framework for programming in Rust for an Embedded system.
%The last sections dives into two projects which were developed in order to drive the deveopment of the framework and to provide a basis to evaluate the platform in the next chapter.

%We start in \autoref{sec:impl:booting} by describing the startup process of a Rust program on a Microcontroller.
%Secondly we define the parts of the Rust standard library \gls{rel} which are applicable for an embedded system.
%In \autoref{sec:impl:oo} we look at the how and why the Object Oriented paradigm can be applied to hardware devices.
%\autoref{sub:interfacing_with_emlib} goes on to describe the hardware specific bindings developed for controlling the Peripherals of the Microcontroller.
%In \autoref{sec:build_system} we take a look at the evolution of the building system from a traditional \file{Makefile} to the integration with the {\rust} package manager {\cargo}.
%The next two sections \autoref{sec:irq-closures} and \autoref{sec:rust-embedded-modules} explores a few experimentation with creating higher level abstractions for the lower level Peripheral libraries.
%\autoref{sec:porting-gpioint} looks at a case study of porting a GPIO driver from {\C} to {\rust}.
%The last section \autoref{sec:impl:projects} gives an overview over the implementation of the projects used to evaluate the programmability and certain metrics presented in \autoref{chap:results}.

\newcommand{\corechapter}[3]{
  \chapter{#1}
  \label{chap:#2}
  \begin{center}
    \includegraphics[scale=0.8]{figures/RustyGecko-#2}
  \end{center}
  \hfill \break
  \hfill \break
  \hfill \break
#3
}

\corechapter{Startup}{startup}{%
  The startup module handles the process of starting a {\rust} application on the \gls{mcu}.
  This involves supporting some basic requirements for the language and setting up some important initialization of the processor.
  This module is a foundation for the {\rg} platform, but can be used in isolation to run a program on the \gls{mcu} without any dependencies.
}
\section{Booting Rust on the Gecko}
\label{sec:impl:booting}

The contents of this section explains how the startup process described in \autoref{sec:back:startup} is implemented for a {\rust} program.
The process for {\rust} is identical to the process in {\C} due to the fact that {\rust} only allows constant initialization\footnote{This is in contrast to languages like {\Cpp} where objects can be initialized with statically called constructors, this must be handled by the startup process.} before the {\main} function.
Therefore we are able to reuse standard runtime components that are available for {\C} embedded toolchains.
Silicon Labs' software suite provides the \func{ResetHandler} and linker script for the EFM32 microcontrollers.
The \func{\_mainCRTStartup} function defined in the {\C} runtime, which is provided by the {\armgcc} toolchain, handles the setup in the \func{\_start} routine, as described in \autoref{sec:before-main}.

\subsection{Minimal Rust program to boot}
\label{ssec:minimal_rust_program_to_boot}

There are a few modification which have to be applied to the default `Hello World' program in order to get it to boot in an embedded environment.
Let us first revisit the canonical version from \autoref{sub:the_rust_programming_language} given in \autoref{lst:rust-hello-world}, we can see that all we have to define is the \code{main} function.

\begin{listing}[H]
  \begin{minted}{rust}
fn main() {
  println!("Hello, World!");
}
  \end{minted}
  \caption{Standard `Hello World' in {\rust}.}
  \label{lst:rust-hello-world}
\end{listing}

A {\rust} program will by default include \gls{rsl} automatically.
As explained in \autoref{sec:rsl}, this library is not usable in an embedded program, and the changes to the `Hello World' program mostly deals with removing the \gls{rsl}.

The embedded `Hello World' program, as given in \autoref{lst:embedded-rust-hello-world}, does not print out the ``Hello, World!'' text because this would have required additional setup, and we are only concerned with the boot process in this example.
In addition to removing the \gls{rsl}, the {\main} function is \emph{exported} to be callable from the {\C} runtime.

\begin{listing}[H]
  \begin{minted}[linenos]{rust}
// Annotations
#![no_std]
#![no_main]
#![feature(no_std, core, lang_items)]

// Crate import
extern crate core;

// Define main function
#[no_mangle]
pub extern "C" fn main() {
  loop {}
}

// Define three functions which are needed by core but defined in std
#[lang="stack_exhausted"] extern fn stack_executed() {}
#[lang="eh_personality"] extern fn eh_personality() {}
#[lang="panic_fmt"] fn panic_fmt() -> ! { loop {} }
\end{minted}
\caption{Embedded `Hello World' in {\rust}.}
\label{lst:embedded-rust-hello-world}
\end{listing}

\attrib{\#\![no\_std]} on line 2 in \autoref{lst:embedded-rust-hello-world} tells the {\rust} compiler not to include the standard library.

Line 3 must be analyzed in conjunction with lines 10 and 11.
Firstly we must guarantee that the function can be called by the \func{\_start} function.
This is done by defining the {\main} function to be a publicly exported symbol denoted by \code{pub extern}.
The second change is to ensure that the function is callable by a {\C} function.
\code{extern} makes this possible by making the function use the {\C} \gls{abi}.
The last thing is to disable the {\rust} name mangling so that the {\C} code can refere to the function by the unmangled name {\main}.
Now that the {\main} function is altered to be callable by C, the function does not resemble the function the rust compiler expects to find.
Therefore we have to tell the compiler that the program does not contain a {\main} function, but that this is ok, hence the \attrib{\#\![no\_main]} on line 3.

The last three lines are a complication due to error handling in {\rust}.
These functions are used by the core library, but implemented in the standard library.
Since we are not using \gls{rsl} in this example we have to define the functions ourselves, the implementation here just ignores all error handling.

\subsection{Storage qualifiers}

As described in \autoref{sec:back:startup} the startup procedure initializes all global variables.
In this section we look at how each storage qualifier,  applicable to global variables in {\rust}, map to different sections of the {\elf} binary.

\begin{listing}[H]
\begin{minted}[linenos]{rust}
const      RUST_CONST_ZERO: u32 = 0;      // not allocated
const      RUST_CONST: u32 = 0xFEED;      // not allocated
static     RUST_STATIC_ZERO: u32 = 0;     // .text
static     RUST_STATIC: u32 = 0xDEAD;     // .text
static mut RUST_STATIC_MUT_ZERO: u32 = 0; // .bss
static mut RUST_STATIC_MUT: u32 = 0xBEEF; // .data

pub extern fn main() { /* Use the variables */ }
\end{minted}
\caption{{\rust} static initialization.}
\label{lst:rust-static-init}
\end{listing}

In \autoref{lst:rust-static-init}, the three different types of declaring globals in {\rust} are shown.
{\rust} divides between two types of global declarations, constants and statics.

A constant declaration, shown in lines 1 and 2 of \autoref{lst:rust-static-init}, represents a value.
There is no need to allocate memory for globals declared as \code{const} as the values can be directly inserted where they are used by the compiler.

The static globals are immutable by default but can be made mutable by the \code{mut} keyword.
The variables on line 3 and 4 are declared to be \code{static}.
As these are immutable they are allocated in the read-only section called \elfsec{.text}.

On line 5 and 6 the declarations are marked with \code{static mut}.
Here we see that the zero initializes variable is assigned to the \elfsec{.bss} section in the {\elf} file.
On line 6 we have a non-zero value that has to be stored in Flash memory prior to execution and is copied to RAM in the \func{ResetHandler}.

\subsection{Bootstrapping startup}
\label{sec:startup}

The {\rust} program that was presented earlier in \autoref{lst:embedded-rust-hello-world} is quite obscure.
For this reason the \lib{startup} \footnote{\url{https://github.com/RustyGecko/startup/}} library was developed in order to minimize the effort of making a new {\rust} program for the {\gecko}.
This library makes the startup process more intuitive and hides all the details that was presented in \autoref{ssec:minimal_rust_program_to_boot}.

A minimal `Hello World' program using the \lib{startup} crate is given in \autoref{lst:embedded-rust-bootstrapped}.
We still have to annotate the program with \code{\#[no\_std]}, but the \code{main} function is now much more similar to the one that was presented in the start of this section, due to the inclusion of the \lib{startup} library.

\begin{listing}[H]
  \begin{minted}{rust}
#![no_std]
#![feature(no_std)]
extern crate startup;

fn main() {
  loop {}
}
  \end{minted}
  \caption{Embedded Hello World using startup library to bootstrap}
  \label{lst:embedded-rust-bootstrapped}
\end{listing}

\subsection{Handling Interrupts in Rust}
% \todo{This sections might be moved}
\label{sec:impl:handling-interrupts}

Interrupts are an integral part of embedded programs and having a native way of handling the interrupts provides a great benefit and flexibility for a programming language.

To handle interrupts on the {\gecko} one have to register a function in the \emph{interrupt vector}.
This vector is defined in the \textbf{startup} file provided by Silicon Labs, and is simply a list of function pointers defined in the \textbf{.isr\_vector} section of the \textbf{elf} binary.
This is the first section in the \textbf{text} segment of the binary which ensures that it is located at memory address \mem{0x0} when the microcontroller starts executing.
When an interrupt occurs, the microcontroller will inspect the interrupt vector and find the address of the handler function for the interrupt which occurred.
Both the interrupt vector and the default interrupt handlers are defined in the \textbf{startup} file for the {\gecko}.
The default implementations are simply infinite loops defined as weak symbols.
These weak symbols allows the programmer to redefine the symbol elsewhere in the code in order to override this default implementation.

\autoref{lst:c:irq} shows how the \func{SysTick\_Handler} can be overridden in {\C}.
The {\gecko} can be triggered to cause interrupts that occur at a timely basis, e.g. once every second.
This function will then be called when these interrupts occur.

\begin{listing}[H]
  \begin{minted}{c}
void SysTick_Handler(void) { /* Handler code */ }
  \end{minted}
  \caption{Defining the SysTick Interrupt Handler in {\C}.}
  \label{lst:c:irq}
\end{listing}

Defining an interrupt handler in {\rust} is just as easy as in {\C} because of the focus on interoperability with {\C} code.
In {\rust}, a function can easly be defined to use the {\C} \gls{abi} required to be called as an interrupt handler.
\autoref{lst:rust:irq} shows how to override the same \func{SysTick\_Handler} function in {\rust}.

\begin{listing}[H]
  \begin{minted}{rust}
#[no_mangle] pub extern fn SysTick_Handlder() { /* Handler code */ }
  \end{minted}
  \caption{SysTick Interrupt Handler in {\rust}.}
  \label{lst:rust:irq}
\end{listing}


\corechapter{Rust Embedded Library}{rel}{%
  The \gls{rel} module defines the subset of the standard {\rust} library which is applicable for embedded applications.
  This module builds on the foundation layed out by the startup module and is used by the Bindings and Application Layer modules of the {\rg} platform.
}
% !TEX root = ../main.tex

\section{The Core Library}
\label{sec:rust:core}

As described in \autoref{sec:rcl}, the \gls{rcl} defines the \emph{core} functionality of the {\rust} language.
The \gls{rcl} does not have any library dependencies, but in order to use the library without \gls{rsl} a few definitions are needed.
These definitions are given in \autoref{tab:core:definitions}.

\begin{table}[H]
  \centering
  \begin{tabular}{l | l}
    \textbf{Functions} & \textbf{Description} \\
    \hline
    \code{memcpy, memcmp, memset} & Basic memory management \\
    \code{rust\_begin\_unwind}    & Handles panicking \\
    \hline
  \end{tabular}
  \caption{External dependencies of \gls{rcl}}
  \label{tab:core:definitions}
\end{table}

The memory management functions given in \autoref{tab:core:definitions} are provided by \lib{newlib} and are exposed through the \lib{startup} library described in \autoref{sec:startup}.
These functions provide the basic memory management that is needed in order to utilize the parts of \gls{rsl} that defines dynamically allocated data structures.

\concept{Panicking} is {\rust}'s way of unwinding the currently executing thread, ultimately resulting in the thread being terminated.
A panic in {\rust} can happen when e.g. an array is indexed out of bounds, which causes the \code{rust\_begin\_unwind} function to be called.
The \code{rust\_begin\_unwind} is also defined in \lib{startup}, but the implementation is only an infinite loop to aid debugging.
In contrast, the definition of \code{rust\_begin\_unwind} given in \gls{rsl} will abort the program and print an error message.

\section{The Allocation Library}
\label{sec:rust:allocation}

Heap allocation is introduced in a library called \lib{alloc}.
The library defines the managed pointer, \code{Box}, which is {\rust}'s main means of allocating memory on the heap.
Also, the allocation library defines the types \code{Rc} and \code{Arc}, which are {\rust}'s \emph{reference counted} and \emph{atomically reference counted} heap pointers.

\begin{listing}[H]
  \begin{rustcode}
fn rust_allocate(usize, usize) -> *mut u8;
fn rust_deallocate(*mut u8, usize, usize);
fn rust_reallocate(*mut u8, usize, usize, usize) -> *mut u8;
  \end{rustcode}
  \caption{External dependencies of the \lib{alloc} library}
  \label{tab:alloc:external-funcs}
\end{listing}

The allocation library is by default dependent on \lib{libc}, but this dependency can be broken by supplying the \flag{--cfg feature="external\_funcs"} flag to the compilation process.
When breaking this dependency, the allocation library requires the functions in \autoref{tab:alloc:external-funcs} to be defined elsewhere.
Note that these functions map directly to the \code{alloc}, \code{dealloc}, and \code{realloc} functions, which are all part of \lib{libc} and {\newlib}.
This design makes it easy to include the {\lib{alloc}} library for new platforms like {\rg}.

\section{The Collection Library}

The {\rust} \lib{collections} library provides general purpose data structures.
Out of these data structures the \code{Vector} (a growable heap allocated list) and the \code{String} (heap-allocated mutable strings) are the most notable.

As one would expect, the \lib{collections} library depends on the \lib{alloc} library, as it needs to allocate memory on the heap.
\texttt{collections} also depends on the \texttt{unicode} library because all strings in {\rust} are UTF-8 encoded.

\section{The Rust Embedded Library}
\label{sec:rel}

The libraries mentioned in the previous sections provides core language constructs and dynamic heap allocation.
Together they form a strong foundation for new {\rust} programs, without depending on an \gls{os}.
We have composed these libraries into what we refer to as the \gls{rel}, and the dependencies of these libraries are is shown \autoref{fig:rust:rel}.

\begin{figure}[H]
  \begin{center}
    \includegraphics[scale=0.3]{figures/background/rust/embedded-rust-lib.png}
  \end{center}
  \caption{Rust Embedded Library}
  \label{fig:rust:rel}
\end{figure}

It is important to note that \gls{rel} is just a way to provide a well-defined definition of the {\rust} language for an embedded system.
\gls{rel} is, unlike \gls{rsl}, not built as a facade, it is not \emph{implemented} by us at all.
However, the libraries that make up \gls{rel} needs to be conditionally compiled for the Cortex-M3 architecture, and this described in \autoref{sec:build_system}.



\corechapter{Binding Libraries}{bindings}{%
  The Bindings module includes the peripheral libraries provided by the \gls{mcu} vendor SiliconLabs, and the architecture designer ARM.
  In order to make use of these libraries in {\rust} we developed binding libraries to expose the underlying {\C} implementation to the {\rust} language.
}
\section{Object-oriented Embedded Programming}
\label{sec:impl:oo}

The interface of many of the modules defined in {\emlib} resembles that of objects found in object-oriented programming.
In its essence, object-oriented programming focuses on organizing a computer program by looking at the data the program operates on.
This is done by grouping related data into objects and defining methods that operate on the data contained within the objects.

The paradigm's essential concept can be applied to embedded {\C} programming, even though the language itself does not directly define any language features to aid the design.
In this section, we look at how control over the memory layout of objects, and static dispatch, can be used to enable the object-oriented paradigm in conjunction with \gls{mmio} in embedded programming.
We use the memory layout of a memory-mapped \gls{adc} as an example to see how this peripheral can be represented as an object.

Static dispatch, as opposed to dynamic dispatch, is the mechanism in which the function to be called can be decided statically by the compiler, and a \code{call} instruction to the function can be inserted into the code directly.
Dynamic dispatch, on the other hand, requires extra runtime information about the function to be called, which adds an additional layer of indirection to the function call.
% Dynamic dispatch, on the other hand, requires runtime information which requires the call to the function to be made through an extra layer of indirection.


\subsection{Memory Mapped I/O}
\label{ssec:memory_mapped_io}

\gls{mmio} is a method for interfacing with peripheral devices in a computer system.
The method entails connecting the control registers of hardware devices to the same address bus as \gls{ram}.
This results in a programming model where the programmer can use common memory operations to control the devices.

Let us consider the \gls{adc} on the {\gecko}.
The \gls{adc} converts an analog signal to a digital representation.
The base address of the \gls{adc} on the {\gecko} is memory mapped to the location \mem{0x40002000} in the memory space.
This means that writing to a pointer that points to this address will write to the control registers in the \gls{adc} device.

\begin{figure}[H]
  \centering
  \begin{tabular}{l|l|l|}
    \textbf{Location} & \textbf{Offset} & \textbf{Name} \\
    \hline
    &...&...\\
    \hline
    \hline
    \mem{0x40002000} & 0x0 & CTRL\\
    \hline
    & \mem{0x4} & CMD\\
    \hline
    &...&...\\
    \hline
    & \mem{0xC} & SINGLECTRL\\
    \hline
    &...&...\\
    \hline
    & \mem{0x24} & SINGLEDATA\\
    \hline
    &...&...\\
    \hline
    \hline
    &&\\
  \end{tabular}
  \caption{Subsection of ADC0 Memory map for the {\gecko}}
  \label{fig:back:memorymapped}
\end{figure}

\autoref{fig:back:memorymapped} shows a subsection of \gls{ram} that contains the \gls{adc} control register.
Only the relevant registers for our discussion in included in the figure.
It shows the control register that is used when performing a single \gls{adc} conversion.
Note that it only includes the register needed for this kind of conversion.
The CTRL register is used to initialize the hardware device before performing a conversion, and the CMD register is used to issue direct commands to the device like \emph{stop} and \emph{start}.
We see that the CTRL register is at offset \mem{0x0} from the base address of the \gls{adc} and that the CMD register is at an offset of \mem{0x4} bytes.
The two registers, SINGLECTRL and SINGLEDATA, are in order to initialize the single conversion and read the results of a conversion, respectively.

\subsection{Memory Layout of Objects}

The traditional memory layout of an object in an object-oriented language is an implementation detail.
This is because the fields of the object might have different sizes, and optimizations can rearrange the memory layout to optimize for size.
The layout is also an implementation detail of {\rust} for the same reasons, but by annotating a struct with \attrib{\#[repr(C)]}, it will ensure that it is compatible with {\C}'s \gls{ffi}.
Objects in a language like {\Java} also includes a tag field at the base of the object as a reference to the class of the object in order to provide dynamic dispatch.

\begin{listing}[H]
  \centering
  \begin{minipage}{0.26\textwidth}
  \begin{listing}
    \begin{javacode}
class ADC {
  int CTRL;
  int CMD;
  // ...
  int SINGLECTRL;
  // ...
  int SINGLEDATA;
  // ...
}
    \end{javacode}
  \end{listing}
  \end{minipage}
  \hfill
  \begin{minipage}{0.27\textwidth}
  \begin{listing}
    \begin{rustcode}
#[repr(C)]
struct ADC {
  CTRL: u32,
  CMD: u32,
  // ...
  SINGLECTRL: u32,
  // ...
  SINGLEDATA: u32,
  // ...
}
    \end{rustcode}
  \end{listing}
  \end{minipage}
  \hfill
  \begin{minipage}{0.33\textwidth}
  \begin{listing}
    \begin{ccode}
typedef struct {
  uint32_t CTRL;
  uint32_t CMD;
  // ...
  uint32_t SINGLECTRL;
  // ...
  uint32_t SINGLEDATA;
  // ...
} ADC;
    \end{ccode}
  \end{listing}
  \end{minipage}

  \caption{Definition of an \gls{adc} in {\Java}, {\rust}, and {\C}}
  \label{lst:back:adc-objects}
\end{listing}

In {\C}, where classes and objects are not part of the language, structs are used to create the representations for objects.
By using structs, the programmer has full control over the layout of the object in memory.
The object-oriented concepts used for \gls{mmio} uses static dispatch, and the structs do not include tag fields or references to virtual tables.
\autoref{lst:back:adc-objects} shows how to define a {\Java} class, and {\rust} and {\C} structs for the \gls{adc} on the {\gecko}.
The memory layout of these objects is given in \autoref{fig:back:memlayout}.

\begin{figure}[H]

  \centering
  \begin{subfigure}{0.31\textwidth}
    \begin{tabular}{|l|l|}
      \hline
      0x0&Object tag \\ \hline
      0x4&CTRL       \\ \hline
      0x8&CMD        \\ \hline
      ...&...        \\ \hline
      0x10&SINGLECTRL\\ \hline
      ...&...        \\ \hline
      0x28&SINGLEDATA\\ \hline
      ...&...        \\ \hline
    \end{tabular}
    \caption{{\Java}}
    \label{fig:back:memlayout:java}
  \end{subfigure}
  \hfill
  \begin{subfigure}{0.31\textwidth}
    \begin{tabular}{|l|l|}
      \hline
      0x0&CTRL       \\ \hline
      0x4&CMD        \\ \hline
      ...&...        \\ \hline
      0xC&SINGLECTRL \\ \hline
      ...&...        \\ \hline
      0x24&SINGLEDATA\\ \hline
      ...&...        \\ \hline
    \end{tabular}
    \caption{\rust}
    \label{fig:back:memlayout:rust}
  \end{subfigure}
  \hfill
  \begin{subfigure}{0.31\textwidth}
    \begin{tabular}{|l|l|}
      \hline
      0x0&CTRL       \\ \hline
      0x4&CMD        \\ \hline
      ...&...        \\ \hline
      0xC&SINGLECTRL \\ \hline
      ...&...        \\ \hline
      0x24&SINGLEDATA\\ \hline
      ...&...        \\ \hline
    \end{tabular}
        \caption{{\C}}
    \label{fig:back:memlayout:c}
  \end{subfigure}
  \caption{Memory layout of objects}
  \label{fig:back:memlayout}

\end{figure}

By comparing \autoref{fig:back:memorymapped} and \autoref{fig:back:memlayout}, we see that the memory layout of a struct defined in {\rust} and {\C} has the exact same layout as the memory mapped control register of the \gls{adc}.
This suggests that, if a pointer to the \gls{mmio} device is considered as a reference to an \gls{adc} object, the object-oriented pattern can be used to directly interface with the \gls{mmio}.

The layout of the {\Java} object in \autoref{fig:back:memlayout:java} could imply that the same analysis can be applied by adding an offset, equal to the size of the object tag, to the reference.
This is not the case as this would map the object tag to the base address of the \gls{adc} minus 4 bytes.
This location is, in the case for the {\gecko}, an unmapped memory section used to add padding between the \gls{adc} and the previous \gls{mmio}.
{\Java} uses this object tag to store a reference used to dynamically dispatch method calls to the object.
Moreover, using the reference in place of a regular {\Java} object would cause the method dispatch mechanism to fail.

\subsection{Adding Object Functionality}

This section shows how we add functionality called \emph{methods} to the \gls{mmio} objects.
Both {\C} and {\rust} uses static dispatch.
This ensures that {\C} and {\rust} provide the same zero-cost abstractions when interacting with the \gls{mmio}.

\subsubsection{Static Dispatch in C}

Implementing objects with static dispatch is a straight forward process in {\C}.
Here, we define a function which takes a reference to the object as the first parameter.
The function then uses the object reference in the same manner as the implicit \var{this} parameter in conventional object-oriented languages such as {\Java}.

\begin{listing}[H]
  \centering
  \begin{minipage}{0.47\textwidth}
  \begin{listing}
      \begin{ccode}
// ADC Member function with
// explicit object reference
uint32_t ADC_DataSingleGet(
           ADC *const adc) {
  // The adc pointer is used as a
  // reference to the this object
  return adc.SINGLEDATA;
}

void main() {
  // The next section describes
  // how to instantiate MMIOs
  ADC adc;
  // Call the member function
  // passing in an explicit
  // reference to the object
  ADC_DataSingleGet(&adc);
}
      \end{ccode}
  \end{listing}
  \end{minipage}
  \hfill
  \begin{minipage}{0.47\textwidth}
  \begin{listing}
      \begin{rustcode}
impl Adc {
  // Rust lets the programmer
  // specify how to accept the
  // object when invoked with
  // the dot notation
  pub fn data_single_get(&self)
  -> u32 { // self is a reference
    // to the ADC MMIO
    self.SINGLEDATA
  }
}

fn main() {
  // Instantiation of MMIOs is
  // handled in the next section
  let adc = Adc;
  // The Rust compiler issues
  // a static call to the
  // member method and passes in
  // the reference to the MMIO
  adc.data_single_get();
}
      \end{rustcode}
  \end{listing}
  \end{minipage}
  \caption{Member methods for {\C} and {\rust}, respectively.}
  \label{lst:back:adc:get}

\end{listing}

\autoref{lst:back:adc:get} shows how to define a getter function for the \gls{adc} single conversion register, as a member method using an object-oriented pattern.
We use the \keyword{impl} block to define the same behavior in {\rust}, but the methods are called with the dot notation known from object-oriented languages.

\subsection{Instantiating a MMIO object}

Now that we have shown that \glspl{mmio} can be represented as objects defined by structs, we consider how to instantiate them.
Usually, an object in the object-oriented paradigm is created with a constructor and deallocated with a destructor.
The constructor is responsible for allocating the object and initializing the fields with values.
Analogously, the destructor is responsible for deallocating the object and any other member objects that it owns.
\gls{mmio} devices have a fixed position in the memory and do not need to be allocated, and they also generally do not have any owned members.
Therefore the constructor-destructor pattern is not applicable for \gls{mmio}s, but we still need to instantiate the variable that holds the reference to the \gls{mmio} and cast it to the desired type.
\autoref{fig:oo:instantiate} shows how to instantiate a \gls{mmio} as an object in both {\C} and {\rust}.

\begin{listing}[H]
  \begin{minipage}{0.46\textwidth}
  \begin{listing}
    \begin{rustcode}
const ADC0_BASE: *mut Adc
      = 0x40002000 as *mut Adc;

fn main() {
  let adc0 = unsafe {
    ADC0_BASE.as_mut().unwrap()
  };
}
    \end{rustcode}
    \caption{Rust}
  \end{listing}
  \end{minipage}
  \hfill
  \begin{minipage}{0.45\textwidth}
  \begin{listing}
    \begin{ccode}
#define ADC0_BASE 0x40002000

void main() {
  ADC* adc0 = (ADC*)ADC0_BASE;
}
    \end{ccode}
    \caption{{\C}}
  \end{listing}
  \end{minipage}
  \caption{Instantiating a \gls{mmio}}
  \label{fig:oo:instantiate}
\end{listing}

%For a further discussion of this pattern in {\rust}, see \autoref{sec:res:aliasing-mmios}.

% !TEX root = ../main.tex

\section{Bindings for emlib}
\label{sub:interfacing_with_emlib}

The \gls{ffi} available in {\rust} has been used to interface with Silabs' emlib.
This way, we have been able to create thin wrappers around the \gls{api} for the different peripherals that we have used in the project, without porting the core logic itself.
The following sections explains the process of defining and implementing the \gls{ffi} in {\rust} that is used to access and control the peripherals on the {\gecko}.
We will start by showing an example of what it is like to use Silab's emlib (with C), and then move on to what it looks like to use the bindings in {\rust}.

\subsection{Defining the Bindings}

The \code{Timer} peripheral \cite{an0014_timer} works as a good example to demonstrate what the {\rust} bindings look like.
The module is fairly small, it mostly exposes functions to set up and initialize four different timers that can be used for up, down, up/down, and input- and output-capture.
The program shown in \autoref{lst:timer_program_c} is an example of initializing the \code{Timer0} peripheral on the {\gecko}, note that this is not a complete working example, it only shows the most important parts required to use the Timer module.
First, in order to use a peripheral, its clock need to be enabled.
Every external clock are disabled by default \todo{Is this true? Is every clock disabled?} in order for the peripherals not to draw any power, by enabling the clock the \gls{mcu} will get access to control it.
Then an initialization structure for the Timer module is acquired, this structure has fields to configure many different properties of the Timer, in the same way as described in \autoref{ssec:memory_mapped_io}.

\begin{listing}[h]
\begin{minted}{c}
// Enable clock for TIMER0 module
CMU_ClockEnable(cmuClock_TIMER0, true);
// Select TIMER0 parameters
TIMER_Init_TypeDef timerInit = TIMER_INIT_DEFAULT;
// Enable overflow interrupt
TIMER_IntEnable(TIMER0, TIMER_IF_OF);
// Enable TIMER0 interrupt vector in NVIC
NVIC_EnableIRQ(TIMER0_IRQn);
// Set TIMER Top value
TIMER_TopSet(TIMER0, TOP);
// Configure TIMER
TIMER_Init(TIMER0, &timerInit);
\end{minted}
\caption{Initializing a Timer in C}
\label{lst:timer_program_c}
\end{listing}

The next lines enables interrupts for the Timer, and the NVIC interrupt vector is set up to call the function shown in \autoref{lst:timer_interrupt_handler} every time an interrupt it triggered by the Timer.
All this function does is to clear the interrupt signal and toggle the value of an LED.
We can imagine an application where the Timer is configured to trigger an interrupt every minute to toggle the LED, and in between the interrupts the \gls{mcu} can be put to sleep in order to save power.
Interrupts like this are an important part of programming the {\gecko}.
They can be used for an asynchronous programming model where the application is defined by the code in the different interrupt handlers, and like in the example above, the \gls{mcu} can be put to sleep in between interrupts.

\begin{listing}[h]
\begin{minted}{c}
void TIMER0_IRQHandler(void) {
  // Clear flag for TIMER0 overflow interrupt
  TIMER_IntClear(TIMER0, TIMER_IF_OF);
  // Toggle LED ON/OFF
  GPIO_PinOutToggle(LED_PORT, LED_PIN);
}
\end{minted}
\caption{Timer Interrupt Handler}
\label{lst:timer_interrupt_handler}
\end{listing}

The equivalent program written in {\rust} is shown in \autoref{lst:timer_program_rust}.
Semantically they are the same, but the usage differs slightly, which is natural since we are using a higher level programming language.
Instead of calling functions that are included through a {\C} header file, we are calling functions that are available through a {\rust} module.
For example, the \code{Clock} enum is part of the \code{cmu} module, and the \func{enable\_irq} function is part of the \code{nvic} module.
This modularization of peripherals can help to make the code less verbose by partially including modules.
It is also worth to notice the difference between how the \code{Timer0} structure can be treated like an object with its own member methods in {\rust}, instead of being passed as the first parameter to every function that requires it, like in {\C}.

\begin{listing}[h]
\begin{minted}{rust}
// Enable clock for TIMER0 module
cmu::clock_enable(cmu::Clock::TIMER0, true);
// Select TIMER0 parameters
let timer_init = Default::default();
// Enable overflow interrupt
let timer0 = timer::Timer::timer0();
timer0.int_enable(timer::TIMER_IF_OF);
// Enable TIMER0 interrupt vector in NVIC
nvic::enable_irq(nvic::IRQn::TIMER0);
// Set TIMER Top value
timer0.top_set(TOP);
// Configure TIMER
timer0.init(&timer_init);
\end{minted}
\caption{Initializing a Timer in {\rust}}
\label{lst:timer_program_rust}
\end{listing}

\subsection{Exposing Static Inline Functions to Rrust}

In order to work with structures and enums originally defined in C, we had to redefine them in {\rust} and mark them with \attrib{\#[repr(C)]} so that {\rust} can guarantee that the data-elements are {\C} compatible.
The header files in the peripheral \gls{api} also defines many functions as \code{static inline}, which only make the functions accessible by including the header file.
Since it is not possible to include {\C} header files directly in {\rust}, we had to expose these functions through one extra layer of {\C} code.
As an example, the \func{TIMER\_IntEnable} function is defined as \code{static inline} in \file{em\_timer.h}, in order to call this function through {\rust} we had to expose it through the file \file{timer.c}, as shown in \autoref{lst:exposing_static_inline}.

\begin{listing}[h]
\begin{minted}{c}
#include "en_timer.h"

void STATIC_INLINE_TIMER_IntEnable(TIMER_TypeDef *timer,
                                   uint32_t flags) {
    TIMER_IntEnable(timer, flags);
}
\end{minted}
\caption{Exposing a \code{static inline} function to {\rust}.}
\label{lst:exposing_static_inline}
\end{listing}

In the {\rust} module definition of Timer, the function has to be made available through the \code{extern} block shown in \autoref{lst:rust_ffi_example}.
As described in \autoref{ssub:unsafe_code}, every function available through the \gls{ffi} are considered {\unsafe} because {\rust} knows nothing about the function, other than its parameters and its return value.
Thus, in order to make it practical to use the library in a seemingly safe manner, we wrap the calls to the foreign functions in an {\unsafe} block, in the respective function defined in {\rust}.

\begin{listing}[h]
\begin{minted}{rust}
impl Timer {
    pub fn int_enable(&self, flags: u32) {
        unsafe { STATIC_INLINE_TIMER_IntEnable(self, flags)}
    }
}

extern {
    fn STATIC_INLINE_TIMER_IntEnable(timer: &Timer, flags: u32);
}
\end{minted}
\caption{Defining and using a function through the {\rust} \gls{ffi}.}
\label{lst:rust_ffi_example}
\end{listing}

If we compare the call-stacks between calling the \code{timer0.int\_enable} function in {\rust} and calling the \func{TIMER\_IntEnable} function in C, we can see that every function call through the \gls{ffi} requires \emph{two} extra function calls.
These are simple wrappers which require extra unconditional jumps in the code, and performance-wise it is a very unnecessary overhead to have one to two extra stack frames allocated for \emph{every} function call through the \gls{ffi}.
By compiling the code with optimization, {\rust} will get rid of the overhead introduced by the function that wraps around the call to the foreign function.
Additionally, by enabling \concept{link-time-optimizations} during the compilations, LLVM will remove the overhead entirely, which results in the same performance and similar call-stacks between {\C} and {\rust}.
This is a working example of one of {\rust}'s many zero-cost abstractions.

\subsection{Naming Conventions}

We have tried to keep \emlib's naming convention across the layer of bindings.
This makes it easy for anyone reading either the C- or the {\rust}-code to translate and understand the code between the two languages.
Since every constant, enum-field, or struct-name is directly accessible by name in C, if the corresponding header file is included, it is important that names of such fields can be separated from each other and do not cause a naming collision.

\begin{listing}[h]
\begin{minted}{c}
typedef enum {
  timerCCModeOff     = _TIMER_CC_CTRL_MODE_OFF,
  timerCCModeCapture = _TIMER_CC_CTRL_MODE_INPUTCAPTURE,
  // ...
} TIMER_CCMode_TypeDef;
\end{minted}
\caption{Part of a Timer enum defined in C.}
\label{lst:enum_naming_c}
\end{listing}

As an example, two fields of an enum from \file{em\_timer.h} is shown in \autoref{lst:enum_naming_c}.
From each field in the enum we can extract 1) its module name \file{timer}, 2) its typedef name \code{CCMode} and 3) its field name \code{Off} or \code{Capture}.
\rust allows us to keep the same naming convention at the same time as utilizing its modularity.
\autoref{lst:enum_naming_rust} shows the enum ported to {\rust}, where both the module name and the typedef name has been left out, and only the field names have remained.
However, the naming convention remains the same when the fields are used, e.g. the expression ``\code{let mode = timer::CCMode::Capture;}'' in {\rust} shows the similarity with the equivalent expression in C: ``\code{int mode = timerCCModeCapture;}''.

\begin{listing}[h]
\begin{minted}{rust}
pub enum CCMode {
  Off     = _TIMER_CC_CTRL_MODE_OFF,
  Capture = _TIMER_CC_CTRL_MODE_INPUTCAPTURE,
  // ...
}
\end{minted}
\caption{The enum ported to {\rust}.}
\label{lst:enum_naming_rust}
\end{listing}

\subsection{The Bindings Library}
\label{ssub:the_bindings_library}

We have written many example applications in Rust that are based on bindings for many of the \texttt{Gecko}'s peripherals, these examples have been a driving factor for defining new bindings.
As an example, if we were to need some functionality for the \gls{adc}, we would define these bindings alongside the development of the examples.
In this way, the bindings have been defined incrementally during the development of either the examples that demonstrate our libraries, or the applications described in \todo{Refer to sections about circle-game and sensor-tracker}.

It has not been a priority of this project to fully support all of the \texttt{Gecko}'s peripherals or provide bindings for the entire \texttt{emlib}.
The bindings have been developed because we have needed functionality from the peripherals, not the other way around.
\autoref{tab:emlib_peripheral_bindings} summarizes a number of modules from \texttt{emlib} that we have written bindings for, and why we wanted these bindings.
A list of examples that demonstrates how the bindings work is shown in \autoref{tab:emlib_examples}, these are either written from scratch, or directly ported from \texttt{emlib}'s examples.

\begin{table}[H]
  \centering
  \begin{tabular}{r|p{10cm}}
    \textbf{Module} & \textbf{Purpose} \\
    \hline
\texttt{cmu}     &
The \gls{cmu} provides functions to manage the different clocks and oscillators on the \texttt{gecko}.
This module is necessary in order to configure the clocks that are required by other peripherals to function. \\

\texttt{dma}     &
We wanted to use \gls{dma} for the \texttt{sensor-tracker} application because it can be used to transfer data without intercepting the \gls{cpu}. \\

\texttt{ebi}     &
The \gls{ebi} is used in the \texttt{Gecko}'s internal memory map, and simplifies the process of writing data to e.g. the LCD on the \texttt{DK}. \\

\texttt{emu}     &
The \gls{emu} module controls the different energy modes on the \texttt{Gecko}.
We needed this functionality for the \texttt{sensor-tracker} application. \\

\texttt{gpio}    &
The \gls{gpio} was one of the first modules to be ported, it is used extensively throughout the bindings and in the applications. \\

\texttt{lesense} &
The \gls{lesense} can be configured to automatically collect data from multiple sensors, we needed this for the \texttt{sensor-tracker}. \\

\texttt{acmp}    &
The \gls{acmp} was ported alongside \gls{lesense}, it is used to compare two analog signals and tell which one is greater. \\

\texttt{adc}     &
The \gls{adc} was needed by the \texttt{sensor-tracker}, it has been used to get the internal temperature of the \gls{cpu}. \\

\texttt{usart}   &
Primarily we wanted the \gls{usart} for easy debugging and I/O from a computer. \\

\texttt{leuart}  &
The \gls{leuart} was ported a while after the \gls{usart}.
It has the same functionality, but it works with a lower-frequency clock and at requires less energy than the \gls{usart}. \\

\texttt{i2c}     &
The \gls{i2c} has many of the same use cases as the different \gls{uart} types, but it works at lower energy levels.
It is used in by the \texttt{sensor-tracker}. \\

\texttt{rtc}     &
The \gls{rtc} was an easy module to write bindings for, it is used in several examples for timing purposes, as well as in the \texttt{sensor-tracker}. \\

\texttt{timer}   &
The Timer was one of the first modules to be ported, it worked as a proof-of-concept for the design of the bindings, as described earlier in this section. \\

    \hline
  \end{tabular}

  \caption{\texttt{emlib} peripheral bindings.}
  \label{tab:emlib_peripheral_bindings}
\end{table}


\begin{table}[H]
  \centering
  \begin{tabular}{r|p{7cm}|p{2.1cm}}
    \textbf{Example} & \textbf{Purpose} & \textbf{Uses} \\
    \hline
\texttt{buttons\_int}  &
Demonstrates interrupts by lighting a led on the \texttt{STK} when the respective button has been pressed. &
\gls{gpio}, IRQ \\

\texttt{rtc\_blink}    &
Toggle a led with an interval of 2 seconds. &
\gls{rtc}, \gls{gpio} \\

\texttt{energy\_modes} &
Demonstrates the four stages of sleep on the \texttt{Gecko}. &
\gls{emu}, \gls{gpio}, \gls{cmu}, IRQ \\

\texttt{uart}          &
Demonstrates serial communication over \gls{usart} by echoing back every byte it receives. &
\gls{usart}, \gls{gpio}, \gls{cmu} \\

\texttt{leuart}        &
Similar example as \texttt{uart}, but the \gls{cpu} is turned off and the functionality is moved to an interrupt handler instead. &
\gls{leuart}, \gls{cmu}, IRQ, \gls{gpio}, \gls{emu} \\

\texttt{i2c}           &
Sends and receives a data-buffer between two devices that supports \gls{i2c}. &
\gls{i2c}, \gls{gpio}, \gls{cmu}, IRQ \\

\texttt{joystick}      &
Reads analog signals generated by a Joystick that is connected to the \texttt{STK}. &
\gls{adc}, \gls{cmu} \\

\texttt{dma}           &
Transfer a data-buffer from one memory location to another. &
\gls{dma} \\

\texttt{light\_sense}  &
Uses the \texttt{STK}'s light sensor to measure the light intensity, and signal when it goes dark by lighting a LED. &
\gls{lesense}, \gls{acmp}, \gls{gpio}, \gls{rtc}\\

\texttt{boxes}         &
Demonstrates dynamic memory allocation with \texttt{Box}'es, provided by the Rust \texttt{alloc} library. &
\texttt{alloc} \\

\texttt{vec}           &
Demonstrates dynamically allocated strings and vectors from the \texttt{collections} library. &
\texttt{collections} \\

    \hline
  \end{tabular}

  \caption{Examples that demonstrates how the bindings work.}
  \label{tab:emlib_examples}
\end{table}

In addition to writing bindings for

\subsection{Testing}
\label{ssub:testing}

This section describes a small Unit testing framework developed for testing the emlib bindings.

\subsubsection{Why Unit testing}

Early on in the development phase of the emlib bindings we saw the need for a testing framework.
This was provoked by the fact that testing on an embedded system is a time consuming and tedious task.
More often than not you find your self running the code in the debugger inspecting the call stack and arguments to ensure that the bindings are calling the correct functions with the correct arguments.
The fact about arguments has a subtle point to it.

We are working in two statically typed languages here which leads to the compiler statically assuring that the correct types are passed around, but at the language boundries there are no checks for assuring that the memory layout of the datatypes in C and Rust match.
Currently the Rust FFI requires the programmer to redefine the datatypes like structs and enums on the Rust side in order to call into C functions that takes these datatypes as arguments.
This process proved to be too error prone and time consuming to verify manually.

\subsubsection{Framework}

To meet this problem, a featherweight testing framework was developed which enabled this verification to be automated.
The goal of the framework was to initialize the data on the Rust side execute the \gls{ffi} and verify the correct function where called with the exact data as supplied.
To do this we set up a framework to replace the emlib code with statically generated test mocks, execute test cases where the datastructures where setup in Rust code and then verify the correctness in C code.
All of this was executed on the \textbf{Gecko} which reported test failure/success over Uart.

The framework implemented is a small test runner utilizing CMock \cite{web:cmock} and Unity \cite{web:unity} for mocking and assertions respectivly.

An test case for verifying that the \texttt{ADC\_Init} with default argument is working properly is given in \autoref{lst:test:adc}.


\begin{listing}[H]
  \centering
  \begin{minipage}{\textwidth}
  \begin{listing}
    \begin{minted}{rust}
fn test_init_called_with_default() {
  // FFI call to the C function below
  unsafe { adc_expect_init_called_with_default(); }

  let adc0 = adc::Adc::adc0();
  // Call the emlib bindings with an default argument
  adc0.init(&Default::default());
}
    \end{minted}
    \label{lst:test:adc:rust}
    \caption{Rust side of ADC\_Init test}
  \end{listing}
  \end{minipage}

  \begin{minipage}{\textwidth}
  \begin{listing}
    \begin{minted}{c}
void adc_expect_init_called_with_default() {
  static ADC_Init_TypeDef init = ADC_INIT_DEFAULT;
  // setup the expected value on the Mock
  ADC_Init_Expect(ADC0, &init);
}
    \end{minted}
    \label{lst:test:adc:c}
    \caption{C side of ADC\_Init test}
  \end{listing}
  \end{minipage}

  \caption{Test case for ADC\_Init with default values}
  \label{lst:test:adc}
\end{listing}

When using mocking in unit tests the workflow of for the user seems reversed compared to standard unit tests.
First you setup the expected results with by calling the \texttt{ADC\_Init\_Expect} function on the mock as shown in \auroref{lst:test:adc:c}.
This method is in fact called through \gls{ffi} right at the top of the test case in \autoref{lst:test:adc:rust}.
When the expected result is setup the test case goes on to create an \gls{adc} object using the Rust bindings and calling the \texttt{init} function causing the \gls{ffi} library bindings to be executed.
When the test returns, the test runner is responsible for calling a \texttt{Verify} function on the mock object.
This function causes the program to fail and report status over Uart if the expectation set up was not met.

\autoref{fig:test-framework} shows a diagram of the flow between \textit{Test Runner}, \textit{Test Case}, \textit{emlib Bindings} as \gls{CUT} and the \textit{emlib Mock} when executing the testcase above.

\begin{figure}[H]
  \begin{center}
    \includegraphics[scale=0.3]{figures/testframework}
  \end{center}
  \caption{Flowchart for test framework}
  \label{fig:test-framework}
\end{figure}

We see that the two boxes marked with \textit{Test Case} are the pieces of code the user of the framework, presented in \autoref{lst:test:adc}, writes.
The stippled vertical line shows the separation between Rust and C code, all the three calls which cross this line is implemented using the Rust \gls{ffi}.

\subsubsection{Rust libtest}

The Rust programming language contains a testing framework within the standard library.
The reasons for not using it to test the bindings is that we wanted to run the tests on the \textbf{Gecko}.
The rational for this is that the tests are mostly checking that the datatypes used on the Rust and C side of the bindings are compatible.
Therefore we need to use the proper compilers and compiler targets for the platform, verifying that the OSX platform gcc and rust compiler has compatible types does not help in this respect.
Consequently, given that the testing framework in the standard library relies heavily on the Rust standard library RSL as discussed in \autoref{sec:back:lib}, this renders the framework unusable for our embedded platform.

\subsection{Thoughts and Review}

\todo{Does it make sense to have this section here? Should content like this be mover somewhere else (like in the discussion chapter), or is this a good place?}

Writing the bindings for different the peripherals was a tedious work, that required careful review of the \emlib source code in order to correctly port enum- and struct-definitions from {\C} to {\rust}.
Additionally, we had to redefine many constants, like the names of memory-mapped register bit-fields like the ones presented earlier in \autoref{fig:back:memorymapped}, or values calculated from various {\C} macros defined in header files that are used throughout the library.
We had to retrieve the value of the constants by debugging the source code and explicitly look up the value of these constants if they were only implicitly defined in the header files.

Since we have constrained our library to only support the EFM32GG990 devices we chose to manually write the bindings for the library, instead of generating the bindings through some kind of automated process.
There were already a couple of tools available for generating such C-bindings automatically, that could possibly have made the process quicker.
However, we chose not to utilize such tools because of the reasons described below.

\begin{itemize}
    \item It was quick and easy to get started with code for a new peripheral.
    This argument was especially important when the project started out, because we still had no clue of how the project would evolve and what it was going to look like.

    \item It was an advantage to depend on as few third party tools as possible, since both {\rust} and all available libraries would be unstable until the 1.0 release of the language.

    \item We wanted to keep the naming convention of our bindings as similar to \emlib as possible, this would not have been easy to keep consistent with an automated process, partly because there are exceptions where these conventions do not fully hold.
    It is however an interesting problem that would have a higher priority if the library were ever to support more than one EFM32 device.

    \item We could focus on writing bindings for smaller parts of each module separately when we first needed them, which would split the work into smaller work-packages.
\end{itemize}



\corechapter{Build System}{build}{%
  In this chapter we step out of the core software components of the {\rg} platform to present an external but highly important part of the {\rg} platform.
  The {\cargo} package manager is an integral part of the {\rust} ecosystem and facilitates sharing code libraries with ease.
  Throughout this chapter we look at how we evolved the build system over time and ultimately migrated the process over to {\cargo}.
}
\section{Build System}
\label{sec:build_system}

Building the library and the EFM32 executables turned out to be one of the major parts of the work done in this project.
This section describes how building the project have changed over time.
This includes managing dependencies, compiling the {\core} library and \gls{rel}, and the Silicon Labs EFM32 {\emlib}.
We have also utilized a continuous integration system that has helped us to keep the project up to date with the nightly builds of {\rust}, and to make sure that the builds have been consistent across the systems it has been built on.

\subsection{Manually Makefile}
\label{ssub:using_make}

When the project first started out it was based upon the \texttt{armboot} \cite{github:armboot} git repository available on GitHub.
\texttt{armboot} is a small template project for using {\rust} bare metal on a STM32 ARM \glspl{mcu}.
These are, similarly with the EFM32 series, also based on the Cortex-M series of ARM processor cores.

The project got up and running pretty fast based on how \texttt{armboot} was built.
We looked at armboot's Makefile to figure out what flags to pass to {\rustc} in order to make it cross-compile for an ARM target architecture.
The components listed in \autoref{tab:build:components} were identified as the minimal set of files needed to build an executable for the Gecko.

\begin{table}[H]
  \centering
  \begin{tabular}{l l}
    \file{thumbv7m-none-eabi.json} & llvm spec for compiler backend \\ \hline
    \file{zero.rs} & minimal runtime requirements \\ \hline
    \file{blinky.rs} & a program to execute \\ \hline
    \file{efm32gg.ld} & linker script \\ \hline
    \file{startup\_efm32gg.s} & ResetHandler and interrupt vector \\ \hline
    \file{system\_efm32gg.c} & System Clock management \\
  \end{tabular}
  \caption{Source files for building {\rust} for ARM Cortex-M}
  \label{tab:build:components}
\end{table}

\autoref{tab:build:components} lists the files included in the initial successfull build for the Gecko\footnote{GitHub link for reference: https://github.com/havardh/geckoboot.rs/tree/8eb1df2417da0d7016d478de9cd011c98d77c592}.
The build process was contained within a manually developed Makefile.
The compilation process consisted of compiling the \file{blinky.rs}, which blinks the LEDs on the STK, file to assembly by passing the target architecture and assembly output flag to the {\rustc} compiler.
This file was then, along with the \file{startup\_efm32gg.s} and \file{system\_efm32gg.c} files, compiled into object files with {\armgcc} and linked into an executable with the linker script, \file{efm32gg.ld}, using {\armld}.

The build system was then was modified to include the \gls{rcl} library and initial bindings for {\emlib}.
To do this the library was cross-compiled manually for the architecture producing a \file{libcore.rlib} library file.
The initial bindings to {\emlib} were created inside the {\rust} source file and linked with \gls{rcl} before compiling to assembly.
The {\C} implementation of the {\emlib} sources where compiled to object files like the \file{startup\_efm32gg.s} and \file{system\_efm32gg.c} files and all the object files are linked together.

The final revision on the build system using Makefiles was to eliminate the assembly step for compiling the {\rust} sources.
This was done by building an archive file for the object files generated by compiling the {\C} sources with the \cmd{arm-none-eabi-ar}, producing \file{libcompiler-rt.a}.
The archive along with linker arguments was supplied to the {\rustc} compiler, when building the {\rust} source, in order to make the binary.
Making this change resulting in the a build system with three steps as listed in \autoref{tab:build-steps}.

\begin{table}[H]
  \centering
  \begin{tabular}{l l l l}
    & \textbf{Description}  & \textbf{Output} & \textbf{Output Type} \\
    &&&\\
    1. & Building emlib+system+startup & \file{libcompiler-rt.a} & static archive \\
    2. & Building \gls{rcl} & \file{libcore.rlib} & static {\rust} crate \\
    3. & Building Rust source and linking executable & \format{\%.bin} & executable {\elf} file \\
  \end{tabular}
  \caption{Build steps}
  \label{tab:build-steps}
\end{table}

At this stage of the evolution of the build system the {\emlib} bindings were build as part of the {\rust} application code.
Later this was separated out to produce a \file{libemlib.rlib} which complements a \file{libemlib.a} containing the object files for the implementation of {\emlib}.
By doing this also the system and startup code were separated to \file{libstartup.rlib} and \file{libstartup.a}.

\subsection{Transitioning to Cargo}
\label{ssub:transitioning_to_cargo}

It was always a goal to use Cargo for building, distributing, and managing the packages and dependencies that would become part of this project.
An obvious reason for this was to lower the bar for other potential users of the library, and to make our project as standalone as possible, so that it is easier to include and extend it as a part of other potential projects.
By letting {\cargo} handle as much as possible in its build routine, it would automate a lot of the work that every programmer using the library otherwise would have to do manually

When the project first started out it was built by compiling {\rust}'s core library and Silabs' {\emlib} {\C} sources separately, and then linking them with the \gls{ffi} bindings for {\emlib} by hand as described in the previous section.
While this approach worked, built it was far from optimal for a number of reasons:

\begin{itemize}
    \item {\rust} was in active development and many of it's unstable \glspl{api} were going through rapid changes. Ensuring that versions of {\rustc} and the {\rust} source code stayed up to date across different systems was not easy.
    \item Compiling and ensuring that all dependencies were consistent across builds and systems for the bindings were a tedious task. A lot of the troubles concerning this came back to the point above.
    \item Linking dependencies with the library required each system to have set up several different \$PATHS to point to the right directories, what worked for one developer on one system might not have worked for different developer on another system.
    \item The Cargo package manager was developed for exactly these purposes among others.
\end{itemize}

As already described, Cargo is a tool that provides many operations to build {\rust} projects that has a certain project structure.
It is designed to integrate with other existing tools, like GNU Make, which has been important in  building this project.
When the transition to Cargo started, we focused on structuring the main library and its modules into the directory structure described in \autoref{ssub:project_structure}.
By invoking the \cmd{cargo build --verbose} command, it was possible to see the output from what Cargo attempted to build when it failed, and then structure the project accordingly.

A big priority was to to shrink the size of the makefiles that were in the project by making them a part of the standard build process for {\emlib} instead, doing this would help us get a long way of ensuring that the builds done by Cargo could be consistent across systems.
By defining a {\rust} build script and utilizing a {\rust} build-dependency called \prog{gcc} \cite{web:cargo_gcc}, we were able to compile the {\C} sources from Silabs' {\emlib} and link them with our bindings directly as part of the build process.
Note that the \prog{gcc} build-dependency is used as a shell to merely \emph{invoke} the underlying C-compiler, in our case it is used to cross-compile with the {\armgcc} compiler.
By removing the dependency of manually compiling the {\C} sources, it was easier to start to automatically fetch the other dependencies, like the {\core} and \lib{collections} libraries.

Because this project is for a different processor architecture than the system that it is built on, we had to conditionally cross-compile all the standard {\rust} libraries that we wanted to utilize for the ARM Cortex-M3.
We could not utilize the pre-compiled libraries that are already included with {\rustc}, since these only works for the current system architecture.
\todo{Mention this in discussion. Is there a better way of solving this? Currently it fetches a release tarball of the language. It fetches more than it needs to for each library, but it's quicker/easier than using e.g. git to do it. It does however take a little while to compile all the dependencies on smaller machines.}
This problem was solved by implementing a new Cargo build-dependency, called \texttt{rust-src} \cite{github:rust_src}, whose purpose is to download the entire {\rust} source code that is compliant with the instance of {\rustc} that is \emph{currently} compiling the library.
By making it a task for each build to fetch its own source code, we were guaranteed that the dependencies we used for the project would always compile, independent of the current instance of {\rustc} that was installed on the system.
The crates that we have fetched from {\rust}'s standard library that make up what we call \gls{rel} are already described in \autoref{sec:rel}, but they are also shown in \autoref{tab:compiled_libraries} for the sake of completeness.
% The crates\autoref{tab:compiled_libraries} summarizes the different crates that we have fetched from {\rust}'s standard library in order to use them with our project.
% \todo{May not need this table here. Need to synchronize with the section that presents REL and such..}

\todo{We need to write about the \lib{rlibc} crate, and the need to define our own \func{memmove} and \func{memalign} functions in order to use alloc (Box, Vector, ...)}

\begin{table}[ht]
\begin{center}
\begin{tabular}{r|p{8cm}}
\textbf{\rust library} & \textbf{Purpose} \\
\hline
core        & {\rust}'s core library that declares basic types. \\
libc        & Types to use with {\rust}'s \gls{ffi}. \\
alloc       & Allows for heap-allocated variables. \\
collections & Provides common collections like dynamically allocated Strings and Vectors. \\
unicode     & Required by collections for e.g. Strings. \\
rand        & Generate random values. \\
\hline
\end{tabular}
\caption{{\rust} libraries conditionally compiled for the Cortex-M3 architecture.}
\label{tab:compiled_libraries}
\end{center}
\end{table}

By design Cargo only supported passing two flags further on to {\rustc}, those were \flag{-L} and \flag{-l}, their purpose is to tell {\rustc} to link with an external library by looking in a directory (specified with the \flag{-L} flag), for a library with the specified name (specified with the \flag{-l} flag).
The last step in the build process involved linking {\emlib} and the other libraries with an actual executable for the Cortex-M3.
This was not possible to do with Cargo since it required us to pass a couple of extra linker-flags further on to {\rustc}.
The flags were needed by {\rustc} in order to tell it to link with an external library for a different architecture and to include a separate linker-script that took care of booting up the executable on this architecture.

Another issue that was introduced by automatic compilation with Cargo, was how it structured the packages it compiled.
When Cargo builds a project and its dependencies it structures all the generated metadata and the compiled libraries within a \dir{target} directory, and an extra filename gets appended to all of these libraries.
This extra filename is part of a hash that is generated based on the code in the library, and ensures that each and every build is consistent and it resolves any problems that might arise if several dependencies within a project depend on different versions of the same library.
This works when Cargo handles the entire build process, but it our case, where we had to manually compile the final executable, it turned out to be a problem because the name of the library would change every time some of its content changed.
We worked around this problem by modifying the build script to store the hash generated by Cargo for {\emlib} to a text file, every time the library was built, and then included it in the makefile for the project.

\subsection{Conditional linking with Cargo}

The build process described in the previous section made it simpler to use third party libraries, but it did not solve all of our issues, the main problem that persisted was to have a good way of making {\emlib} itself, portable.
With the setup that we had, it was easy to create new executables \emph{within} the project, but it was hard to create new executables that \emph{depended} on {\emlib}.
Basically, because we had to work around Cargo in the final part of the build process, it also meant that \emph{every} project that wanted to depend on {\emlib} also had implement the same workarounds.
Thus, we needed to solve the problem of knowing where Cargo would store the project metadata, and we needed a way to get Cargo to compile the final executables with the extra linker-arguments needed by {\rustc} in order to compile the binary for the Cortex-M3.

Cargo does not have much documentation over how its internal works, or how to interfere with the build process, but the documentation does mention that Cargo can be extended with additional \emph{plugins}.
If Cargo is to be invoked with a command that it does not have by default, it will query the system for this command.
This means that if Cargo is invoked with e.g. the command \cmd{cargo foo <args>...}, it will query the system for an executable with the name \file{cargo-foo} and it will invoke this command with the trailing arguments if it exists.
By looking at Cargo's source code, we could see that every triggered build included a structure called \code{CompileOptions}.
The arguments passed to Cargo's different build commands are then used to compose this structure and trigger an internal compilation process, this process handles the compilation of all dependencies and generates all the different targets for the current package to be compiled.

\begin{table}[ht]
\begin{center}
\begin{tabular}{r|p{8cm}}
\textbf{Flags} & \textbf{Purpose} \\
\hline
\texttt{[$<$args$>$]} &
The trailing argument to the command was the linker-arguments that were to be passed further on to the invocation of {\rustc}.
If any \emph{args} are present, Cargo will append \flag{-C link-args="$<$args$>$"} when any executables from the package is being built. \\

\flag{--examples NAME} &
The library had many executables located in the projects \dir{examples} directory.
This flag made it easier to compile one of these examples by specifying its name. \\

\flag{--build-examples} &
This flag filtered out every executable marked as an example and compiled all of them. \\

\flag{--print-link-args} &
This flag was included for debugging purposes. \\

\hline
\end{tabular}
\caption{}
\label{tab:cargo_linkargs}
\end{center}
\end{table}


In order to solve the problems we had with building the project, we created a new subcommand called \cmd{cargo-linkargs} \cite{github:cargo_linkargs} that depends on Cargo itself.
This subcommand was created specifically with {\emlib} in mind, and supports all the flags that the \cmd{cargo-build} command supports, including the flags shown in \autoref{tab:cargo_linkargs}.
We got rid of the two problems we had with building {\emlib} once \cmd{cargo-linkargs} was working.
The problem with resolving the location of generated metadata was solved implicitly just by utilizing Cargo, and the extra linker-arguments could easily be passed on to the invocation of \cmd{cargo-linkargs} via the project's makefile.

\subsection{Continuous Integration}
\label{ssub:continuous_integration}

When we first started this project, {\rust} had reached a 1.0-alpha version.
This meant that the programming language had reached a relatively stable state, but there was still big parts of the language and its standard libraries that was marked as unstable and up for review before the planned 1.0 release.
The standard libraries, and third party {\rust} libraries that have evolved in the {\rust} community, have made little guarantee about their stability, and the \glspl{api} have been subject to change without much notice.

Continuous Integration refers to the practice of testing the whole system \emph{continuously}, for every smaller change introduced to the code base, usually with some kind of automated test framework.
Continuous Integration is advantageous to normal regression testing because it can reduce the amount of code rework that is needed in later phases of development, as well as speed up overall development time  \cite{Orso2014}.
Many {\rust} projects have utilized a continuous integration system called Travis CI \cite{web:travis_ci} for ensuring that the code in the project have been compatible with the nightly builds of {\rust}.
By registering our projects with Travis CI, and a community-developed service called {\rust} CI \cite{web:rust_ci}, we had automatic, daily builds of our projects on a third-party server.
Builds were triggered every time we released a change to the code on GitHub, and every time a new nightly release of {\rust} was published, and if a build failed we would get notified of the error.
By making continuous integration part of the normal build routine and review-process for new project code, we had an extra step of verification that the project would build on other systems then the one it was developed on.

It is important to note that continuous integration only helped us to verify that the project could be \emph{built}, it could not help us to verify that the compiled code would actually \emph{work} for its target architecture.
In order to verify that the code would work for on the Cortex-M3 we had to actually run in on one of the microcontrollers that we had a available for this project.
An experimental process of testing and mocking the {\emlib} \gls{api} bindings is described in greater detail in \autoref{ssub:testing}.

\subsection{Contributing to Cargo}
\label{ssub:contributing_to_cargo}

As already mentioned, the ability to pass arbitrary flags further on to the invocation of {\rustc} was by design not supported by Cargo, but many people in the {\rust} community have wanted the ability to do so.
The reasoning for not allowing arbitrary flags to be passed down is described here.

A compilation can go awry very quickly if it is up to the package \emph{author} what flags should be passed to {\rustc}, instead it should be up to the \emph{user} of the package.
This would have given the author the ability to set the restriction for a library, and limit the possibilities of what a user could have done with it.
Different systems do not necessarily support all flags and possibilities, so if a package dependency says that it is to be built in a specific way it might not work for the system it is being building for.

On {\cargo} project issue tracker, several related issues concerning passing arbitrary flags further to {\rustc}, was open.
All these were formalized in one issue\footnote{https://github.com/rust-lang/cargo/issues/595} for implementing a new subcommand (called \cmd{cargo-rustc}) for the package manager.
%There has been an \emph{issue} on Cargo's project page about implementing a new subcommand (called \cmd{cargo-rustc}) for the package manager.
This subcommand would have allowed for passing these flags on to {\rustc}, but with the restriction of only compiling a \emph{single} target at a time.
This means that only \emph{either} the library, a binary, an example or a test (or a package dependency), may be compiled with the extra flags, and \emph{not} the entire package.

These rules are restrictive enough to get libraries to not depend on a set of extra flags, but loose enough so that specialized projects, like our {\emlib}, can depend on it for completing the build.
Indeed, the functionality proposed with this subcommand would be enough to cover all the cases that we solved with our implementation of \cmd{cargo-linkargs}.

After gaining insight into Cargo's internals during the development of \cmd{cargo\-linkargs}, it was interesting to see if we could get this same functionality into Cargo itself by implementing \cmd{cargo-rustc}.
Even though \cmd{cargo-linkargs} worked great for its purpose, it was not very ergonomic for {\emlib} to depend on a third party plugin in order to work, especially if this functionality could be natively supported by Cargo.
Not only would it benefit our project, it would also give many other {\rust} projects the ability to use Cargo for the entire compilation process, which now resort to manually maintained Makefile \cite{}.
Since both {\rust} and Cargo are open source projects, it was quick to get in contact with the project maintainers about the issue, and eventually submit a patch with the new subcommand.
After it was reviewed by one of the project maintainers, the patch was accepted and merged into Cargo's code base, and the subcommand developed as part of our build system is now a part of {\rust}'s nightly builds.

\subsection{Final Library Build Artifacts}

The resulting files of compiling the libraries are presented in \autoref{fig:lib:structure}.

\begin{figure}[H]
  \begin{center}
    \includegraphics[scale=0.5]{figures/lib-structure.png}
  \end{center}
  \caption{The organization files of libraries}
  \label{fig:lib:structure}
\end{figure}

The figure shows that all of the libraries except for \lib{modules} consists of both a {\C} static archive (\format{*.a}) and a {\rust} library (\format{*.rlib}).
The \lib{modules} library is a high level library that is built on top of {\emlib} and \lib{emdrv}.
The rest of the libraries provides the {\rust} bindings in the \format{*.rlib} part and the C implementations in the \format{*.a} portion.
We also see the dependencies, denoted by the arrows, between the libraries generally flowing from the top level abstraction down to the lower level abstractions.


\corechapter{Application Layer}{app}{%
  In the Application Layer module of the {\rg} platform we find the application level libraries and actual executable programs.
  As part of this chapter we have included some experimental libraries developed to explore the language facilities of the {\rust} programming language in an embedded system.
  We also look at two projects developed for providing a qualitative study of the platform.

  All of the programs and libraries presented in this section builds upon the foundation provided by the lower levels of the {\rg} platform.
  None of these components are mandatory when using the platform, and when writing an application using this platform that code itself is considered part of this layer.
  The components in the following section represent example applications and possible future extentions to the base {\rg} platform as given by the previous sections.
}
% !TEX root = ../main.tex

\section{Handling interrupts with Closures}
\label{sec:irq-closures}

This section describes an experimental approach to handling interrupts with closures.
The motivation for using this pattern is to make the code more {\rust} idiomatic and to make use of the ownership rules applied to closures.
The {\rust} programming language frowns upon using global variables, especially when the variables are mutable.
Using such variables forces the programmer to use unsafe blocks, thus transferring the responsibility of the safety analysis from the compiler to the programmer.
Therefore, avoiding mutable global state is a goal of any {\rust} program.

\subsection{Motivation}
\label{sec:irq:motivation}

Let us consider a simple example for motivating the use of closures to handle interrupts.
The example application samples an analog signal and saves the result to a memory buffer.
An example of such an application is an audio filter, which samples an audio input connected to the \gls{adc} and stores a window of samples in \gls{ram} for further processing.

\begin{listing}[H]
  \begin{rustcode}
// Declaring a Circular Buffer Type globally
const N: usize = 1024;
static mut IDX = 0;
static mut BUFFER: [u32; N] = [0; N];

fn main() {
  let adc = adc0();
  // Using BUFFER requires unsafe
  // e.g.: unsafe { &BUFFER[..] }
  loop { /* ... */ }
}
pub extern fn ADC0_IRQHandler() {
  let adc = adc0();
  let sample = adc.get_single_data();
  // Writing to the buffer is considered unsafe
  unsafe { BUFFER[IDX % N] = sample; IDX += 1; }
}
  \end{rustcode}
  \caption{Analog sampler with global buffer}
  \label{lst:irq:global}
\end{listing}

\autoref{lst:irq:global} shows the proposed example with a conventional interrupt handler.
The interrupt handler is in the global scope, so it can only access global variables and therefore the buffer must be declared as \code{static mut}.
This require all read and writes to the buffer to be handled within \code{unsafe} blocks.
A huge restriction on variables defined in the global scope in {\rust}, is that they can only be of types which has constant-expression constructors.
This is a fact which we praised in \autoref{sec:impl:booting}, as it provides a very simple startup procedure, but it limits the datatypes which can be shared between interrupt handlers and the rest of the code.

\begin{listing}[H]
  \begin{rustcode}
fn main() {
  let mut adc = adc0();
  let buffer = CircularBuffer::new();
  let mut ch = buffer.in();
  // Register a closure on the ADC. The closure will be called each
  // time a new sample is ready with the sample as an argument. The
  // `move' keyword is used to move ownership of the `ch' variable.
  adc.on_single(move |sample| ch.send(sample));

  // Reading from buffer is safe
  // e.g.: &buffer[..];
  loop { /* ... */ }
}
  \end{rustcode}
  \caption{Analog sampler with local buffer}
  \label{lst:irq:local}
\end{listing}

In \autoref{lst:irq:local} we present an example implementation using a closure as an interrupt handler.
The global state is now replaced with a buffer that is owned by the \code{main} function stack frame.
In this discussion, we consider the main stack frame, the stack frame for the \code{main} function, to be a special frame.
This comes from the fact that the \code{main} function contains an infinite loop causing it to never terminate, and the frame will not get deallocated.
On an embedded system, this is true as long as a fatal error does not occur.
So we can rely on variables owned by the main stack frame to live for the duration of the application.
This ensures that the \code{buffer}, for practical purposes, has the same lifetime as the \code{BUFFER} from \autoref{lst:irq:global}, but because the variable is not a \code{static mut}, it lets the programmer avoid {\unsafe} blocks and let the compiler ensure safety.

One implementation detail here is the imagined \code{CircularBuffer} type.
The type is based on the same principle as the {\rust} standard library \code{channel} type, which provides a facility for interprocess communication.
The channel has one \textit{read end} and one \textit{write end} enabling the producing process to send a stream of messages to the consuming process.
This is required for the \gls{adc} callback to be able to write to the buffer while the main function retains ownership of the buffer.
In the example the \code{in} function creates the \textit{write end} of the circular buffer.

The core of this example is the line \code{adc.on\_single(move |sample| ch.send(sample));}.
This creates a closure that takes ownership over the \textit{write end} for the circular buffer.
The closure is passed as an argument to the \code{on\_single} method on the \gls{adc} ensuring that the closure is called each time a new sample is ready with the sample as an argument.

\subsection{Implementation}

This section looks at how we can achieve the same behavior as was presented in the previous section.

The process of handling an interrupt was described in \autoref{sec:impl:handling-interrupts}.
In short, a public function with a specific name handles the corresponding interrupt.
To implement the pattern above, we need to get the globally defined interrupt handler \code{ADC0\_IRQHandler} to call a closure created in the \code{main} function.

As already discussed, the interrupt handlers can only access global variables, which requires us to store the closure value in a static variable.
We can not do this directly because closures do not have static initializer functions, thus, we need to use raw pointers.
\autoref{lst:raw-global-pointer-to-closure} shows what a raw pointer to a closure looks like in {\rust}, note that we put the pointer inside an \code{Option} to avoid using a null pointer.

\begin{listing}[H]
  \begin{rustcode}
static mut CLOSURE: Option<*const Fn()> = None;
  \end{rustcode}
  \caption{Storing a raw pointer to the closure globally}
  \label{lst:raw-global-pointer-to-closure}
\end{listing}

This \code{CLOSURE} variable is unsafe to use because it is a static mutable variable.
\autoref{lst:reg-disp} shows a safe abstraction to register and dispatch events with this handle.
Notice how the responsibility of handling the ownership of the closure is transferred from the compiler to the programmer in the {\unsafe} block in the \code{register} function.
This is done with the \code{from\_raw} and the \code{into\_raw} functions, which converts between \emph{managed} and \emph{raw} pointers.
Respectively, these functions tell the compiler to start and stop the borrow-checker for the pointer that is returned from the two functions.

\begin{listing}[H]
  \begin{rustcode}
// Registers an interrupt handler
fn register(f: Box<Fn()>) {
  unsafe {
    // Deallocate old handler if existing
    if let Some(old) = CLOSURE {
      // Remove the global reference
      CLOSURE = None;
      // Return the ownership of the pointer to a managed Box.
      // This transfers the responsibility of deallocating the
      // closure back to the compiler.
      let _ = boxed::Box::from_raw(old);
      // Omitting the above line will not trigger compilation
      // error, but the old closure value will be leaked.
    }
    // Consume the Box pointer and return a raw pointer to the
    // closure. This transfers the responsibility of deallocating
    // the closure from the compiler to the programmer
    let raw = boxed::into_raw(f);
    // Save the closure pointer in the global reference
    CLOSURE = Some(raw);
  }
}
// Dispatch an event by calling the closure if it is registered
fn dispatch() {
  // The closure is stored in a global mutable variable,
  // so the access to this variable is unsafe
  unsafe {
    // Unwrap the closure value
    if let Some(func) = CLOSURE {
        // The closure is stored behind a pointer which must
        // be dereferenced, it is called with its environment
        // by invoking the `call' function
        (*func).call(())
    }
  }
}
  \end{rustcode}
  \caption{Safe abstraction over global raw pointer}
  \label{lst:reg-disp}
\end{listing}

The above listing use {\unsafe} code in order to provide a safe abstraction to interact with the globally stored closure.
The functionality is described throughout the comments.
In the listing, we see all of the three operations which were defined as {\unsafe} in \autoref{ssub:unsafe_code}.
These are mutating a static mutable variable (the \code{CLOSURE}), calling unsafe functions (\code{into\_raw} and \code{from\_raw}), and dereferencing a raw pointer (the \code{func} variable in the \code{dispatch} function).
The {\unsafe} functions deal with transferring the ownership of the heap allocated pointer from the compiler, to the callee, and back to the compiler.
If these functions are used improperly, it can lead to different memory related problems like double-free, use-after-free, and memory leaks.
Dereferencing the \code{func} pointer is considered unsafe because it might point to invalid memory, the \code{register} function makes sure that this pointer is valid.

% The unsafety here comes from the fact that, used improperly these functions can lead to memory problems like double-free, or use-after-free.

%As discussed in the beginning of this section the {\gecko} provides a vector of interrupts.
%The design for the closure described this far is only capable of storing a single closure.
%We now alter the design to handle an arbitrary number of handlers.

%First we define two datatypes which will aid our implementation.
%The Event enumeration given in \autoref{lst:enum:Event} which mirrors the interrupt handlers defined in the interrupt vector and an event hub structure shown in \autoref{lst:struct:Hub}

%\begin{listing}[H]
%  \begin{rustcode}
%enum Event {
%    ADC0,
%    TIMER0
%    // mirrors interrupt vector
%}
%  \end{rustcode}
%  \caption{Enumeration for Irq Handler Tag}
%  \label{lst:enum:Event}
%\end{listing}
%
%\begin{listing}[H]
%  \begin{rustcode}
%// Global reference to the Hub
%static mut HUB: Option<*mut Hub> = None;
%
%// Datastructure with a Binary Map from Event to handler
%struct Hub {
%  handlers: BTreeMap<Event, Box<Fn()>>
%}
%// Signatures for implementation of the Event Hub
%impl Hub {
%  // Create an empty hub
%  fn new() -> Hub { /* .. */ }
%  // Register a handler for an Event
%  fn register(&mut self, event: Event, f: Box<Fn()>) { /*...*/ }
%  // Trigger an event
%  fn dispatch(&self, event: Event) { /*...*/ }
%}
%// We do not want users to interact with the hub directly so we provide
%// a public interface for registering and dispatching events
%pub fn register(event: Event, f: Box<Fn()>) { /*...*/ }
%pub fn dispatch(event: Event) { /*...*/ }
%  \end{rustcode}
%  \caption{Event Hub structure}
%  \label{lst:struct:Hub}
%\end{listing}

%The \code{register} and \code{dispatch} functions implemented on the \code{Hub} \code{struct} in \autoref{lst:struct:Hub} are expanded from those in \autoref{lst:reg-disp} to look access the closure based on a the given event.
%The public interface \code{register} and \code{dispatch} looks up the global \code{HUB} and calls the corresponding function on the \code{Hub} object.

\autoref{lst:adc-abstraction} implements the interface that was presented in \autoref{lst:irq:local} for the \gls{adc}, by utilizing the \code{register} and \code{dispatch} functions from \autoref{lst:reg-disp}.

% And finally the interface used in \autoref{lst:irq:local} is created using the utilities \texttt{register} and \texttt{dispatch}.

\begin{listing}[H]
  \begin{rustcode}
impl Adc {
  // This constructor is defined in `emlib'
  pub fn adc0() -> &'static Adc { /*...*/ }
  pub fn on_single(&'static self, callback: Box<Fn(u32)>) {
    // Make sure the callback for a single conversion is enabled
    self.int_enable(IEN_SINGLE);
    // Call the utility to register an interrupt handler.
    // The `move' keyword is used to move ownership of
    // the callback function into the closure
    register(Box::new(move || {
      // Clear the interrupt signal to mark it as handled
      self.int_clear(IF_SINGLE);
      // Call the interrupt handler with the ADC sample
      callback(self.data_single_get())
    }));
  }
  // Clears interrupt signals
  pub fn int_clear(&self, flag: u32) { /*...*/ }
  // Triggers an ADC conversion and returns the result
  pub fn data_single_get(&self) -> u32 { /*...*/ }
}
// The ADC0 Interrupt handler
// This is defined as a part of the library
#[allow(non_snake_case)]
fn ADC0_IRQHandler() {
  dispatch();
}
  \end{rustcode}
  \caption{\gls{adc} abstraction over an Event Hub}
  \label{lst:adc-abstraction}
\end{listing}

%Now looking back at out target implementation in \autoref{lst:irq:local}, we augment the example to be functional with the library presented in this section.
%\autoref{lst:irq:closure} shows a runnable version of the example application from \autoref{sec:irq:motivation}.

%\begin{listing}[H]
%  \begin{rustcode}
%fn main() {
%  let mut adc = adc0();
%  let buffer = CircularBuffer::new();
%  let mut ch = buffer.in();

%  adc.on_single(Box::new(move |sample| ch.send(sample) });
%  // Using buffer is safe
%  // e.g.: &buffer[..];
%  loop {}
%}
%  \end{rustcode}
%  \caption{Running example of example application}
%  \label{lst:irq:closure}
%\end{listing}

As discussed in \autoref{sec:impl:handling-interrupts}, the {\gecko} provides a vector of interrupt handlers.
To minimize the boilerplate code for this method, the registering and dispatch mechanisms are implemented as part of an \emph{event hub} in the actual library.
This hub contains a map from event, corresponding to the interrupt names, to the interrupt handler closure.
We have left out the implementation of the event hub in this section, in order to focus more on explaining how we have handled the ownership of the closures.

\subsection{Discussion}

The method for handling interrupts with closures discussed in this section provides a nice facility to use idioms that are common in {\rust} code.
They proved a means to avoid explicit use of {\unsafe} \code{static mut} variables and, as we showed in \autoref{lst:adc-abstraction}, let us build higher level abstractions.
We revisit this section as part of a larger discussion concerning \emph{ownership to hardware} in \autoref{sec:avoiding_mutable_aliases_to_hardware}.

%The changes in \autoref{lst:irq:closure} compared to the \autoref{lst:irq:local} is the callsite of registering the cloures where it is passed inside of a owning Box pointer. \todo{We should be able to pass it by value as std::thread::spawn does}
%The other more significant change is the creation and registration of the \texttt{hub} variable, it is also stored in the global variable defined in \autoref{lst:struct:Hub}.
%This code pattern is highly risky and must be used with great care.
%The method allocates an object on the stack and saves a mutable reference to it in an global variable.
%If this technique is utilized on an arbitrary stack frame and the reference is read from, the program will most probably crash.
%This is because the stack frame in question will probably be deallocated and either unused or used for another stack frame.
%As mentioned in the start of this section we consider the stack frame of the \texttt{main} function as a special frame which will not be subject for deallocating, unless the application is about to terminate.
%This ensures that the \texttt{hub} variable has a lifetime that in practice is \texttt{static} and can be used store the boxed closures while still being reachable from the global reference \texttt{HUB}.
%The inspiration for this \textit{hack} was drawn from the pattern being used in the Servo \cite{web:servo} project for setting up an Event Loop.

%Elimintaing the \texttt{main} stack frame dependence was considerable easy to do, by using the same conversion used in the register call in \autoref{lst:reg-disp}.
%\autoref{lst:irq:init} shows a initializing function which creates a \texttt{Hub} object and registrers it in the \texttt{HUB} reference.

%\begin{listing}[H]
%  \begin{rustcode}
%static mut HUB: Option<*mut Hub> = None;
%pub fn init() {
%  if unsafe { HUB.is_none() } {
%    let hub = Box::new(Hub::new());
%    let raw = unsafe { boxed::into_raw(hub); }
%    unsafe { HUB = Some(raw); }
%  } else {
%    // already registered so ignore the call
%  }
%}
%  \end{rustcode}
%  \caption{Function for initializing the Event Hub}
%  \label{lst:irq:init}
%\end{listing}

%This lets us have an Event Hub with truly static lifetime without allocating it in the \texttt{main} function stack frame.

%\subsection{Discussion}

%The issue with the end solution presented in \autoref{lst:irq:init} which makes it unsuited for Servo is the fact that when converting the \texttt{Box} to a raw pointer, the programmer takes on the responsibility of deallocating the object.
%In Servos use case, the desired behaviour is to deallocate the object when the \texttt{main} stack frame is deallocated.
%Letting the reference live on the \texttt{main} stack frame ensures this behaviour is provided automatically by the {\rust} compiler.

%The approach taken in \autoref{lst:irq:init} is similar to the one taken by the \textit{lazy\_statics} \cite{web:lazy_statics} library.
%\textit{lazy\_statics} provides a macro for declaring static variables which need initializers executed at runtime.
%This macro avoids the need for the initilizer method used here as it implements the \texttt{Deref} trait for the hidden `static mut` variable, the \texttt{HUB} variable in our example.
%The library uses a {\rust} RSL feature to register an initializer function which is guaranteed to run once only on the first dereference of the variable.

\section{Porting GPIO Interrupt Driver}

This section considers a case study of an issue discovered while porting the \gls{gpio} interrupt driver from \lib{emdrv} to {\rust}.
The issue was discoverd when annotating the \texttt{unsafe} blocks for referencing mutable global state and gives an example of the awareness the inclusion of the {\unsafe} keyword provides.

\subsection{Presenting the Problem}

The \texttt{gpioint} driver lets the user registers a callback function to be called when an interrupt occures at a given \gls{gpio} pin.
It is implemented with a global mutable list of 16 function pointers, a register function to assign functions to indices of the list corresponding to \gls{gpio} pins, and a dispatch mechanisms which calls the correct functions when an interrupt occurs.
The issue arises in the dispatch function in \autoref{lst:irq-dispatch:c}.

\begin{listing}
  \begin{minted}{c}
static void GPIOINT_IRQDispatcher(uint32_t iflags) {
  while(iflags) {
    // Utility for iterating through all active interrupt signals
    uint32_t irqIdx = GPIOINT_MASK2IDX(iflags);
    // Mark interrupt as handled
    iflags &= ~(1 << irqIdx);
    // Check if the interrupt has a callback
    if (gpioCallbacks[irqIdx]) {
      // Call the callback
      gpioCallbacks[irqIdx](irqIdx);
    }
  }
}
  \end{minted}
  \caption{GPIO Dispatcher from emlib}
  \label{lst:irq-dispatch:c}
\end{listing}

\begin{listing}
  \begin{minted}{rust}
static mut GPIO_CALLBACKS: [Option<fn(u8)>; 16] = [None; 16];

fn dispatcher(iflags: u32) {
  while(iflags) {
    // Utility for iterating through all active interrupt signals
    let irq_idx = mask_to_index(iflags);
    // Mark interrupt as handled
    iflags &= !(1 << irq_idx);
    // Check if the interrupt has a callback
    if (unsafe { GPIO_CALLBACKS[irq_idx] }.is_some()) {

      // Window of opportunity

      // Unwrap the callback and call the function
      unsafe { GPIO_CALLBACKS[irq_idx] }.unwrap()(irq_idx);
    }
  }
}
  \end{minted}
  \caption{GPIO Dispatcher naively ported to Rust}
  \label{lst:irq-dispatch:rust}
\end{listing}

A first take at porting the dispatcher function to Rust yields the code in \autoref{lst:irq-dispatch:rust}.
It is quite easy to see that the global mutable state is read twice.
This means that there is a possibility of the second reference to return a different value than the first.
For instance the function \func{GPIOINT\_CallbackUnRegister(uin32\_t pin)} will insert the value \mem{0x0} into the specified pin.
If this function is called inside a interrupt handler and this interrupt is triggered while the GPIO driver is dispatching an interrupt the function pointer can be set to \mem{0x0} between the check and the call.
Calling a function pointer that points to \mem{0x0} will cause a \func{HardFault}.

\subsection{Analysis of Assembly}
To dive a bit further into the issue and to prove that it is only present at optimization level O0 we consider the assembly code for the dispatcher.

The subsection of the \func{GPIOINT\_IRQDispatcher} in assembly generated by \prog{arm-none-eabi-gcc -O0 -S} is reproduced in \autoref{lst:gpio:asm:O0}.

\begin{listing}[H]
  \begin{minted}{ca65}
GPIOINT_IRQDispatcher:
  ;; ...                    ;;
  ldr r3, .L34              ;; r3 = gpioCallbacks
  ldr r2, [fp, #-8]         ;; r2 = irqIdx
  ldr r3, [r3, r2, asl #2]  ;; r3 = gpioCallbacks[irqIdx]
; start interrupt window    ;;
  cmp r3, #0                ;; if (r3 == 0) {
  beq .L30                  ;;
  ldr r3, .L34              ;;   r3 = gpioCallbacks
  ldr r2, [fp, #-8]         ;;   r2 = irqIdx
; end interrupt window      ;;
  ldr r3, [r3, r2, asl #2]  ;;   r3 = gpioCallbacks[irqIdx]
  ;; ...                    ;;
  bx  r3                    ;;   (*r3)();  call the function
  ;; ...                    ;; }
  \end{minted}
  \caption{GPIOINT Dispatcher in assembly with O0}
  \label{lst:gpio:asm:O0}
\end{listing}

Here we see a window of 4 instructions where the proposed harmfull interrupt can occure.
It is the \func{ldr} instruction just before the window opening and the one just after that causes the issue.
These two loads must load the same address for the logic to be valid, although the second one is just required to not load \mem{0x0} in order not to cause a \func{HardFault}.

Looking at the same assembler generated by compiling the code with \prog{arm-none-eabi-gcc -O1 -S} in \autoref{lst:gpio:asm:O1} we see that the issue has been eliminated.

\begin{listing}[H]
  \begin{minted}{ca65}
GPIOINT_IRQDispatcher:
  ;; ...                    ;;
  ldr r5, .L5               ;; r5 = gpioCallbacks
  ;; ...                    ;;
  and r3, r0, #255          ;; r3 = irqIdx
  ;; ...                    ;;
  ldr r3, [r5, r3, asl #2]  ;; r3 = gpioCallbacks[irqIdx]
  cmp r3, #0                ;; if (r3 == 0) {
  ;; ...                    ;;
  bxne    r3                ;;   (*r3)()
  ;; ...                    ;; }
  \end{minted}
  \caption{GPIOINT Dispatcher in assembly with O1}
  \label{lst:gpio:asm:O1}
\end{listing}

At O1 the compiler has performed \gls{cse} to remove the duplicate load present in \autoref{lst:gpio:asm:O0}.
This can be elimintated with the assumption that the gpioCallback will not be changed by external code.
But as the Rust version in \autoref{lst:irq-dispatch:rust} suggests this code can lead to a \concept{data race} because it is referencing a global mutable variable.

\subsection{Proposed solution}

The solution to this problem is quite straightforward by performening the \gls{cse} manually.
\autoref{lst:irq-dispatch:c-fixed} contains the implementation proposed to SiliconLabs to resolve this issue.

\begin{listing}[H]
  \begin{minted}{rust}
static void GPIOINT_IRQDispatcher(uint32_t iflags) {
  while(iflags) {
    uint32_t irqIdx = GPIOINT_MASK2IDX(iflags);
    iflags &= ~(1 << irqIdx);
    GPIOINT_IrqCallbackPtr_t callback = gpioCallbacks[irqIdx];
    if (callback) {
      callback(irqIdx);
    }
  }
}
  \end{minted}
  \caption{GPIOINT Dispatcher without Data Race}
  \label{lst:irq-dispatch:c-fixed}
\end{listing}

We again consult the assembly code in \autoref{lst:irq-dispatch:asm-fixed} generated to verify that this resolved the issue.

\begin{listing}[H]
  \begin{minted}{ca65}
GPIOINT_IRQDispatcher:
  ;; ...                     ;;
  ldr   r3, .L34             ;; r3 = gpioCallbacks
  ldr   r2, [fp, #-8]        ;; r2 = irqIdx
  ldr   r3, [r3, r2, asl #2] ;; r3 = gpioCallbacks[irqIdx]
  ;; ...                     ;;
  cmp   r3, #0               ;; if (r3 == 0) {
  ;; ...                     ;;
  bx    r3                   ;;   (*r3)()
  ;; ...                     ;; }
  \end{minted}
  \caption{GPIOINT Dispatcher for proposed solution at O0}
  \label{lst:irq-dispatch:asm-fixed}
\end{listing}

\subsection{Discussion}

The issue presented in this section is a minor one and will probably never cause a \func{HardFault} in a real world application.
Nevertheless is serves as an example of how the {\unsafe} keyword in Rust makes the programmer think twice about code in these unsafe sections.
It aslo points to the gains of using Rust to prototype subsets of an C library to see if issues with a more strict compiler will arise from the patterns and constructs used.

% !TEX root = ../main.tex

\section{Rust Embedded Modules}
\label{sec:rust-embedded-modules}

This section describes a separate project, which we refer to as \gls{rem}.
\gls{rem} have been developed alongside the implementation of the various binding libraries, and contains a handful of higher-level modules for {\emlib}'s different peripheral bindings.
The peripheral abstractions that have been implemented as part of \gls{rem} are still in very early development

\subsection{Peripheral Abstractions}
\label{sub:peripheral_abstractions}

 % and act as playground to experiment with how {\rust} can be used to define high-level

This section describes the different peripheral abstractions that are part of \gls{rem}.
Some of these modules are more complete than others.

\subsubsection{Usart}
\label{ssub:usart}

The \gls{usart} have many different use-cases, it is a peripheral that is used for transferring of data, but it is also a very convenient tool to use for simple debugging of programs.
It can be used to send single strings of text between a PC and the \gls{mcu}, which is convenient for ``println''-debugging, and it is a good tool for defining \gls{cli} programs.

The {\gecko} has a total of three different \glspl{usart} which can be configured to run on a total of eleven different locations i.e. \gls{gpio} Ports and Pins.
If the \gls{gpio} configuration for a \gls{usart}  is not specified correctly, the peripheral will not function correctly either.
The goal of the \gls{usart} abstraction was to make it easy to initialize the peripheral, as well as providing simple methods to read and write strings, and transfer data between two end-points.

The \texttt{Usart} module has an initialization procedure that takes care of initializing its required \gls{gpio} pins based on a specified location.
This is similar to the approach made by {\zinc}, but instead failing with a compilation-error if it is incorrectly configured, as in {\zinc}, it will fail at run-time.
A minimal example that shows how to initialize an \gls{usart}, and send and receive strings, is shown in \autoref{lst:usart_abstraction}.
Note that the example is simplified slightly, we have trimmed away some \code{extern} and \code{use} statements in order to make the important parts more clear.
The point of this example is to demonstrate a program that initializes and uses an \gls{usart} with only four lines of code.

\begin{listing}[H]
  \begin{minted}{rust}
extern crate emlib;         // Include emlib bindings
extern crate modules;       // Include `REM'
use modules::Usart;         // The USART module

fn main() {
  // Acquire a USART with default configuration...
  let mut usart: Usart = Default::default();
  // ... and initialize its GPIO.
  usart.init_async();

  loop {
    // Perform a blocking read operation...
    let name = usart.read_line();
    // ... and echo back with a nice message.
    usart.write_line(&format!("Thank you, {}!", name));
  }
}
  \end{minted}
  \caption{Example usage of the Usart module.}
  \label{lst:usart_abstraction}
\end{listing}

% !TEX root = ../main.tex

\section{Projects}
\label{sec:impl:projects}

This section describes two applications that were developed as part of the evaluation process of the {\rg} platform.
The two applications have been implemented in both {\rust} and {\C}, and we have gathered results from energy measurements, code size, and execution performance for all of the applications.
We use these findings to try and give an evaluation of {\rust} for an embedded system.
The results are presented in \autoref{chap:results} and discussed in \autoref{chap:discussion}, respectively.

\include*{implementation/project-i}
\include*{implementation/project-ii}

