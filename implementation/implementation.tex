% !TEX root = ../main.tex

\newcommand{\corechapter}[3]{
  \chapter{#1}
  \label{chap:#2}
  \begin{center}
    \includegraphics[scale=0.8]{figures/RustyGecko-#2}
  \end{center}
  \hfill \break
  \hfill \break
#3
}

\corechapter{Startup for Rust}{startup}{%
The \lib{startup} module handles the process of starting a {\rust} application on the \glsdesc{mcu}.
This module is a foundation for the {\rg} platform, but it can be used in isolation to run a program on the \gls{mcu} without any dependencies.

This chapter describes two important parts of a {\rust} program that needs to be configured correctly for the language to be functional with the {\gecko}.
First, we describe how we configured a {\rust} program to become an executable for the {\gecko}.
At the end of this chapter we describe how we get hardware interrupts to trigger interrupt handlers implemented in {\rust}.
}
\include*{implementation/booting-rust}

\corechapter{Rust Embedded Library}{rel}{%
The \glsdesc{rel} module defines the subset of the standard {\rust} library that is applicable for embedded applications.
This module builds on the foundation laid out by the startup module and is used by the \lib{bindings} and \lib{Application Layer} modules of the {\rg} platform.

This chapter starts with presenting the parts of \gls{rcl} that needs to be configured in order for the library to work for a new platform.
We then move over to describe the standard {\rust} libraries that provide heap allocation and dynamically allocated structures.
At the end of this chapter, we present our definition of the \gls{rel}.
}
\include*{implementation/rust-embedded-library}


\corechapter{Binding Libraries}{bindings}{%
The \lib{bindings} module includes the peripheral libraries provided by the \gls{mcu} vendor Silicon Labs, and the architecture designer ARM.
In order to make use of these libraries in {\rust}, we developed binding libraries to expose the underlying {\C} implementation to the {\rust} language.

We start this chapter with a section that maps the object-oriented paradigm to hardware peripherals, which lays a foundation for how we think about the hardware throughout this thesis.
In the last section of this chapter, we describe how we structured, implemented, and tested the library bindings.
}
\include*{implementation/objectoriented}
\include*{implementation/interfacing_with_emlib}


\corechapter{Build System}{build}{%
In this chapter, we step out of the core software components of the {\rg} platform to present an external, but highly important, part of the platform.
The {\cargo} package manager is an integral part of the {\rust} ecosystem and facilitates sharing code and libraries with ease.

Throughout this chapter, we look at how we evolved the build system over time and ultimately migrated the process over to {\cargo}.
This includes managing project dependencies and building \gls{rel} and the \lib{bindings} modules for the ARM architecture.
We have also utilized a continuous integration system that has helped us to keep the project up to date with the nightly builds of {\rust}, and to ensure that the builds have been consistent across the systems it has been built on.
}
\include*{implementation/building}

\corechapter{Application Layer}{app}{%
The \lib{Application Layer} is the top module of the {\rg} platform.
This module represents all higher-level libraries and executables built on top of the \lib{bindings} module and the \glsdesc{rel}.

Throughout this thesis, we have worked on many different project ideas that build upon the work presented in the previous chapters.
A collection of the more interesting projects are presented in this chapter.
First, we describe a library that was ported from {\C} to {\rust}, which shows how {\rust}'s concept of \emph{unsafe} can guide the programmer to write more secure code.
In \autoref{sec:irq-closures}, we describe a library that provides a {\rust}-idiomatic way to handle interrupts, which is motivated by the desire to provide a safer way of interacting with the \gls{mcu} peripherals.
In \autoref{sec:rust-embedded-modules}, we apply {\rust}'s trait system to the hardware peripherals in order to build a higher-level \gls{api} than {\emlib}.
In the last section, we describe the two applications that were developed for evaluating the {\rg} platform.
}
\include*{implementation/gpioint}
\include*{implementation/irq-closures}
\include*{implementation/rust-embedded-modules}
\include*{implementation/projects}
