% !TEX root = ../main.tex

\section{Handling interrupts with Closures}
\label{sec:irq-closures}

This section describes an experimental approach to handling interrupts with closures.
The motivation for using this pattern is to make the code more {\rust} idiomatic and to make use of the ownership rules applied to closures.
The {\rust} programming language frowns upon using global variables, especially when the variables are mutable.
Using such variables forces the programmer to use unsafe blocks, thus transferring the responsibility of the safety analysis from the compiler to the programmer.
Therefore, avoiding mutable global state is a goal of any {\rust} program.

\subsection{Motivation}
\label{sec:irq:motivation}

Let us consider a simple example for motivating the use of closures to handle interrupts.
The example application samples an analog signal and saves the result to a memory buffer.
An example of such an application is an audio filter, which samples an audio input connected to the \gls{adc} and stores a window of samples in \gls{ram} for further processing.

\begin{listing}[H]
  \begin{rustcode}
// Declaring a Circular Buffer Type globally
const N: usize = 1024;
static mut IDX = 0;
static mut BUFFER: [u32; N] = [0; N];

fn main() {
  let adc = adc0();
  // Using BUFFER requires unsafe
  // e.g.: unsafe { &BUFFER[..] }
  loop { /* ... */ }
}
pub extern fn ADC0_IRQHandler() {
  let adc = adc0();
  let sample = adc.get_single_data();
  // Writing to the buffer is considered unsafe
  unsafe { BUFFER[IDX % N] = sample; IDX += 1; }
}
  \end{rustcode}
  \caption{Analog sampler with global buffer}
  \label{lst:irq:global}
\end{listing}

\autoref{lst:irq:global} shows the proposed example with a conventional interrupt handler.
The interrupt handler is in the global scope, so it can only access global variables and therefore the buffer must be declared as \code{static mut}.
This require all read and writes to the buffer to be handled within \code{unsafe} blocks.
A huge restriction on variables defined in the global scope in {\rust}, is that they can only be of types which has constant-expression constructors.
This is a fact which we praised in \autoref{sec:impl:booting}, as it provides a very simple startup procedure, but it limits the datatypes which can be shared between interrupt handlers and the rest of the code.

\begin{listing}[H]
  \begin{rustcode}
fn main() {
  let mut adc = adc0();
  let buffer = CircularBuffer::new();
  let mut ch = buffer.in();
  // Register a closure on the ADC. The closure will be called each
  // time a new sample is ready with the sample as an argument. The
  // `move' keyword is used to move ownership of the `ch' variable.
  adc.on_single(move |sample| ch.send(sample));

  // Reading from buffer is safe
  // e.g.: &buffer[..];
  loop { /* ... */ }
}
  \end{rustcode}
  \caption{Analog sampler with local buffer}
  \label{lst:irq:local}
\end{listing}

In \autoref{lst:irq:local} we present an example implementation using a closure as an interrupt handler.
The global state is now replaced with a buffer that is owned by the \code{main} function stack frame.
In this discussion, we consider the main stack frame, the stack frame for the \code{main} function, to be a special frame.
This comes from the fact that the \code{main} function contains an infinite loop causing it to never terminate, and the frame will not get deallocated.
On an embedded system, this is true as long as a fatal error does not occur.
So we can rely on variables owned by the main stack frame to live for the duration of the application.
This ensures that the \code{buffer}, for practical purposes, has the same lifetime as the \code{BUFFER} from \autoref{lst:irq:global}, but because the variable is not a \code{static mut}, it lets the programmer avoid {\unsafe} blocks and let the compiler ensure safety.

One implementation detail here is the imagined \code{CircularBuffer} type.
The type is based on the same principle as the {\rust} standard library \code{channel} type, which provides a facility for interprocess communication.
The channel has one \textit{read end} and one \textit{write end} enabling the producing process to send a stream of messages to the consuming process.
This is required for the \gls{adc} callback to be able to write to the buffer while the main function retains ownership of the buffer.
In the example the \code{in} function creates the \textit{write end} of the circlar buffer.

The core of this example is the line \code{adc.on\_single(move |sample| ch.send(sample));}.
This creates a closure that takes ownership over the \textit{write end} for the circular buffer.
The closure is passed as an argument to the \code{on\_single} method on the \gls{adc} ensuring that the closure is called each time a new sample is ready with the sample as an argument.

\subsection{Implementation}

This section looks at how we can achieve the same behavior as was presented in the previous section.

The process of handling an interrupt was described in \autoref{sec:impl:handling-interrupts}.
In short, a public function with a specific name handles the corresponding interrupt.
To implement the pattern above, we need to get the globally defined interrupt handler \code{ADC0\_IRQHandler} to call a closure created in the \code{main} function.

As already discussed, the interrupt handlers can only access global variables, which requires us to store the closure value in a static variable.
We can not do this directly because closures do not have static initializer functions, thus, we need to use raw pointers.
\autoref{lst:raw-global-pointer-to-closure} shows what a raw pointer to a closure looks like in {\rust}, note that we put the pointer inside an \code{Option} to avoid using a null pointer.

\begin{listing}[H]
  \begin{rustcode}
static mut CLOSURE: Option<*const Fn()> = None;
  \end{rustcode}
  \caption{Storing a raw pointer to the closure globally}
  \label{lst:raw-global-pointer-to-closure}
\end{listing}

This \code{CLOSURE} variable is unsafe to use because it is a static mutable variable.
\autoref{lst:reg-disp} shows a safe abstraction to register and dispatch events with this handle.
Notice how the responsibility of handling the ownership of the closure is transferred from the compiler to the programmer in the {\unsafe} block in the \code{register} function.
This is done with the \code{from\_raw} and the \code{into\_raw} functions, which converts between \emph{managed} and \emph{raw} pointers.
Respectively, these functions tell the compiler to start and stop the borrow-checker for the pointer that is returned from the two functions.

\begin{listing}[H]
  \begin{rustcode}
// Registers an interrupt handler
fn register(f: Box<Fn()>) {
  unsafe {
    // Deallocate old handler if existing
    if let Some(old) = CLOSURE {
      // Remove the global reference
      CLOSURE = None;
      // Return the ownership of the pointer to a managed Box.
      // This transfers the responsibility of deallocating the
      // closure back to the compiler.
      let _ = boxed::Box::from_raw(old);
      // Omitting the above line will not trigger compilation
      // error, but the old closure value will be leaked.
    }
    // Consume the Box pointer and return a raw pointer to the
    // closure. This transfers the responsibility of deallocating
    // the closure from the compiler to the programmer
    let raw = boxed::into_raw(f);
    // Save the closure pointer in the global reference
    CLOSURE = Some(raw);
  }
}
// Dispatch an event by calling the closure if it is registered
fn dispatch() {
  // The closure is stored in a global mutable variable,
  // so the access to this variable is unsafe
  unsafe {
    // Unwrap the closure value
    if let Some(func) = CLOSURE {
        // The closure is stored behind a pointer which must
        // be dereferenced, it is called with its environment
        // by invoking the `call' function
        (*func).call(())
    }
  }
}
  \end{rustcode}
  \caption{Safe abstraction over global raw pointer}
  \label{lst:reg-disp}
\end{listing}

The above listing use {\unsafe} code in order to provide a safe abstraction to interact with the globally stored closure.
The functionality is described throughout the comments.
In the listing, we see all of the three operations which were defined as {\unsafe} in \autoref{ssub:unsafe_code}.
These are mutating a static mutable variable (the \code{CLOSURE}), calling unsafe functions (\code{into\_raw} and \code{from\_raw}), and dereferencing a raw pointer (the \code{func} variable in the \code{dispatch} function).
The {\unsafe} functions deal with transferring the ownership of the heap allocated pointer from the compiler, to the callee, and back to the compiler.
If these functions are used improperly, it can lead to different memory related problems like double-free, use-after-free, and memory leaks.
Dereferencing the \code{func} pointer is considered unsafe because it might point to invalid memory, the \code{register} function makes sure that this pointer is valid.

% The unsafety here comes from the fact that, used improperly these functions can lead to memory problems like double-free, or use-after-free.

%As discussed in the beginning of this section the {\gecko} provides a vector of interrupts.
%The design for the closure described this far is only capable of storing a single closure.
%We now alter the design to handle an arbitrary number of handlers.

%First we define two datatypes which will aid our implementation.
%The Event enumeration given in \autoref{lst:enum:Event} which mirrors the interrupt handlers defined in the interrupt vector and an event hub structure shown in \autoref{lst:struct:Hub}

%\begin{listing}[H]
%  \begin{rustcode}
%enum Event {
%    ADC0,
%    TIMER0
%    // mirrors interrupt vector
%}
%  \end{rustcode}
%  \caption{Enumeration for Irq Handler Tag}
%  \label{lst:enum:Event}
%\end{listing}
%
%\begin{listing}[H]
%  \begin{rustcode}
%// Global reference to the Hub
%static mut HUB: Option<*mut Hub> = None;
%
%// Datastructure with a Binary Map from Event to handler
%struct Hub {
%  handlers: BTreeMap<Event, Box<Fn()>>
%}
%// Signatures for implementation of the Event Hub
%impl Hub {
%  // Create an empty hub
%  fn new() -> Hub { /* .. */ }
%  // Register a handler for an Event
%  fn register(&mut self, event: Event, f: Box<Fn()>) { /*...*/ }
%  // Trigger an event
%  fn dispatch(&self, event: Event) { /*...*/ }
%}
%// We do not want users to interact with the hub directly so we provide
%// a public interface for registering and dispatching events
%pub fn register(event: Event, f: Box<Fn()>) { /*...*/ }
%pub fn dispatch(event: Event) { /*...*/ }
%  \end{rustcode}
%  \caption{Event Hub structure}
%  \label{lst:struct:Hub}
%\end{listing}

%The \code{register} and \code{dispatch} functions implemented on the \code{Hub} \code{struct} in \autoref{lst:struct:Hub} are expanded from those in \autoref{lst:reg-disp} to look access the closure based on a the given event.
%The public interface \code{register} and \code{dispatch} looks up the global \code{HUB} and calls the corresponding function on the \code{Hub} object.

\autoref{lst:adc-abstraction} implements the interface that was presented in \autoref{lst:irq:local} for the \gls{adc}, by utilizing the \code{register} and \code{dispatch} functions from \autoref{lst:reg-disp}.

% And finally the interface used in \autoref{lst:irq:local} is created using the utilities \texttt{register} and \texttt{dispatch}.

\begin{listing}[H]
  \begin{rustcode}
impl Adc {
  // This constructor is defined in `emlib'
  pub fn adc0() -> &'static Adc { /*...*/ }
  pub fn on_single(&'static self, callback: Box<Fn(u32)>) {
    // Make sure the callback for a single conversion is enabled
    self.int_enable(IEN_SINGLE);
    // Call the utility to register an interrupt handler.
    // The `move' keyword is used to move ownership of
    // the callback function into the closure
    register(Box::new(move || {
      // Clear the interrupt signal to mark it as handled
      self.int_clear(IF_SINGLE);
      // Call the interrupt handler with the ADC sample
      callback(self.data_single_get())
    }));
  }
  // Clears interrupt signals
  pub fn int_clear(&self, flag: u32) { /*...*/ }
  // Triggers an ADC conversion and returns the result
  pub fn data_single_get(&self) -> u32 { /*...*/ }
}
// The ADC0 Interrupt handler
// This is defined as a part of the library
#[allow(non_snake_case)]
fn ADC0_IRQHandler() {
  dispatch();
}
  \end{rustcode}
  \caption{\gls{adc} abstraction over an Event Hub}
  \label{lst:adc-abstraction}
\end{listing}

%Now looking back at out target implementation in \autoref{lst:irq:local}, we augment the example to be functional with the library presented in this section.
%\autoref{lst:irq:closure} shows a runnable version of the example application from \autoref{sec:irq:motivation}.

%\begin{listing}[H]
%  \begin{rustcode}
%fn main() {
%  let mut adc = adc0();
%  let buffer = CircularBuffer::new();
%  let mut ch = buffer.in();

%  adc.on_single(Box::new(move |sample| ch.send(sample) });
%  // Using buffer is safe
%  // e.g.: &buffer[..];
%  loop {}
%}
%  \end{rustcode}
%  \caption{Running example of example application}
%  \label{lst:irq:closure}
%\end{listing}

As discussed in \autoref{sec:impl:handling-interrupts}, the {\gecko} provides a vector of interrupt handlers.
To minimize the boilerplate code for this method, the registering and dispatch mechanisms are implemented as part of an \emph{event hub} in the actual library.
This hub contains a map from event, corresponding to the interrupt names, to the interrupt handler closure.
We have left out the implementation of the event hub in this section, in order to focus more on explaining how we have handled the ownership of the closures.

\subsection{Discussion}

The method for handling interrupts with closures discussed in this section provides a nice facility to use idioms that are common in {\rust} code.
They proved a means to avoid explicit use of {\unsafe} \code{static mut} variables and, as we showed in \autoref{lst:adc-abstraction}, let us build higher level abstractions.
We revisit this section as part of a larger discussion concerning \emph{ownership to hardware} in \autoref{sec:ownership_allied_to_hardware}.

%The changes in \autoref{lst:irq:closure} compared to the \autoref{lst:irq:local} is the callsite of registering the cloures where it is passed inside of a owning Box pointer. \todo{We should be able to pass it by value as std::thread::spawn does}
%The other more significant change is the creation and registration of the \texttt{hub} variable, it is also stored in the global variable defined in \autoref{lst:struct:Hub}.
%This code pattern is highly risky and must be used with great care.
%The method allocates an object on the stack and saves a mutable reference to it in an global variable.
%If this technique is utilized on an arbitrary stack frame and the reference is read from, the program will most probably crash.
%This is because the stack frame in question will probably be deallocated and either unused or used for another stack frame.
%As mentioned in the start of this section we consider the stack frame of the \texttt{main} function as a special frame which will not be subject for deallocating, unless the application is about to terminate.
%This ensures that the \texttt{hub} variable has a lifetime that in practice is \texttt{static} and can be used store the boxed closures while still being reachable from the global reference \texttt{HUB}.
%The inspiration for this \textit{hack} was drawn from the pattern being used in the Servo \cite{web:servo} project for setting up an Event Loop.

%Elimintaing the \texttt{main} stack frame dependence was considerable easy to do, by using the same conversion used in the register call in \autoref{lst:reg-disp}.
%\autoref{lst:irq:init} shows a initializing function which creates a \texttt{Hub} object and registrers it in the \texttt{HUB} reference.

%\begin{listing}[H]
%  \begin{rustcode}
%static mut HUB: Option<*mut Hub> = None;
%pub fn init() {
%  if unsafe { HUB.is_none() } {
%    let hub = Box::new(Hub::new());
%    let raw = unsafe { boxed::into_raw(hub); }
%    unsafe { HUB = Some(raw); }
%  } else {
%    // already registered so ignore the call
%  }
%}
%  \end{rustcode}
%  \caption{Function for initializing the Event Hub}
%  \label{lst:irq:init}
%\end{listing}

%This lets us have an Event Hub with truly static lifetime without allocating it in the \texttt{main} function stack frame.

%\subsection{Discussion}

%The issue with the end solution presented in \autoref{lst:irq:init} which makes it unsuited for Servo is the fact that when converting the \texttt{Box} to a raw pointer, the programmer takes on the responsibility of deallocating the object.
%In Servos use case, the desired behaviour is to deallocate the object when the \texttt{main} stack frame is deallocated.
%Letting the reference live on the \texttt{main} stack frame ensures this behaviour is provided automatically by the {\rust} compiler.

%The approach taken in \autoref{lst:irq:init} is similar to the one taken by the \textit{lazy\_statics} \cite{web:lazy_statics} library.
%\textit{lazy\_statics} provides a macro for declaring static variables which need initializers executed at runtime.
%This macro avoids the need for the initilizer method used here as it implements the \texttt{Deref} trait for the hidden `static mut` variable, the \texttt{HUB} variable in our example.
%The library uses a {\rust} RSL feature to register an initializer function which is guaranteed to run once only on the first dereference of the variable.
