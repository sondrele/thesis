\section{Booting Rust on the Gecko}
\label{sec:impl:booting}

The contents of this section explains how the startup process described in \autoref{sec:back:startup} is implemented for a {\rust} program.
The process for {\rust} is identical to the process in {\C} due to the fact that {\rust} only allows constant initialization\footnote{This is in contrast to languages like {\Cpp}, where objects can be initialized with statically called constructors. For such langauges, this must be handled by the startup process.} before the {\main} function.
Therefore we are able to reuse standard runtime components that are available for {\C} embedded toolchains.
Silicon Labs' software suite provides the \func{ResetHandler} and linker script for the EFM32 microcontrollers.
The \func{\_mainCRTStartup} function defined in the {\C} runtime, which is provided by the {\armgcc} toolchain, handles the setup in the \func{\_start} routine, as described in \autoref{sec:before-main}.

\subsection{Minimal Rust program to boot}
\label{ssec:minimal_rust_program_to_boot}

There are a few modification which have to be applied to the default `Hello World' program in order to get it to boot in an embedded environment.
Let us first revisit the canonical version from \autoref{sub:the_rust_programming_language} given in \autoref{lst:rust-hello-world}.
We can see that all we have to define is the \code{main} function.

\begin{listing}[H]
  \begin{rustcode}
fn main() {
  println!("Hello, World!");
}
  \end{rustcode}
  \caption{Standard `Hello World' in {\rust}.}
  \label{lst:rust-hello-world}
\end{listing}

A {\rust} program will by default include \glsdesc{rsl} automatically.
As explained in \autoref{sec:rsl}, this library is not usable in an embedded program, and the changes to the `Hello World' program mostly deals with removing the \gls{rsl}.

The embedded `Hello World' program, as given in \autoref{lst:embedded-rust-hello-world}, does not print out the ``Hello, World!'' text because this would have required additional setup, and we are only concerned with the boot process in this example.
In addition to removing the \gls{rsl}, the {\main} function is \emph{exported} to be callable from the {\C} runtime.

\begin{listing}[H]
  \begin{rustcode}
// Annotations
#![no_std]
#![no_main]
#![feature(no_std, core, lang_items)]

// Crate import
extern crate core;

// Define main function
#[no_mangle]
pub extern fn main() {
  loop {}
}

// Define three functions which are needed by core but defined in std
#[lang="stack_exhausted"] extern fn stack_executed() {}
#[lang="eh_personality"] extern fn eh_personality() {}
#[lang="panic_fmt"] fn panic_fmt() -> ! { loop {} }
\end{rustcode}
\caption{Embedded `Hello World' in {\rust}.}
\label{lst:embedded-rust-hello-world}
\end{listing}

\attrib{\#\![no\_std]} on line 2 in \autoref{lst:embedded-rust-hello-world} tells the {\rust} compiler not to include the standard library.
Line 3 must be analyzed in conjunction with lines 10 and 11.
Firstly, we must guarantee that the function can be called by the \func{\_start} function.
This is done by defining the {\main} function to be a publicly exported symbol denoted by \code{pub extern}.
The second change is to ensure that the function is callable by a {\C} function.
\code{extern} makes this possible by making the function use the {\C} \gls{abi}.
The last thing is to disable the {\rust} name mangling so that the {\C} code can refere to the function by the unmangled name {\main}.
Now that the {\main} function is altered to be callable by C, the function does not resemble the function the rust compiler expects to find.
Therefore we have to tell the compiler that the program does not contain a {\main} function, but that this is ok, hence the \attrib{\#\![no\_main]} on line 3.

The last three lines are a complication due to error handling in {\rust}.
These functions are used by the core library, but implemented in the standard library.
Since we are not using \gls{rsl} in this example we have to define the functions ourselves.
The implementation shown here just ignores all error handling.

\subsection{Storage qualifiers}

As described in \autoref{sec:back:startup} the startup procedure initializes all global variables.
In this section we look at how each storage qualifier,  applicable to global variables in {\rust}, map to different sections of the {\elf} binary.

\begin{listing}[H]
\begin{rustcode}
const      RUST_CONST_ZERO: u32 = 0;      // not allocated
const      RUST_CONST: u32 = 0xFEED;      // not allocated
static     RUST_STATIC_ZERO: u32 = 0;     // .text
static     RUST_STATIC: u32 = 0xDEAD;     // .text
static mut RUST_STATIC_MUT_ZERO: u32 = 0; // .bss
static mut RUST_STATIC_MUT: u32 = 0xBEEF; // .data

pub extern fn main() { /* Use the variables */ }
\end{rustcode}
\caption{{\rust} static initialization.}
\label{lst:rust-static-init}
\end{listing}

In \autoref{lst:rust-static-init}, the three different types of declaring globals in {\rust} are shown.
{\rust} divides between two types of global declarations, constants and statics.

A constant declaration, shown in lines 1 and 2 of \autoref{lst:rust-static-init}, represents a value.
There is no need to allocate memory for globals declared as \code{const}, as the values can be directly inserted where they are used by the compiler.

The static globals are immutable by default, but can be made mutable by the \code{mut} keyword.
The variables on line 3 and 4 are declared to be \code{static}.
As these are immutable, they are allocated in the read-only section called \elfsec{.text}.

On line 5 and 6 the declarations are marked with \code{static mut}.
Here we see that the zero initializes variable is assigned to the \elfsec{.bss} section in the {\elf} file.
On line 6 we have a non-zero value that has to be stored in Flash memory prior to execution and is copied to RAM in the \func{ResetHandler}.

\subsection{Bootstrapping startup}
\label{sec:startup}

The {\rust} program that was presented earlier in \autoref{lst:embedded-rust-hello-world} is quite obscure.
For this reason the \lib{startup} \footnote{\url{https://github.com/RustyGecko/startup/}} library was developed in order to minimize the effort of making a new {\rust} program for the {\gecko}.
This library makes the startup process more intuitive and hides all the details that were presented in \autoref{ssec:minimal_rust_program_to_boot}.

A minimal `Hello World' program using the \lib{startup} crate is given in \autoref{lst:embedded-rust-bootstrapped}.
We still have to annotate the program with \code{\#[no\_std]}, but the \code{main} function is now much more similar to the one that was presented at the start of this section.
This is due to the inclusion of the \lib{startup} library.

\begin{listing}[H]
  \begin{rustcode}
#![no_std]
#![feature(no_std)]
extern crate startup;

fn main() {
  loop {}
}
  \end{rustcode}
  \label{lst:embedded-rust-bootstrapped}
  \caption{Embedded Hello World bootstrapped with the \lib{startup} library}
\end{listing}

\subsection{Handling Interrupts in Rust}
% \todo{This sections might be moved}
\label{sec:impl:handling-interrupts}

Interrupts are an integral part of embedded programs, and having a native way of handling the interrupts provides a great benefit and flexibility for a programming language.

To handle interrupts on the {\gecko}, one have to register a function in the \emph{interrupt vector}.
This vector is defined in the \textbf{startup} file provided by Silicon Labs and is simply a list of function pointers defined in the \textbf{.isr\_vector} section of the {\elf} binary.
This is the first section in the \textbf{text} segment of the binary which ensures that it is located at memory address \mem{0x0} when the \gls{mcu} starts executing.
When an interrupt occurs, the microcontroller will inspect the interrupt vector and find the address of the handler function for the interrupt which occurred.
Both the interrupt vector and the default interrupt handlers are defined in the \textbf{startup} file for the {\gecko}.
The default implementations are simply infinite loops defined as weak symbols.
These weak symbols allows the programmer to redefine the symbol elsewhere in the code in order to override this default implementation.

\autoref{lst:c:irq} shows how the \func{SysTick\_Handler} can be overridden in {\C}.
The {\gecko} can be triggered to cause interrupts that occur at a timely basis, e.g. once every second.
This function will then be called when these interrupts occur.

\begin{listing}[H]
  \begin{ccode}
void SysTick_Handler(void) { /* Handler code */ }
  \end{ccode}
  \caption{Defining the SysTick Interrupt Handler in {\C}.}
  \label{lst:c:irq}
\end{listing}

Defining an interrupt handler in {\rust} is just as easy as in {\C} because of the focus on interoperability with {\C} code.
In {\rust}, a function can easly be defined to use the {\C} \glsdesc{abi} required to be called as an interrupt handler.
\autoref{lst:rust:irq} shows how to override the same \func{SysTick\_Handler} function in {\rust}.

\begin{listing}[H]
  \begin{rustcode}
#[no_mangle] pub extern fn SysTick_Handlder() { /* Handler code */ }
  \end{rustcode}
  \caption{SysTick Interrupt Handler in {\rust}.}
  \label{lst:rust:irq}
\end{listing}
