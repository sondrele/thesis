\section{Language Challenges}
\label{sec:disc:lang-challenges}

In \autoref{sec:intro:assignment} we identified six language challenges that must be considered when using a language in an embedded system.
Here, we revisit each challenge and discuss how they were solved for the {\rg} platform.

\begin{description}
\item [LC1 {\lci}] \hfill \\
  {\rust} exposes two \concept{intrinsics} for handling volatile read and write.
  These makes the code more verbose compared to the mechanism in {\C}, where a variable is marked with the \code{volatile} keyword.
  In {\rust}, the intrinsic functions must be used each time a variable is read or written.
  This, however, gives the programmer more fine-grained control as a variable can be used both as volatile and non-volatile.

\item [LC2 {\lcii}] \hfill \\
  Interrupt handlers must use the {\C} \gls{abi}, because the interrupts are dispatched from the {\C} runtime.
  This makes the code more verbose, but it is easily achievable in {\rust}.
  As discussed in \autoref{sec:ownership_allied_to_hardware}, a downside to the interrupted programming model found in embedded programming for {\rust} is the reliance on global mutable state.
  In these circumstances, the compiler is limited in its ability to statically verify safety because accesses to these variables must be contained inside {\unsafe} blocks.

\item [LC3 {\lciii}] \hfill \\
  In this thesis, we have only considered hardware devices as \glspl{mmio}.
  As {\rust} supports raw pointers and allows the programmer to access arbitrary memory addresses and cast these as structs, the handling of hardware registers are equally practical in {\rust} as in {\C}.
  Given a more mature embedded platform in {\rust}, we foresee that even more of the compile-time ownership analysis provided by the {\rust} compiler can be used to ensure safe interactions with hardware registers.

\item [LC4 {\lciv}] \hfill \\
  {\rust} (and {\C}) have, unlike many other programming languages, no life before the {\main} function.
  In this statement lies the fact that in the global scope, one can only initialize objects which have constant initializers.
  Therefore, all the initializers only contain constant data and they can be handled by the startup mechanisms described in \autoref{sec:back:startup}.
  This implies that the static object construction problem is non-existent in {\rust} programs.

\item [LC5 {\lcv}] \hfill \\
  Dynamic allocation in {\rust} is implemented in the \lib{alloc} library.
  We were easily able to include this functionality in \gls{rel}.
  In \autoref{sec:res:heap}, we showed that the heap allocation in {\rust} and {\C} are identical.
  This is due to the allocation algorithm in {\rust} being directly dependent on the \lib{newlib} \func{malloc} implementation, and therefore the memory fragmentation is equal to that of the existing {\C} platform.
  Note that this does not hold true for the regular {\rust} platform because it uses a different allocation algorithm.

\item [LC6 {\lcvi}] \hfill \\
  When the {\rust} allocator runs out of memory, it will call a function which is defined in the \lib{alloc} library.
  At the time of writing, this function does not provide any error handling, as it only calls a compiler intrinsic to abort the program.
  This means that a {\rust} program, which runs out of memory, will not end up in an infinite error handling loop, but the program ends up with a HardFault.

\end{description}
