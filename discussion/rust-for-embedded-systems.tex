% !TEX root = ../main.tex

\section{Rust for Embedded Systems} % (fold)
\label{sec:disc:rust_for_arm}
% \todo{Mention this in discussion. Is there a better way of solving this? Currently it fetches a release tarball of the language. It fetches more than it needs to for each library, but it's quicker/easier than using e.g. git to do it. It does however take a little while to compile all the dependencies on smaller machines.}

Before we started on this thesis, we were already aware that there were a couple of other projects that were running {\rust} bare-metal on ARM \glspl{mcu}.
We did, however, not know how easy it would be to get up and running with {\rust} on the EFM32.
It took us less than one day to implement and execute a {\rust} blink-demo on the {\STK}.
The {\rust} documentation was sufficient for figuring out how to define non-std {\rust} programs (i.e. the ones that are annotated with \texttt{\#[no\_std]}), and projects like \texttt{armboot} helped us build the project for the ARM architecture.
{\rust}'s runtime is comparable to {\C}'s, because the startup requirements in the two languages is the same, but {\rust}'s {\core} library is larger and provides more functionality than the {\C} standard library, which we consider to be an added bonus of {\rust}.

\subsection{The Standard Library} % (fold)
\label{sub:using_the_standard_library}

{\rust} has a rich standard library that offers a strong foundation for most {\rust} programs.
As we described in \autoref{ssub:rust:features}, the {\std} crate acts like an abstraction over all the different libraries in \gls{rsl} and the {\rustc} compiler is itself dependent on a couple of traits to be present in this module.
This crate is not applicable to all domains because it requires an \gls{os}.
In our domain, we have found the dual responsibility of the {\std} crate, of both providing a facade to \gls{rsl} and \gls{os}-dependent functionality, to be limiting.
In principle, the facade applies to our platform, but the added functionality does not.
This prevents us from using the facade, and thus also all libraries that depends on {\std}.

\gls{rcl} does provide us with enough functionality to write idiomatic {\rust} code, but we had to implement a few workarounds in order to get support for {\rust}'s standard constructs for dynamic memory allocation.
Currently, {\rust} does not have functionality to provide, or automatically compile, any of its standard libraries for target architectures other than the ones it supports by default.
It was hard to handle these dependencies before we introduced the {\cargo} build process to the project.
First when {\cargo} was introduced, was it easy to stay up to date with the nightly releases of the {\rust} compiler and its standard libraries.

Because our project targets an ARM processor architecture, we have to conditionally download and compile the libraries (like \lib{alloc} and \lib{collections}) that have functionality that we want to utilize, as described in \autoref{ssub:transitioning_to_cargo}.
This is not necessarily a big problem in itself, as it does not complicate the process much more than the need to explicitly define the libraries as custom dependencies in the \texttt{Cargo.toml}.
We do, however, think that the distinction between standard and non-standard programs make it considerably harder to enable us to use the great ecosystem that {\rust} is surrounded by.

\subsection{Using and Distributing Libraries}

We have described {\cargo} and its ability to manage package dependencies, and a package repository closely associated with {\cargo} is called \texttt{crates.io}\footnote{\url{https://crates.io}}.
This repository stores, at the time of writing, a couple of thousand different {\rust} crates, all made available for anyone who builds {\rust} programs with {\cargo}.
The vast majority of these crates are, to the best of our knowledge, linked to {\std} and utilize some part of it, either directly or transitively through a package dependency.
That being said, they do not necessarily use any functionality that we have not already made available through \gls{rel} in our project.
This simple fact renders nearly all (if not all) packages available through \texttt{crates.io} unusable for our project, even though the functionality that many of these packages depend on might already be available through \gls{rel}.

As an example to demonstrate this problem, we look to the \texttt{lazy-static}\footnote{\url{https://crates.io/crates/lazy_static}} project available on \texttt{crates.io}.
This project allows the programmer to declare \emph{static} variables that get initialized at \emph{runtime}.
It provides some of the same functionality that we implemented for the event-hub described in \autoref{sec:irq-closures}.
We can not use this project directly because it depends on {\std}.
With a closer look at the project, we can see that the project author at some point did prototype an alternate (and outdated) version which was only dependent on {\core}.
Another example is the \texttt{smallvec}\footnote{\url{https://crates.io/crates/smallvec}} project, which provides a handful of optimized versions of {\rust}'s vector structure that have a length-limit of 32 entries.
Similar to the \texttt{lazy-static} project, all the dependencies for \texttt{smallvec} are included from {\std}, but are also available in {\core}.
We have implemented a structure that is similar to \texttt{smallvec} as part of the {\tracker} application, but it feels like an unnecessary addition because it is already available on \texttt{crates.io} \emph{and} it only depends on features from {\core}.

This distinction between {\std} and \gls{rsl} feels somewhat contradictory to {\rust}'s focus on modularization and package distribution with {\cargo}.
It also feels destructive for non-standard projects like ours, because we would either have to re-implement or modify the already existing projects in order to get them to work for our platform.
We hope that it will get easier in time to define non-standard projects, while simultaneously being able to utilize many of the great open-source libraries that are already available.

\subsection{Language Challenges}
\label{sec:disc:lang-challenges}

In \autoref{sec:intro:assignment} we identified six language challenges that must be considered when using a language in an embedded system.
Here, we revisit each challenge and discuss how they were solved for the {\rg} platform.

\begin{description}
\item [LC1] - {\lci} \hfill \\
  {\rust} exposes two \concept{intrinsics} for handling volatile read and write.
  These makes the code more verbose compared to the mechanism in {\C}, where a variable is marked with the \code{volatile} keyword.
  In {\rust}, the intrinsic functions must be used each time a variable is read or written.
  This, however, gives the programmer more fine-grained control as a variable can be used both as volatile and non-volatile.

\item [LC2] - {\lcii} \hfill \\
  Interrupt handlers, in the {\rg} platform, must use the {\C} \gls{abi}, because the interrupts are dispatched from the {\C} runtime.
  This makes the code more verbose, but it is easily achievable in {\rust}.
  As discussed in \autoref{sec:avoiding_mutable_aliases_to_hardware}, a downside to the interrupted programming model found in embedded programming for {\rust} is the reliance on global mutable state.
  In these circumstances, the compiler is limited in its ability to statically verify safety because accesses to these variables must be contained inside {\unsafe} blocks.

\item [LC3] - {\lciii} \hfill \\
  In this thesis, we have only considered hardware devices as \glspl{mmio}.
  As {\rust} supports raw pointers and allows the programmer to access arbitrary memory addresses and cast these as structs, the handling of hardware registers are equally practical in {\rust} as in {\C}.
  Given a more mature embedded platform in {\rust}, we foresee that even more of the compile-time ownership analysis provided by the {\rust} compiler can be used to ensure safer interactions with hardware registers.

\item [LC4] - {\lciv} \hfill \\
  {\rust} (and {\C}) have, unlike many other programming languages, no life before the {\main} function.
  In this statement lies the fact that in the global scope, one can only initialize objects which have constant initializers.
  Therefore, all the initializers will only contain constant data and these can be handled by the startup mechanisms described in \autoref{sec:back:startup}.
  This implies that the static object construction problem is non-existent in {\rust} programs.

\item [LC5] - {\lcv} \hfill \\
  Dynamic allocation in {\rust} is implemented in the \lib{alloc} library.
  We were easily able to include this functionality in \gls{rel}.
  In \autoref{sec:res:heap}, we showed that the heap allocation in {\rust} and {\C} are identical.
  This is due to the allocation algorithm in {\rust} being directly dependent on the \lib{newlib} \func{malloc} implementation, and therefore the memory fragmentation is equal to that of the existing {\C} platform.
  Note that this does not strictly hold true for all {\rust} platforms because they can use different allocation algorithms.

\item [LC6] - {\lcvi} \hfill \\
  When the {\rust} allocator runs out of memory, it will call a function which is defined in the \lib{alloc} library.
  At the time of writing, this function does not provide any error handling, as it only calls a compiler intrinsic to abort the program.
  This means that a {\rust} program, which runs out of memory, will not end up in an infinite error handling loop, but the program ends up with a HardFault.

\end{description}
