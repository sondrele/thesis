
% Used in order to get the desired kind of abbreviations. I think...
\renewcommand{\glossarysection}[2][]{}

% Inserts 'Listing 4' when '\autoref{lst:listing}' a listing is referenced
\providecommand*{\listingautorefname}{Listing}

%\renewcommand\thesection{\arabic{chapter}.\arabic{section}}
%\renewcommand\thesubsection{\arabic{chapter}.\arabic{section}.\arabic{subsection}}
%\renewcommand\thesubsubsection{\arabic{section}.\arabic{subsection}.\arabic{subsubsection}}

% \newcommand{\frontchapter}[1]{
%     \setcounter{chapter}{1}
%     \setcounter{section}{1}
%     \chapter*{#1}
%     \addcontentsline{toc}{chapter}{#1}
% }

\newcommand{\reqi}{Identify and describe the requirements for an embedded platform in {\rust}}
\newcommand{\reqii}{Prototype an embedded software platform for {\rust} on the EFM32}
\newcommand{\reqiii}{Evaluate code size, performance and energy consumption}

\newcommand{\lci}{Volatile read and write}
\newcommand{\lcii}{Handling interrupts}
\newcommand{\lciii}{Reading and writing hardware registers}
\newcommand{\lciv}{Static object construction}
\newcommand{\lcv}{Heap allocation}
\newcommand{\lcvi}{Error handling without allocation}
