\section{Project}
\todo{Should this chapter be in Methodology or Introduction?}

\subsection{Project Breakdown}

The aim of this thesis is to evaluate the Rust language used in an embedded computer system.
The work was divided into several phases each having its own primary goal and subgoals.
\autoref{tab:meth:phases} summarizes the phases with the primary goals annotated.

\begin{table}[H]
  \begin{center}
    \begin{tabular}{|l|l|}
      \hline
      Phase&Goal \\
      \hline
      \hline
      Initial programmability&Getting Hello World up an running.\\
      \hline
      Library Support&Define a standard for creating library bindings\\
      \hline
      Evaluation&Evaluate the work so far and define how to proceed\\
      \hline
      Communication&Add support for communcation protocol (UART/I2C)\\
      \hline
      Development&Developing projects for evaluating the platform\\
      \hline
      Evaluating&Measuring performance\\
      \hline
      Reporting&Finalize project report\\
      \hline
    \end{tabular}
  \end{center}
  \caption{Phases of the project}
  \label{tab:meth:phases}
\end{table}

In addition to the main phases in the project, the build system has evolved continously.

The rest of this section describes each phase in more detail.

\subsubsection{Phase 1 - Getting Started}
The initial phase of the project defined two main activities.
Firstly the initial direction of the project along with major challenges was identified in a meeting with Silicon Labs w/Marius Grannæs and Mikael Berg.
Secondly the milestone of getting up and running with Rust on the Gecko was reached.

\paragraph{Initial direction}
The initial direction set in the first meeting with Silicon Labs.
Here a tentative project plan was set, given in \autoref{fig:meth:initial}

\begin{listing}[H]
  \begin{itemize}
    \item Get up and running with Rust
    \item Port/bind central libraries
    \item Evaluate system
    \item Handle Interrupts
    \item Evaluate system for performance/energy
  \end{itemize}
  \caption{Initial Project Plan}
  \label{fig:meth:initial}
\end{listing}

\paragraph{Major Challenges}

The major challenges identified in the initial meeting with Silicon Labs are given in \autoref{fig:meth:challenges}

\begin{listing}[H]
  \begin{itemize}
    \item Volatile read and write
    \item Handling Interrupts
    \item Reading and Writing Hardware registers
    \item Statical object construction
    \item Heap allocation
    \item Error Handling without allocation
  \end{itemize}
  \caption{Major Challenges}
  \label{fig:meth:challenges}
\end{listing}

These challenges and their relation to the project are described in \autoref{} \todo{Write this section in implementation chapter.}

\paragraph{Up and running}
The activity identified as getting up and running consists of successfully compiling, linking and executing a Rust program on the EFM32GG.
There were two options for assessing a successfull program on the Gecko, a complete and a simple.
The complete would be to start the program in debugging with the GDB debugger and verify the success.
While the simple was to create the simplest program with visual feedback on the STK.
The latter was chosen as the starting point and is described in \autoref{} \todo{Add to Implementation chapter}

\subsubsection{Phase 2 - Library Support}

After the initial compilation process was in place, the bindings for the \textbf{cmsis} and \textbf{emlib} libraries were started.
In this phase the technical difficulities for creating the FFI bindings was in focus when selecting which part of the libraries to develop.
As described in \autoref{} \todo{Background section on emlib design} the libraries to bind to expose their interfaces either as static utility functions or object oriented-like interfaces.
\autoref{tab:meth:lib} describes the initial modules developed with a description of their interaface and rational for chosing to develop the module.

\begin{table}[H]
  \begin{center}
    \begin{tabular}{|l|l|l|}
      \hline
      Module/Library & Interface & Rational \\
      \hline
      \hline
      core/cmsis & Static Utility & Low level dependency needed for Interrupts \\
      rtc/emlib & Singleton Object Oriented & Simple example of the interface type \\
      timer/emlib & Object Oriented & Example of the interface type \\
      cmu/emlib & Object Oriented & Dependency for handling clocks in emlib \\
      gpio/emlib & Static Utility & Simple module to provide testablility \\
    \end{tabular}
  \end{center}
  \caption{Table of initial library modules}
  \label{tab:meth:lib}
\end{table}

In addition to the

\subsubsection{Phase 3 - Evaluation}

\subsubsection{Phase 4 - Communication}

\subsubsection{Phase 5 - Development}

The goal of the development phase was to get enough empirical data to evaluate the system in Phase 5.
Two complementary projects was specified given in \autoref{tab:meth:projects}.

\begin{table}[H]
  \begin{tabular}{|l|l|l|}
    \hline
    & Name & Emphasis \\
    \hline
    \hline
    Project I & Sensor Tracker & Energy Efficiency \\
    \hline
    Project II & Classic Game & Performance \\
    \hline
  \end{tabular}
  \caption{Projects developed in development phase}
  \label{tab:meth:projects}
\end{table}

Both projects are implemented in both Rust, using the framework developed for this thesis, and C using the libraries provided by Silicon Labs.

\paragraph{Project I - Sensor Tracker}
The Sensor Tracker application was chosen to measure the energy efficiency of the system.
The application tracks sensor data periodically from the sensors provided on the STK and the BIOMETRIC-EXP-EVB expansion board.
Sensor data is stored in RAM and can be extracted over UART.

A full specification of Project I is included in Appendix \autoref{app:project:i}.

\paragraph{Project II - Classic Game}
The Game project was developed to measure the performance of the complete system.
As a starting point for the project the game assignment for the course TDT4258 developed by Antonio Garcia Guirado was used.
The game is a interactive graphical application written for the DK.

A full specification of Project II is included in Appendix \autoref{app:project:ii}

\subsubsection{Phase 6 - Evaluating}
\subsubsection{Phase 7 - Reporting}

\section{Toolchain}

About using
\begin{itemize}
 \item arm-none-eabi-gcc/newlib
 \item CMSIS/emlib
 \item rustc/cargo
\end{itemize}

\section{Using C as a framework for Rust}

About
\begin{itemize}
 \item why link to c vs only rust/zinc.rs
 \item ffi
\end{itemize}
