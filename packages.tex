\usepackage[utf8]{inputenc}
\usepackage[english]{babel}

% More defined colors, used for custom todos
\usepackage[pdftex,dvipsnames]{xcolor}

% Used for new commands taking more than one argument
\usepackage{xargs}

% For orange todos
\usepackage{todonotes}

% Lorem ipsum
\usepackage{lipsum}

% For b5 papers instead of a4
\usepackage{geometry}
\geometry{b5paper}

% Multirow and multicolumn tables
\usepackage{multirow}

% Line break in tables
\usepackage{pbox}

% Create and format mathematical things
\usepackage{amsmath}

% For \checkbox and other symbols
\usepackage{amssymb}
\usepackage{tipa}
% Provides frowny face \Frowny
\usepackage{marvosym}

% Rotate text sideways, might be used in tables
\usepackage{rotating}

% Generate urls
\usepackage{url}

% Nesting includes
\usepackage{newclude}

% Correct citations...? Not sure, doesn't hurt
\usepackage{natbib}

% Floating in tables and figures
\usepackage{float}

% Generate abbreviations, include after hypreref to link. Cite with \gls and \glspl
% Define with: \newacronym{mac}{MAC}{Media Access Control}
% Complile with 'makeglossaries main'
% \usepackage[acronyms, nonumberlist]{glossaries}
\usepackage{glossaries}

% Double line shift instead of indentation for paragraphs
\usepackage[parfill]{parskip}

% Automatically include 'Figure' with \autoref
\usepackage{hyperref}
\newcommand*{\Appendixautorefname}{Appendix}
% Has to be included after hyperref, makes numbering in pdf index
\usepackage[numbered]{bookmark}

% Makes the bibliography part of the contents
\usepackage[nottoc,numbib]{tocbibind}

% Per-chapter numbering of figures and tables
\usepackage{chngcntr}
% See: https://tex.stackexchange.com/questions/28333/continuous-v-per-chapter-section-numbering-of-figures-tables-and-other-docume
\counterwithin{figure}{section}
\counterwithin{table}{section}

% To provide appendices in the contents
\usepackage[toc,page]{appendix}

% \usepackage{pgfplots}
% \usepackage[nottoc]{tocbibind}

% For listings. This package requires the pdf to be compiled with '-shell-escape' and that
% python-pygments is found in the system path
\usepackage{minted}
