\section{Report Outline}

\newcommand{\outlineref}[1]{\textbf{\autoref{#1} \nameref{#1}}}

\outlineref{chap:intro} introduces and gives motivation for using {\rust} in an embedded system.
The interpretation of the assignment and an outline of the project is presented along with a summary of the contributions of the project.

\outlineref{chap:background} provides background material for the rest of the thesis.
The Rust programming language is introduced along with the bundled package manager, {\cargo}.
Further, the existing hardware platform, EFM32, and software libraries used for developing embedded applications are presented.

\outlineref{chap:startup} presents what happens in order for a {\rust} program to start executing on the \gls{mcu}.

\outlineref{chap:rel} gives an overview over the {\rust} standard library for embedded systems.

\outlineref{chap:bindings} goes into detail on the bindings developed for the peripheral libraries used to control the \gls{mcu}.

\outlineref{chap:build} looks at the build system used to build application for the {\rg} platform.

\outlineref{chap:app} present some high level libraries and some application build on the core {\rg} functionality.

\outlineref{chap:results} present how the platform was evaluated and the results from the evaluation.
The platform was evaluated by considering \emph{code size}, \emph{performance} and \emph{energy efficiency}.

\outlineref{chap:discussion} provides a discussion of the platform and the results presented in \autoref{chap:results}.

\outlineref{chap:conclusion} presents a conclusion based on the discussion of the platform and outlines possible extensions and suggests further work based on this project.
