\section{Report Outline}

\newcommand{\outlineref}[1]{\autoref{#1} \nameref{#1}}

\paragraph{\outlineref{chap:intro}} introduces and gives motivation to using {\rust} in an embedded system.
A outline of the project and is presented along with a summary of the contributions of the project.

\paragraph{\outlineref{chap:background}} provides background material for the rest of the Thesis.
The Rust programming language is introduced along with the builtin package manager, Cargo.
Further, the existing hardware platform, EFM32, and software libraries used for developing embedded applications are presented.

\paragraph{\outlineref{chap:impl}} looks at the components developed to provide the platform described in the Project Description.
As most of these components are largely orthogonal to each other, each are described in a separate section followed by a small dicussion, where applicable.
In addition, the two projects developed to evaluate the platform in \autoref{chap:results} is described.

\paragraph{\outlineref{chap:results}} presents how the platform was evaluated and the results of the evaluation.
The platform was evaluated by considering \emph{code size}, \emph{performance} and \emph{energy efficiency}.

\paragraph{\outlineref{chap:conclusion}} presents conclusion drawn based on the state of the platform and results.

\paragraph{\outlineref{chap:future}} outlines possible extentions and suggests further work based on this project.
