\section{Contributions}

\newcommand{\cbuild}[0]{\textbf{C1}}
\newcommand{\crustc}[0]{\textbf{C2}}
\newcommand{\cmethods}[0]{\textbf{C3}}
\newcommand{\cbugfix}[0]{\textbf{C4}}
\newcommand{\ceval}[0]{\textbf{C5}}

The contributions of this thesis are given in \autoref{tab:contributions}.

\begin{table}[H]
  \centering
  \begin{tabular}{r | l}
    \textbf{Contribution} & \textbf{Description} \\
    \hline
    {\cbuild} & Build System \\
    {\crustc} & The {\rustc} {\cargo} Subcommand \\
    {\cmethods} & Methods for using Rust abstractions \\
    {\cbugfix} & Minor bugfix in a Silicon Labs library \\
    {\ceval} & Evaluation of Embedded Rust \\
    \hline
  \end{tabular}
  \caption{Contributions of the Thesis}
  \label{tab:contributions}
\end{table}

The {\cbuild} build system is to our knowledge the first ergonomic build system using the {\rust} standard package manager to build executables for non standard targets.
Other projects have resorted to custom Makefiles to handle the build process and dependencies.

In order to develop the build system {\cbuild}, the standard package manager had to be modified.
This was at first done by a problem specific plugin, but was later made more generic and resulted in implementing and contributing the \cmd{rustc} subcommand {\crustc}, to the Cargo package manager.

Throughout the development phase of the projects the high level abstractions of Rust was tested out in an embedded environment.
These experimentations resulted in the a few successfull and promissing patterns {\cmethods}.

By porting a driver in the Silicon Labs library from {\C} to {\rust} a minor bug was found and reported with a suggested patch {\cbugfix}, to fix the issue.

At last the results reported and discussed in this thesis provides an evaluation {\ceval}, of the Rust platform in an embedded system.
