\section{Contributions}

\newcommand{\crustygecko}[0]{\textbf{C1}}
\newcommand{\cbuild}[0]{\textbf{C2}}
\newcommand{\crustc}[0]{\textbf{C3}}
\newcommand{\cmethods}[0]{\textbf{C4}}
\newcommand{\cbugfix}[0]{\textbf{C5}}
\newcommand{\ceval}[0]{\textbf{C6}}

The contributions of this thesis are given in \autoref{tab:contributions}.

\begin{table}[H]
  \centering
  \begin{tabular}{r | l}
    \textbf{Contribution} & \textbf{Description} \\
    \hline
    {\crustygecko} & The {\rg} platform \\
    {\cbuild} & Build process \\
    {\crustc} & The {\cargo} {\rustc} Subcommand \\
    {\cmethods} & Methods for using {\rust} abstractions \\
    {\cbugfix} & Minor bugfix in a Silicon Labs library \\
    {\ceval} & Evaluation of {\rust} for embedded system \\
    \hline
  \end{tabular}
  \caption{Contributions of the Thesis}
  \label{tab:contributions}
\end{table}

The design and implementation of the {\rg} platform ({\crustygecko}), in addition to examples for using the platform.

The implemented build process ({\cbuild}) is, to our knowledge, the first standard build process using the {\rust} package manager, {\cargo}, for embedded systems.
Other projects have resorted to custom Makefiles to handle the build process and dependencies.

In order to develop the build process ({\cbuild}), {\cargo} had to be modified.
This resulted in implementing and contributing the subcommand ({\crustc}) to the {\cargo} package manager.

Throughout the development phase of the project the high level abstractions of {\rust} was tested out in an embedded environment.
These experimentations resulted in the a few successful and promising patterns ({\cmethods}).

By porting a driver in one of Silicon Labs' software libraries from {\C} to {\rust}, a minor bug was found and reported with a suggested patch ({\cbugfix}) to fix the issue.

At last the results reported and discussed in this thesis provides an evaluation ({\ceval}) of the {\rust} platform in an embedded system.
