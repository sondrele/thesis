\section{Project Outline}
\label{sec:project-outline}

The aim of this thesis is to evaluate the {\rust} language used in an embedded computer system.
The work was divided into several phases each having its own primary goal and subgoals.
\autoref{tab:meth:phases} summarizes the phases with the primary goals annotated.

\begin{table}[H]
  \centering
  \begin{tabular}{l|l|l}
    \textbf{Phase}&\textbf{Goal}&\textbf{Finished} \\
    \hline
    Initial Programmability & Getting Hello World up an running. & 09.01.15\\
    Library Support & Design initial libraries & 13.03.15\\
    Development & Develop projects evaluation &  09.04.15 \\
    Evaluating & Measuring performance & 20.05.15 \\
    Reporting & Finalize project report & 08.06.15\\
    \hline
  \end{tabular}
  \caption{Phases of the project}
  \label{tab:meth:phases}
\end{table}

In addition to the main phases in the project, the build system has evolved continously.

The rest of this section describes each phase in more detail.

\subsection{Phase 1 - Initial Programmability}
The initial phase of the project defined two main activities.
Firstly the initial direction of the project along with major challenges was identified in a meeting with Silicon Labs w/Marius Grannæs and Mikael Berg.
Secondly the milestone of running the first {\rust} application on the {\gecko} was reached.


\subsection{Phase 2 - Library Support}

After the initial compilation process was in place, the focus shifted to developing the platform for writing larger applications.
Early in this phase the technical challenges involved was in focus when selecting which part of the libraries to develop.
Throughout this Phase we developed initial support for the bindings described in \autoref{sub:interfacing_with_emlib} and the \gls{rel} described in \autoref{sec:rel}.
The library support was continouosly evolved when needed in the development phase.

\subsection{Phase 3 - Development}
\label{sec:projects}

The goal of the development phase was to create applications that would provide enough empirical data to evaluate the system in Phase 4.
Two complementary projects were specified given in \autoref{tab:meth:projects}.

\begin{table}[H]
  \centering
  \begin{tabular}{l|l|l}
    & \textbf{Name} & \textbf{Emphasis} \\
    \hline
    Project I & Sensor Tracker & Energy Efficiency \\
    Project II & Classic Game & Performance \\
    \hline
  \end{tabular}
  \caption{Projects developed in development phase}
  \label{tab:meth:projects}
\end{table}

Both projects are implemented in both {\rust}, using the platform developed for this thesis, and {\C} using the libraries provided by Silicon Labs.

\subsection{Phase 4 - Evaluation}
During the phase for evaluation of the platform was evaluated based on the metrics in \autoref{lst:evaluation:criterias}.
The results from the evaluation are presented in \autoref{chap:results} along side with results obtained by considering the existing platform in {\C}.

\begin{listing}
  \begin{itemize}
  \item Performance
  \item Code Size
  \item Energy Consumption
  \end{itemize}
  \caption{Metrics for evaluation of the platform}
  \label{lst:evaluation:criterias}
\end{listing}

\subsection{Phase 5 - Reporting}

Throughout the reporting phase all the other phases described above were concluded and improved to be presented in this report.
The inconsistensies in results where investigated and a discussion of the project as whole was conducted and is presented in \autoref{chap:discussion}.
