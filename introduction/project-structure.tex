\section{Project Outline}
\label{sec:project-outline}

The aim of this thesis is to evaluate the {\rust} language used in an embedded computer system.
The work was divided into several phases each having its own primary goal and subgoals.
\autoref{tab:meth:phases} summarizes the phases with the primary goals annotated.

\begin{table}[H]
  \begin{center}
    \begin{tabular}{l|l}
      \textbf{Phase}&\textbf{Goal} \\
      \hline
      Initial Programmability & Getting Hello World up an running.\\
      Library Support & Define a standard for creating library bindings\\
      Development & Developing projects for evaluating the platform\\
      Evaluating & Measuring performance\\
      Reporting & Finalize project report\\
      \hline
    \end{tabular}
  \end{center}
  \caption{Phases of the project}
  \label{tab:meth:phases}
\end{table}

In addition to the main phases in the project, the build system has evolved continously.

The rest of this section describes each phase in more detail.

\subsection{Phase 1 - Initial Programmability}
The initial phase of the project defined two main activities.
Firstly the initial direction of the project along with major challenges was identified in a meeting with Silicon Labs w/Marius Grannæs and Mikael Berg.
Secondly the milestone of running the first {\rust} application on the {\gecko} was reached.

\paragraph{Major Challenges}
The major challenges identified in the initial meeting with Silicon Labs are given in \autoref{fig:meth:challenges}

\begin{listing}[H]
  \begin{itemize}
    \item Volatile read and write
    \item Handling Interrupts
    \item Reading and Writing Hardware registers
    \item Statical object construction
    \item Heap allocation
    \item Error Handling without allocation
  \end{itemize}
  \caption{Major Challenges}
  \label{fig:meth:challenges}
\end{listing}

These challenges and their relation to the project are described in \autoref{} \todo{Write this section in discussion chapter.}

\subsection{Phase 2 - Library Support}

After the initial compilation process was in place, the bindings for the \lib{cmsis} and {\emlib} libraries were started.
Early in this phase the technical challenges involved in creating the FFI bindings was in focus when selecting which part of the libraries to develop.
The library support was continouosly evolved when needed in the Development Phase.

\subsection{Phase 3 - Development}
\label{sec:projects}

The goal of the development phase was to create application that would provide enough empirical data to evaluate the system in Phase 4.
Two complementary projects were specified given in \autoref{tab:meth:projects}.

\begin{table}[H]
  \centering
  \begin{tabular}{l|l|l}
    & \textbf{Name} & \textbf{Emphasis} \\
    \hline
    Project I & Sensor Tracker & Energy Efficiency \\
    Project II & Classic Game & Performance \\
    \hline
  \end{tabular}
  \caption{Projects developed in development phase}
  \label{tab:meth:projects}
\end{table}

Both projects are implemented in both {\rust}, using the framework developed for this thesis, and {\C} using the libraries provided by Silicon Labs.

\subsection{Phase 4 - Evaluating}
During the evaluating phase the develop platform was evaluated based on the following metrics in \autoref{lst:evaluation:criterias}.
The results of the evaluation is presented in \autoref{chap:results} along side with results obtained by considering the existing platform in {\C}.

\begin{listing}
  \begin{itemize}
  \item Performance
  \item Code Size
  \item Energy Consumption
  \end{itemize}
  \caption{Metrics for evaluation of the platform}
  \label{lst:evaluation:criterias}
\end{listing}
This phase also included tuning the applications and libraries to minimize the bias in the results.
