\section{Project Outline}
\label{sec:project-outline}

In this section, we break down the project into five phases.
\autoref{tab:meth:phases} presents each of the phases with their primary goals, and the rest of this section will describe how each phase was carried out, in more detail.
In addition to the main phases of the project, the build system has evolved continuously.

\begin{table}[H]
  \centering
  \begin{tabular}{l|l|l}
    \textbf{Phase}&\textbf{Goal}&\textbf{Finished} \\
    \hline
    Hello World     & Blink LED with {\rust} & 09.01.15 \\
    Platform Design & Implement support libraries & 13.03.15 \\
    Development     & Develop projects for measurements &  09.04.15 \\
    Measurement     & Measuring the projects & 20.05.15 \\
    Evaluation      & Evaluating the platform & 08.06.15 \\
    \hline
  \end{tabular}
  \caption{Phases of the project}
  \label{tab:meth:phases}
\end{table}

\subsection{Phase 1 - Hello World}
The initial phase of the project defined two main activities.
Firstly, the direction of the project along with some major challenges was identified in a meeting with Marius Grannæs and Mikael Berg from Silicon Labs.
Secondly, the milestone of running the first {\rust} application on the \gls{mcu} was reached.

\subsection{Phase 2 - Platform Design}

After the initial compilation process was in place, the focus shifted towards developing the platform for writing larger applications.
Throughout this phase, we developed the support libraries for the MCU, which are described in \autoref{sec:rel} and \autoref{sub:interfacing_with_emlib}.
Early in this phase, technical challenges guided the choice of which part of the platform to develop.
The platform evolved continuously during the development phase.

\subsection{Phase 3 - Development}
\label{sec:projects}

The goal of the development phase was to create applications that would provide enough empirical data to evaluate the platform in phase 4.
The two complementary projects in \autoref{tab:meth:projects} were specified and implemented as part of this phase.

\begin{table}[H]
  \centering
  \begin{tabular}{l|l|l}
    & \textbf{Name} & \textbf{Emphasis} \\
    \hline
    Project I & {\tracker} & Energy Efficiency \\
    Project II & {\cg} & Performance \\
    \hline
  \end{tabular}
  \caption{Projects developed in development phase}
  \label{tab:meth:projects}
\end{table}

Both of these projects were implemented in {\rust}, using the platform developed for this thesis, and in {\C}, using the libraries provided by Silicon Labs.

\subsection{Phase 4 - Measurement}
During the measurement phase, the platform was evaluated based on the following metrics:

\begin{itemize}
  \item Performance
  \item Energy Consumption
  \item Code Size
\end{itemize}

The results of the evaluation are presented and compared with the existing {\C} platform in \autoref{chap:results}.

\subsection{Phase 5 - Evaluation}

Throughout this phase, we investigated the results that were gathered in the previous phase.
These results provided a basis for a discussion of the project as a whole, which is presented in \autoref{chap:discussion}.
In this discussion, we look at the viability of using {\rust} in an embedded system and present the thoughts we had, and the experiences we made, during the work for this thesis.
