\section{Libraries}
\label{sec:back:lib}

This section describes the software libraries utilized in for the implemenetation described in this thesis.
The libraries are written in C and used through Rusts Foreign Function Interface as described in \autoref{} \todo{Write FFI section}.

\autoref{tab:back:libs} lists the libraries that are used to run Rust on the Gecko.

\begin{table}[H]
  \begin{tabular}{|l|l|l|}
    \hline
    Library & Description & Source \\
    \hline
    \hline
    emdrv & The Energy Micro energyAware Drivers & SiliconLabs \\
    emlib & The Energy Micro Peripheral Support library & SiliconLabs \\
    CMSIS & The Cortex M HAL & ARM \\
    newlib & C library for embedded systems & GNU ARM Toolchain \\
    \hline
  \end{tabular}
  \caption{Underlying libraries}
  \label{tab:back:libs}
\end{table}

\subsection{emdrv}
\subsection{emlib}

The Energy Micro Peripheral Support library exposes the vast range of peripherals that are possible to connect the Gecko.
The emlib library is divided into separate file that defines modules such as \textbf{adc}, \textbf{dac}, \textbf{timer}, \textbf{dma}, et cetera.
The library functions are exposed in three different API patterns.
Either as standalone utility methods, singleton object methods or C object oriented fashion.
To see these differnet patterns we consider the \textbf{timer}, \textbf{rtc} and \textbf{gpio} module.

\begin{listing}[H]
  \begin{minted}{c}
    void TIMER_Init(TIMER_TypeDef *timer, const TIMER_Init_TypeDef *init);
    void TIMER_Enable(TIMER_TypeDef *timer, bool enable);
    void TIMER_TopGet(TIMER_TypeDef *timer);
    void TIMER_TopSet(TIMER_TypeDef *timer, uint32_t val);
  \end{minted}
  \caption{Timer Module - C Object Oriented}
  \label{lst:back:lib:timer}
\end{listing}

\begin{listing}[H]
  \begin{minted}{c}
    void RTC_Init(const RTC_Init_TypeDef *init);
    void RTC_Enable(bool enable);
    uint32_t RTC_CompareGet(unsinged int comp);
    void RTC_CompareSet(unsigned int comp, uint32_t value);
  \end{minted}
  \caption{RTC Module - Singleton object}
  \label{lst:back:lib:rtc}
\end{listing}

\begin{listing}[H]
  \begin{minted}{c}
    void GPIO_PinModeSet(GPIO_Port_TypeDef port, unsigned int pin,
                        GPIO_Mode_TypeDef mode, unsigned int out);
    void GPIO_PinOutSet(GPIO_Port_TypeDef port, unsigned int pin);
    unsigned int GPIO_PinOutGet(GPIO_Port_TypeDef port, unsigned int pin);
  \end{minted}
  \caption{}
  \label{lst:back:lib:gpio}
\end{listing}



\subsection{CMSIS}

The ARM CMSIS, Cortex Microcontroller Software Interface Standard, is a Hardware Abstraction Layer for the Cortex-M series.
The library is divided into a few subsections namely; Core, DSP, RTOS and SVD.
Out of these only the Core section is used for our library.
The Core library consists of functions to control Interrupts though NVIC, the system clock, program tracing through ITM and references to memory mapped registers.


\subsection{newlib}

Newlib is a implementation of a C library for embedded devices.
Most notably this library defines heap memory management facilities through \textbf{malloc}, \textbf{realloc} and \textbf{free}.
These base function are needed in order to compile both the \textbf{emlib}, \textbf{emdrv} and Rust \textbf{alloc} libraries.
The functions in newlib is used directly by the C libraries or in Rust through the \textbf{libc} library which exposes the memory management symbols mentioned above.
This library is distributed as a part of the GNU ARM Toolchain described in \autoref{} \todo{Refere to toolchain section}.
