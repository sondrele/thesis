\section{Startup}
\label{sec:back:startup}

This section describes the process of starting a program on an embedded system.
The {\main} function is typically considered as the starting point of a program.
While this can be true from the programmer's point of view, in this section, it is evident that that is not usually the case.
This section will present the reader with what occurs prior to the program reaching the \code{main} function.

\subsection{Prelude}

Before we look at the start of execution, lets look at the process between compilation and execution.
The compiler outputs object-files which the linker combines into a {\elf} file.
This file is loaded into the microcontrollers \gls{rom} Flash memory.
Then, the reset signal is sent to the \gls{cpu} to start execution.

\subsection{Executable and Linkable File Format}
\label{sec:back:elf}
%\todo{Reference: http://infocenter.arm.com/help/topic/com.arm.doc.ihi0044e/IHI0044E\_aaelf.pdf}
{\elf} is a file format for storing code and data for a program.
This file format can be loaded onto a microcontroller and executed.
For our purposes, this format defines three interesting sections, namely \elfsec{.text}, \elfsec{.data}, and \elfsec{.bss}.
\autoref{fig:elf} describes the content of the these sections.

\begin{table}[H]
  \centering
  \begin{tabular}{l|l}
    \textbf{ELF Header} & \textbf{Header describing size of sections} \\
    \hline
    \elfsec{.bss}  & Logical section for zero-initialized data \\
    \elfsec{.text} & Read-only section for code and constants \\
    \elfsec{.data} & Section containing values for non-zero initialized data \\
    \hline
  \end{tabular}
  \caption{Sections of {\elf} file format}
  \label{fig:elf}
\end{table}

The \elfsec{.text} section contains the program code and any read-only data defined in the code.
Strings are, for instance, stored in the \elfsec{.text} section as read-only.
The \elfsec{.data} section contains the values for non-zero initialized data.
The \elfsec{.bss} section is a logical section so it is not really stored in the file.
The header specfies the size of the \elfsec{.bss} section, which describes the size of the zero-initialized data in the program.

\subsection{Before main}
\label{sec:before-main}
All static data must be initialized before the \code{main} function starts executing.
The data is divided into two categories; zero and non-zero initialized.
The non-zero initialized data is contained either in the \elfsec{.text} (read-only-data) or \elfsec{.data} sections of the {\elf} binary.
On the chip, the \elfsec{.data} section resides in flash memory, as these data structures might change at runtime they must be copied into \gls{ram}.
The \elfsec{.text} section contains the instructions and read-only data, this data does not need to be copied as they can be read directly from flash at runtime.
All zero-initialized data is represented by the \elfsec{.bss} section in the {\elf} binary.
A portion of \gls{ram} corresponding to the size of the \elfsec{.bss} section must be set to zero.

Copying the \elfsec{.data} section into \gls{ram} is handled by the \func{ResetHandler} function shown in \autoref{lst:reset-handler}.
Before the \gls{ram} is initialized, the \code{ResetHandler} calls a \code{SystemInit} routine, which can be used to set up the external \gls{ram} for the {\gecko}.
The function is the real entry point of the program and is executed when the \gls{cpu} receives the reset signal.

\begin{listing}[H]
\begin{ccode}
// Variables set in linker script
extern etext;      // address of .data segment in FLASH
extern data_start; // start of .data segement in RAM
extern data_end;   // end of .data segment in RAM
void ResetHandler() {
  SystemInit();
  for (i=0; i<data_end - data_start; i++){
    RAM[data_start + i] = FLASH[etext+i];
  }
  _start();
}
\end{ccode}
\caption{The \gls{mcu} \code{ResetHandler}}
\label{lst:reset-handler}
\end{listing}

After the \func{ResetHandler} has copied the non-zero initialized data into \gls{ram}, it calls the \func{\_start} function in the {\C} runtime.
The \func{\_start} function finds the size of the \elfsec{.bss} section, which describes the size of zero-initialized area in \gls{ram}.
It then calls on the \func{memset} function to zero out the \gls{ram}, before it goes on to call the {\main} function defined by the programmer.

\begin{listing}[H]
\begin{ccode}
// Variables set in linker script
bss_start; // start of .bss segment in RAM
bss_end;   // end of .bss segment in RAM

void _start() {
  memset(bss_start, 0, bss_end - bss_start);
  main(); // User defined main function
}
\end{ccode}
\caption{{\C} runtime start routine}
\label{lst:start}
\end{listing}
