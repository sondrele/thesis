\subsection{Startup}

The \textbf{main} function is usually considered the starting point of a program.
This is sometimes true from the point of view of the programmer, but as we will see in this section usually it is not.
This section looks at what happens before the program reaches the main function.

\subsubsection{Before main}

Before the \textbf{main} function executes, all static data must be initialized.
The data is divided into two categories; zero initialized and non-zero.
The non-zero initialized data are contained either in the .text (read-only-data) or .data sections of the \textbf{elf} binary.
On the chip the .data section resides in Flash memory, and must be copied into RAM.
The \textbf{.text} section contains the instructions and read-only data 
This process is handled in the \textbf{ResetHandler} function.
The function is the entry point of the program and is executed when the CPU receives the reset signal.

When the \textbf{ResetHandler} has copied the non-zero initialized data into RAM it calls the \textbf{\_start} in the C runtime.
The \textbf{\_start} function finds the size of the .bss section which describes the size of zero-initialized area in RAM.
It then calls on the \textbf{memset} function to zero out the RAM area.
\textbf{\_start} then goes on to call the \textbf{main} function supplied by the programmer.

With an example we will now see how this maps to the storage qualifiers in Rust.
\begin{figure}
\begin{verbatim}
const      RUST_CONST_ZERO: u32 = 0;      // not allocated
const      RUST_CONST: u32 = 0xFEED;      // not allocated
static     RUST_STATIC_ZERO: u32 = 0;     // .text
static     RUST_STATIC: u32 = 0xDEAD;     // .text
static mut RUST_STATIC_MUT_ZERO: u32 = 0; // .bss
static mut RUST_STATIC_MUT: u32 = 0xBEEF; // .data

fn main() { /* Use the variables */ }
\end{verbatim}
\caption{Rust static initialization}
\label{fig:rust-static-init}
\end{figure}
In Figure \ref{fig:rust-static-init} the three different types of declaring globals in Rust are shown.
Firstly Rust divides between two types of global declarations, constants and statics.
A constant declaration, shown in lines 1 and 2 of Figure \ref{fig:rust-static-init}, represents a value.
There is no need to allocate memory for globals declared as \textbf{const} and they behave like \textbf{\#define} in C.
Variables on line 3 and 4 declares the variables to be \textbf{static}.
As variables in Rust is immutable by default these two are allocated in the read-only section called \textbf{.text}.
The values are read directly from Flash and need not be copied to RAM.
On line 5 and 6 the declarations are marked with \textbf{static mut}. 
