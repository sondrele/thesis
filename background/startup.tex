\subsection{Startup}

The \textbf{main} function is usually considered the starting point of a program.
This is sometimes true from the point of view of the programmer, but as we will see in this section usually not.
This section looks at what happens before the program reaches the main function.

\subsubsection{Before main}

Before the \textbf{main} function executes, all static data must be initialized.
The data is divided into two categories; zero initialized and non-zero.
The non-zero initialized data are contained either in the .text (read-only-data) or .data sections of the \textbf{elf} binary.
On the chip the .data section resides in Flash memory, and must be copied into RAM.
This process is handled in the \textbf{ResetHandler} function.
The function is the entry point of the program and is executed when the CPU receives the reset signal.

When the \textbf{ResetHandler} has copied the non-zero initialized data into RAM it calls the \textbf{\_start} in the C runtime.
The \textbf{\_start} function finds the size of the .bss section which describes the size of zero-initialized area in RAM.
It then calls on the \textbf{memset} function to zero out the RAM area.
\textbf{\_start} then goes on to call the \textbf{main} function supplied by the programmer. 
