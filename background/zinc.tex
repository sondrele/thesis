% !TEX root = ../main.tex

\section{The Zinc Project} % (fold)
\label{sec:zinc}

The {\zinc} \footnote{\url{http://zinc.rs/}} project tries to write a complete ARM stack, similar to that of \gls{cmsis}, but written completely in {\rust} and assembly, and with no interference of {\C}.
{\zinc} is an attempt at applying {\rust}'s safety features to embedded programming, but it is still in early development.
The project have primarily focused on supporting two different ARM development boards, the EFM32 are not part of these.

One of {\zinc}'s main features is the ability to safely initialize a program's peripherals with a \emph{Platform Tree} specification, which has the ability to statically catch any mis-configured \gls{mcu} pins or peripherals during compilation.
This setup-routine guarantees that all the peripherals gets initialized correctly.
This Platform Tree is realized with a {\rust} \emph{compiler plugin}, which means that {\zinc} can hook on to {\rustc}'s internal compilation routine, and verify the correctness of the platform specification that it is currently attempting to compile.

An example program that demonstrates the Platform Tree is shown in \autoref{lst:platform_tree_example}, we have left out parts of the \code{clock} and \code{os} specifications to make the example more clear.
We can see from this example that we define a platform for the LPC17 \gls{mcu}, and initialize the main \gls{mcu} clock, a Timer peripheral, and configure a \gls{gpio} pin as a LED.
If {\zinc} notices that e.g. a LED and a Timer is configured to use the same \gls{mcu} pin, it will exit the compilation with an error message.
In the \code{os} block we specify that we want access to the Timer as \code{timer} and the LED as \code{led1}, {\zinc} will handle the task of initializing the peripherals and pass them as arguments to the \code{run} function.

\begin{listing}[H]
  \begin{minted}{rust}
platformtree!(
  lpc17xx@mcu {
    clock { source = "main-oscillator"; /* ... */ }
    timer { timer@1 { counter = 25;  divisor = 4; } }
    gpio { 1 { led1@18 { direction = "out"; } } }
  }
  os {
    args { timer = &timer; led1 = &led1; }
    /* ... */
  }
);
// Blink LED every second
fn run(args: &pt::run_args) {
  args.led1.set_high();
  args.timer.wait(1);

  args.led1.set_low();
  args.timer.wait(1);
}
  \end{minted}
  \caption{Simplified example usage of Zinc's Platform Tree}
  \label{lst:platform_tree_example}
\end{listing}

Although the {\zinc} project did not end up as part of our platform, we have used it for inspiration during design and development.
In \autoref{chap:discussion} we discuss the problem that arises with mutable aliases to hardware peripherals, and look at how {\zinc}'s approach of handling peripheral initialization can help to provide a safer abstraction layer over the hardware, with respect to our platform.

