% !TEX root = ../main.tex

\section{EFM32}
\label{sub:emf32}

EFM32 refers to a family of 32-bit ARM microcontrollers that is developed by Silicon Labs.
They are based on the ARM Cortex-M series of RISC processor cores, a range of different processor cores intended for microcrontrollers, that focuses on low cost and energy-usage.
These factors are crucial in modern systems and applications where energy efficiency is of great importance.
For example with the \gls{iot}, where we predict a future where tens of billions of devices will be connected to the Internet, ranging from Super Computers down to small embedded devices that might be used to power up and control everything from cars to light bulbs via the Internet.
The different processor cores of the Cortex-M family are summarized in \autoref{tab:cortex_m}, ranging from the smallest and simplest processor cores, to the more advanced ones with more features.

\begin{table}
\begin{center}
\begin{tabular}{r|l}
    \textbf{Name} & \textbf{Purpose/features}            \\
    \hline
    Cortex-M0 & Lowest cost and lowest area              \\
    Cortex-M0+ & Lowest power                            \\
    Cortex-M1 & Designed for implementation in FPGAs     \\
    Cortex-M3 & Performance efficiency                   \\
    Cortex-M4 & DSP, SIMD, FP                            \\
    Cortex-M7 & Cache, TCM, AXI, ECC, double + single FP \\
    \end{tabular}
\end{center}
\caption{Cortex-M}
\label{tab:cortex_m}
\end{table}

The EFM32 family of microcontrollers are all based on different Cortex-M processors, some of their
features are summarized in \autoref{tab:efm32_family}.
The focus of these microcontrollers is energy efficiency and low power-consumption in resource-constraint environments.
The microcontrollers implement several different methods for reducing the power consumption.
The most important way to achieve low power consumption is by turning off the different part of the processor that are inactive, so that these parts do not longer draw any power from the overall system.
The EFM32 processors features five different energy modes, or sleep modes, ranging from EM0 (Energy Mode 0) to EM4, where EM4 is the lowest and wakeup on interrupts.
The different peripherals provided with the EFM32's operate in different energy modes.
This allows applications to utilize many different peripherals for data collection and  while the processor itself is turned off, but with the opportunity to wake up the processor on different interrupt-signals in order to do general processing.

\begin{table}
\begin{center}
    \begin{tabular}{r|l|l}
    \textbf{Name} & \textbf{Processor} & \textbf{Speed (MHz)} \\
    \hline
    Zero Gecko    & ARM Cortex-M0+ & 24 \\
    Tiny Gecko    & ARM Cortex-M3  & 32 \\
    Gecko         & ARM Cortex-M3  & 32 \\
    Leopard Gecko & ARM Cortex-M3  & 48 \\
    Giant Gecko   & ARM Cortex-M3  & 48 \\
    Wonder Gecko  & ARM Cortex-M4  & 48 \\
    \end{tabular}
\end{center}
\caption{EFM32}
\label{tab:efm32_family}
\end{table}



\section{Emlib} % (fold)
\label{sub:emlib}

\emlib is a peripheral \gls{api} by Silicon Labs that provides an abstraction layer over the different peripherals for their EFM32 series of ARM microcontrollers.
It is a general library written in C that provides functions and modules to read, write and control the different peripherals that is supported by the EMF32 microcontrollers.
With the \gls{api}, the application programmer can control and set up the different peripherals and easily control sleep modes and wakeup, and interrupt handlers among others.
The library is built on the \gls{cmsis} library provided by ARM.
It is possible to control the EFM32's with this library as well, or a combination between the two, but \emlib is designed to function as a standalone library.
