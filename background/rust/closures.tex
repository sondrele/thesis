\subsection{Closures}
\label{sec:back:rust:closures}

Closures, a type of anonymous functions, exposes many expressive programming patterns which have emerged the last few years through the popularization of JavaScript, and several functional languages.
The consquence is that both C++ and Java has added support for creating anonymous functions called lambda expressions.

A closure is a function has free variable which is contained within the scope the function is defined.
The function is said to close over these variables called the environment of the closure.
The closure can be used as an argument to another function or returned from a maker function.
\autoref{lst:closure:filter} shows how a closure can be used to filter a list of numbers.

\begin{listing}[H]
  \begin{minted}{rust}
fn main() {
  let limit = 3;

  let nums = vec![1,2,3,4];
  let filtered = nums.iter().filter(|el| el < limit).collect();

  println!("{}", filtered);
}
  \end{minted}
  \caption{}
  \label{lst:closure:filter}
\end{listing}
\todo{This example does not work}

The example creates a list of 4 numbers, makes an iterator which is filtered with the closure and collected into a new vector.
The expression \texttt{|el| el < limit} defines the closure with a regular parameter \texttt{el}.
The \texttt{limit} is a free variable which is found in the scope of the \texttt{main} function.
This variable is added to the closures environment.
