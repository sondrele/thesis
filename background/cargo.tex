% !TEX root = ../main.tex

\section{The Cargo Package Manager}
\label{sec:cargo}

\cargo is Rust's package manager, a program that automates the process of building and distributing Rust programs.
It comes bundled with the default installation of Rust.
\cargo comes with a collection of tools that makes building, testing and running Rust programs a lot easier than invoking \rustc directly.
\cargo also defines a standard Rust project layout, in addition to downloading and maintaining package dependencies.
This section will cover the most important features of \cargo and how it works, enough of what is required in order to understand the work that is described later in this project report.

\subsection{Project Structure}
\label{ssub:project_structure}

Every crate, or package, built by \cargo requires a \file{Cargo.toml} file to be present in the root directory.
This file is interpreted by \cargo and is used to determine the name of the library and executables to be built in the package.
Any dependencies that the package might depend on is also specified in this file, \cargo will automatically download, compile, and link these dependencies with the project.
It also includes information that tells \cargo how the package can be compiled for different target architectures, and it is used to define different \concept{features} of a package - a way to conditionally compile certain parts of the code present in the library.

\begin{listing}
\dirtree{%
.1 hello\_world.
.2 Cargo.lock.
.2 Cargo.toml.
.2 src.
.3 main.rs.
}
\caption{Minimal Cargo project structure}
\label{lst:standard_cargo_project_structure}
\end{listing}

\autoref{lst:standard_cargo_project_structure} shows a standard \cargo project structure for an executable target.
If a project contains a \file{main.rs} file, it will be compiled into an executable with the same name as the project (\file{hello\_world} for this example).
It is required by \cargo that this file also contains the \main function for the program.
Finally, \cargo also generates a \file{Cargo.lock} file that contains \emph{specific} information about every package that are used in the project.
This information includes the name and version of the package and its dependencies, including any transitive dependencies they might have.
This file helps \cargo to determine if packages needs to be re-downloaded, updated, or re-compiled in order for the project to be built consistently - independent on the target architecture it is built on and for.

\begin{listing}
\dirtree{%
.1 cargo.
.2 Cargo.lock.
.2 Cargo.toml.
.2 src.
.3 bin.
.4 build.rs.
.4 test.rs.
.4 ....
.3 lib.rs.
.3 compile.rs.
.3 ....
.2 tests.
.3 test\_cargo\_build.rs.
.3 test\_cargo\_test.rs.
.3 ....
}
\caption{Expanded Cargo project structure}
\label{lst:expanded_cargo_project_structure}
\end{listing}

\autoref{lst:expanded_cargo_project_structure} shows an example of a library project structure, and is in fact a simplified version of \cargos actual structure.
If the intended use of a project is as a library for other applications, they include a \file{lib.rs}.
This file, including its submodules like \file{src/compile.rs} in this example, will be compiled into a Rust crate with the same name as the project, in this case it will be \file{cargo.rlib}.
Any files found under the \dir{src/bin} directory will also be compiled into its own executables, \cargo will automatically link the library and all of the its dependencies with these executables.
In \cargos case, every command shown in \autoref{tab:common_cargo_commands} has its own dedicated executable target in this directory.

A package can also contain a \dir{tests} directory with integration tests, like in the above example, and finally an \dir{examples} directory with different executables that demonstrates how to use the library (this has not been included in the above example).
A package can also have it's own build-routine by specifying a \file{build} script in the \file{Cargo.toml}.
This file is executed prior to building the package itself, and provides the possibility to e.g. compile and link third party C-libraries, or generate code prior to compilation.

\subsection{Building and testing}

As mentioned earlier, \cargo comes with a collection of tools that makes it easy to build and test Rust projects.
The most common commands are shown in \autoref{tab:common_cargo_commands}.
Most of these tools are self explanatory, but the \cmd{build} and the \cmd{test} commands will be described a little more thoroughly in this section.

\begin{table}[ht]
\begin{center}
\begin{tabular}{r|l}
\textbf{Command} & \textbf{Description}                           \\
\hline
build  & Compile the current project                              \\
clean  & Remove the target directory                              \\
doc    & Build this project's and its dependencies' documentation \\
new    & Create a new cargo project                               \\
run    & Build and execute src/main.rs                            \\
test   & Run the tests                                            \\
bench  & Run the benchmarks                                       \\
update & Update dependencies listed in Cargo.lock                 \\
\hline
\end{tabular}
\caption{Common cargo commands}
\label{tab:common_cargo_commands}
\end{center}
\end{table}

By invoking the \cmd{cargo build} command, \cargo will download and resolve all package dependencies, and trigger \prog{rustc} to compile them and link them with each other in the correct order.
When all dependencies have been built, the build script will be invoked (if it is present in the package), before the library itself is compiled.
Lastly, the projects executables will be compiled if the project contains a \file{main.rs} or sources in the \dir{src/bin} directory.

The \cmd{cargo test} command will also trigger \prog{rustc} to compile the library in the same manner as \cmd{cargo build}, but it will leave out compilation of the project executables.
When the library is compiled, any function that is marked with \attrib{\#[test]} will be included and treated like a unit test, this is a feature from Rust itself.
\cargo also treats all the sources found in the \dir{examples} directory as tests, so these will be compiled instead of the other project executables, together with all the integration tests.
When \cargo has finished compiling the library and its executables, it will by default run all the unit tests, including the integration tests, but it will not run the examples.

\begin{table}[ht]
\begin{center}
\begin{tabular}{r|l}
\textbf{Flag} & \textbf{Description}                                   \\
\hline
--features FEATURES   & Space-separated list of features to also build \\
--target TRIPLE       & Build for the target triple                    \\
\hline
\end{tabular}
\caption{Cargo flags to alter the package library and executables}
\label{tab:cargo_flags}
\end{center}
\end{table}

Both of these commands support several optional build-specific flags that are passed further on to the invocation of \rustc, we will take extra notice to the two flags shown in \autoref{tab:cargo_flags}.
The \flag{--target} flag is used if the project should be compiled for a different target architecture than the machine it is invoked on, this is necessary when cross-compiling the package.
The list following the \flag{--features} flag will be used by \cargo and \rustc to conditionally compile code that is present in the project.
Consider the example shown in \autoref{lst:rust_features} and its output shown in \autoref{tab:rust_features_output}.
We can see from this example that the definition of the \func{num} function will be different based on the feature flag that is passed together with \cargo.

\begin{listing}[H]
\begin{minted}{rust}
// src/main.rs
#[cfg(feature = "one")]
fn num() -> u32 { 1 }

#[cfg(feature = "two")]
fn num() -> u32 { 2 }

fn main() {
    println!("num() + num() = {}", num() + num())
}
\end{minted}
\caption{Example usage of features}
\label{lst:rust_features}
\end{listing}

\begin{table}[ht]
\begin{center}
\begin{tabular}{r|l}
\textbf{Command} & \textbf{Output}                          \\
\hline
\texttt{\$ cargo build --features one}  & num() + num() = 2 \\
\texttt{\$ cargo build --features two}  & num() + num() = 4 \\
\hline
\end{tabular}
\caption{Example output of features}
\label{tab:rust_features_output}
\end{center}
\end{table}
