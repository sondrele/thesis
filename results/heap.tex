\section{Heap Allocation}
\label{sec:res:heap}
In this section we consider a small experiment to investigate possible differences in heap fragmentation caused by dynamic allocation in {\C} and {\rust}.
We do know, from reading the source code of the library, that the heap allocation in {\rust} builds upon the same functionality as the allocation in {\C}.
However, we want to verify and show this explicitly.
The conducted experiment is a simple program that \textbf{1)} allocates 128 objects on the heap, \textbf{2)} replaces one object picked at random, and \textbf{3)} repeats step \textbf{2)} 1024 times.
The objects are of the three sizes given in \autoref{tab:heap:objects}.

\begin{table}[H]
  \centering
  \begin{tabular}{l|c|l}
    \textbf{Name} & \textbf{Size (32-bit words)} & \textbf{Color} \\
    \hline
    A & 2 & Green \\
    B & 3 & Red \\
    C & 6 & Blue \\
    \hline
  \end{tabular}
  \caption{Object sizes}
  \label{tab:heap:objects}
\end{table}

The heap allocation pattern was recorded after the first phase of the program, and are given in \autoref{fig:heap:init}.
The figure shows a 4KB subsection of the heap with normalized addressing to the first allocated object to aid the comparison.
Each line represents 64B of memory.

We see that the initial allocation pattern is alternating between these three object sizes, and the white spaces in between are padding.
These are due to alignment rules of the allocator.
The objects are created in the same order in both {\C} and {\rust} and we see that the allocation patterns are identical.

\begin{figure}[H]

  \centering
  \begin{subfigure}{0.47\textwidth}
    \centering
    \includegraphics[scale=0.15]{results/plots/heap/cinit}
    \caption{C}
    \label{fig:heap:init:c}
  \end{subfigure}
  \hfill
  \begin{subfigure}{0.47\textwidth}
      \centering
    \includegraphics[scale=0.15]{results/plots/heap/rinit}
    \caption{Rust}
    \label{fig:heap:init:r}
  \end{subfigure}
  \caption{Initial heap allocation of 128 objects in {\rust} and {\C}}
  \label{fig:heap:init}

\end{figure}

To get comparable results for both {\C} and {\rust}, we make sure that the pseudo randomly picked objects are the same.
This is controlled by using the same random number generator for both languages (this is the same \gls{rng} as discussed in \autoref{sec:res:perf}).
\autoref{fig:heap:frag} presents the heap allocation pattern after the second phase has been executed.
From this figure, we see that the allocation patterns in {\rust}, on the {\rg} platform, and {\C}, are identical.

\begin{figure}[H]

  \centering
  \begin{subfigure}{0.47\textwidth}
    \centering
    \includegraphics[scale=0.15]{results/plots/heap/cfrag}
    \caption{C}
    \label{fig:heap:frag:c}
  \end{subfigure}
  \hfill
  \begin{subfigure}{0.47\textwidth}
      \centering
    \includegraphics[scale=0.15]{results/plots/heap/rfrag}
    \caption{Rust}
    \label{fig:heap:frag:r}
  \end{subfigure}
  \caption{Heap allocation after processing in {\rust} and {\C}}
  \label{fig:heap:frag}

\end{figure}
