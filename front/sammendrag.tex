% !TEX root = ../main.tex

\chapter{Sammendrag}
\label{chap:sammendrag}

Innvevde datasystemer blir gradvis en større del av vår hverdag.
Disse datamaskinene må stadig tilpasse seg nye domener, slik som \emph{tingenes internett}.
Sammenlignet med andre konvensjonelle datasystemer (mobile-, personlige-, og tjeneste-systemer), er mange begrensninger i disse systemene som setter krav til hvilke utviklingsverktøy som kan brukes.

I nyere tid har det kommet mange nye programmeringsspråk som er tilrettelagt for konvensjonelle datasystemer.
Kategoriseringen av \emph{høyere-nivå programmeringsspråk} har endret seg i løpet av de siste tiårene, fra å handle om statiske, maskinvare-agnostiske språk som C, til å dreie seg om dynamiske språk med kjøre-tidssytemer, slik som JavaScript.
Sikkerhetsmekanismene som er tilgjengelige i disse høyere-nivå språkene kommer ofte på bekostning av lav-nivå kontroll, som er tilgjengelig i \emph{lav-nivå programmeringsspråk}.

Rust er et nytt og voksende programmeringsspråk, som gjør et forsøk på å skape et nytt kompromiss mellom kontroll og sikkerhet.
Dette språket kan garantere sikkerhet ved statisk analyse, som i andre høyere-nivå språk blir løst dynamisk.

I denne avhandlingen presenterer vi vår metode for å benytte Rust i et innvevd datasystem, og en evaluering av denne.
Vi beskriver designet og implementasjonen av vår operativsystemløse plattform kalt {\rg}, som omfatter biblioteker for å kontrollere maskinvaren.
Vi vister i tilleg flere programmer og abstrakte biblioteker som er blitt bygget på denne plattformen.

For å støtte plattformen har vi også implementert og presentert en utvidelse til Rust sin standard pakkebehandler.
Denne utvidelsen gjør det enklere å bygge Rust-applikasjoner for ikke-standard plattformer, og har også blitt inkludert i det opprinnelige prosjektet som utvikler pakkebehandleren.

Vi har evaluert plattformen basert på ytelse, energieffektivitet og kodestørrelse, ved bruk av en ARM Cortex-M3-basert EFM32-brikke kalt Giant Gecko.
Disse resultatene har blitt sammenlignet med den allerede eksisterende C plattformen.
Våre evalueringer viser at Rust har tilsvarende ytelse og energieffektivitet som C.
Vi har imidlertid oppdaget at kodestørrelsen kan øke betraktelig, særlig for applikasjoner som er bygget for feilsøking.
