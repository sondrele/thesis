% !TEX root = ../main.tex

\chapter{Conclusion}
\label{chap:conclusion}

\section{The RustyGecko Platform}

In this thesis we have presented the {\rg} platform for embedded programming using the {\rust} programming language.
This platform consists of a subset of the {\rust} standard library and a binding library around peripheral libraries for the EFM32 series of \glspl{mcu}.

We have explained that the platform makes use of many of the language and community features which comes with a modern programming language.
From our measured results we showed that by using this {\rust} based platform, we can still make applications which perform equally well as the existing {\C} platform, based on performance and energy consumption metrics.
Still, we did find evidence of the platform producing larger binaries, which makes {\rust} application more expensive, in terms of hardware costs, or rendering the platform useless for the most resource constrained applications.
We have also shown that there does exist some obstacles for making use of the convenient ways of sharing code between projects, although the basic mechanisms are in place.

\section{Requirements}

In \autoref{chap:intro} we presented three requirements as interpreted from the Project Description for this thesis.
Requirement \textbf{R1} is achieved through the presentation of the existing {\C} platform, given in \autoref{chap:background}, in comparison to the {\rust} platform that was presented throughout the core chapters.
\autoref{sec:disc:lang-challenges} covered the language challenges discussed in \autoref{sec:intro:assignment}, as part of requirement \textbf{R1}, and shows that each challenge is considered in the {\rg} platform.
We therefore conclude that these challenges are solved for using {\rust} in an embedded system.
Further we consider the {\rg} platform to fulfill requirement \textbf{R2}.
Throughout \autoref{chap:results} we presented our results of measuring the {\rg} platform and our evaluation of the measurements are given in \autoref{chap:discussion} to satisfy requirement \textbf{R3}.

\section{Future work}
\label{chap:future}

As we presented in \autoref{sub:interfacing_with_emlib} the platform consists of partially finished bindings.
An obvious extension to our project is to continue and finalize the binding libraries for the {\emlib}, {\emdrv} and \gls{cmsis} libraries.
The foundations for these binding have been laid out, so the implementations here are not challenging, only time-consuming.
An alternative approach is to use a bindings generator such as \lib{bindgen}\footnote{\url{https://github.com/crabtw/rust-bindgen}} to generate the binding automatically.
The binding effort might be redundant altogether, as the {\rust} roadmap states that the project wants to develop better integration with {\C}/{\Cpp} code \cite{web:roadmap}.
This might eventually lead to bindings like these being made obsolete, and that {\C} libraries can called into based on their header files, alone.

As discussed in \autoref{sec:disc:rust_for_arm}, we found the current basis for writing reusable libraries to be limiting for our platform.
We leave it as future work to go into detail on ways to solve this problem, by either applying a different approach or to modify the blocking obstacle in the \lib{std} crate.

In \autoref{sec:rust-embedded-modules} we looked at a library which provided higher level abstraction using the language features of {\rust}.

The interrupt-driven programming model used to program embedded system did not prove to be as applicable to {\rust} as we had hoped.
In \autoref{sec:irq-closures} we looked at a technique for changing this model by using ownership rules similar to how concurrency is handled in {\rust}.
This library has potential to provide a more idiomatic programming style to embedded programs in {\rust} and will make the platform more familiar to {\rust} developers.
This library can be extended to control the ownership of \glspl{mmio} as discussed in \autoref{sec:avoiding_mutable_aliases_to_hardware}.

In the \autoref{sec:disc:project_evaluations} we argued that in order to make a more thoroughly comparison of the {\C} and {\rust} languages in this domain, more of the platform has to be implemented in {\rust}.
On possible way to do this is to start this project from scratch by using the \lib{zinc.rs} framework and porting the actual implementations of the {\emlib} library to {\rust}.
In doing so, the examples provided with {\emlib} can be used as a benchmark suite to compare the two platforms.
This can both provide a better basis for evaluation, but can also, as we showed in \autoref{sec:porting-gpioint}, uncover bugs in the {\C} implementation of {\emlib}.
These discoveries can help motivate the usage of {\rust} in this domain.
Throughout this thesis we have only worked with the {\rust} language as the application layer language and binding to {\C}, it is possible to do this the other way around.
Therefore parts of the {\emlib} library can actually be replaced by an {\rust} implementation and still expose the same \gls{api} to the user.

An aspect that we did not go into in this project is {\rust}'s ability to, by using traits, redefine the semantics when using values.
This concept can possibly be applied to remove the verbosity of the volatile loads and store operations discussed in \autoref{sec:disc:lang-challenges}.
