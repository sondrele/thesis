% !TEX root = ../main.tex

\chapter{Conclusion}
\label{chap:conclusion}

In this chapter, we draw conclusions derived from the discussion of the {\rg} platform.
We also look back at the requirements that were defined in \autoref{chap:intro}, and give an evaluation of the project's success.
Finally, we give our thoughts on further developments to improve the {\rg} platform, and suggest some changes that could make {\rust} more suited for embedded systems.

\section{The RustyGecko Platform}

In this thesis, we have presented the {\rg} platform for programming an embedded system by using the {\rust} programming language.
This platform consists of a subset of the {\rust} standard library and a handful of libraries that define bindings for the EFM32 peripheral \gls{api}.
In total, we have been able to utilize many great aspects that surround the {\rust} programming language in the {\rg} platform.

One of the tools we were able to incorporate is the {\cargo} package manager.
This enabled us to give the {\rg} platform a modular design, which makes the different parts of the platform reusable as isolated units of code.
However, the \gls{rsl} is an implied dependency for all standard {\rust} projects.
Because of this, and given the current design of \gls{rsl}, it is hard for non-standard projects to utilize third-party libraries.

{\rust} implements many interesting features, among these are its ability to statically prevent data-races.
We attempted to apply these concepts directly to the hardware peripherals but found the resulting programming model to be too limited.
Instead, we propose to explore this idea further by considering a more dynamic approach for future work.

From our measured results, we showed that on the {\rg} platform, we can still make applications that perform as well as the existing {\C} platform.
This conclusion is derived from our qualitative study of performance and energy consumption for the two applications that was developed on this platform.
However, we did find indications that the platform produces larger binaries, which make {\rust} applications more expensive in terms of hardware costs.
The binaries are stored in flash memory and the size of this memory correlates with the cost of the microcontroller.
The increased cost in hardware can be expensive for resource constrained applications, which are shipped in the millions.
This, consequently, can turn out to be a hindrance for applications in the industry.

\section{Requirements}

In \autoref{chap:intro}, we presented three requirements as interpreted from the Project Description for this thesis.
In this section, we describe how each of these requirements were fulfilled.

\begin{description}
    \item[R1] - {{\reqi}} \hfill \\
    We identified six language challenges that was important to solve to get {\rust} to work on an embedded system.
    These challenges were all handled in the implementation of the {\rg} platform.

  \item[R2] - {{\reqii}} \hfill \\
    We consider the {\rg} platform to fulfill this requirement.
    The prototype provides the same functionality as the existing {\C} platform in principal, and needs only be fleshed out to have full support.

    \item[R3] - {{\reqiii}} \hfill \\
    Throughout \autoref{chap:results}, we presented our results of measuring the {\rg} platform, and our evaluation of the measurements are given in \autoref{chap:discussion} to satisfy this requirement.
\end{description}

Based on these requirements and their solutions, we consider the project as a whole to solve the problems deduced from the Project Description.
This conclusion leads us to further conclude that the project was successful.

\section{Future Work}
\label{chap:future}

We have worked on many different projects on the {\rg} platform throughout this thesis.
In this section, we suggest some of these projects as future work.
We also suggest some new project ideas based on our experiences during the development of the {\rg} platform.

\begin{description}
\item[Bindings] \hfill \\
  As we presented in \autoref{sub:interfacing_with_emlib}, the {\rg} platform consists of partially finished bindings.
  An obvious extension to our project is to continue and finalize the binding libraries for the {\emlib}, {\emdrv} and \gls{cmsis} libraries.
  The foundations for these bindings have been laid out, so the implementations here are not challenging, only time-consuming.
  An alternative approach is to use a bindings generator, such as \lib{bindgen}\footnote{\url{https://crates.io/crates/bindgen}}, to generate the bindings automatically.
  Note that the binding effort might be redundant altogether, as the {\rust} roadmap states that the project wants to develop better integration with {\C} and {\Cpp} code \cite{web:roadmap}.
  This might eventually lead to bindings like these being made obsolete and that {\C} libraries can be called into based on their header files alone.

\item[Trait-based Peripheral Access] \hfill \\
  An aspect that we did not go into in this project is {\rust}'s ability to, by using traits, redefine the semantics when using values.
  This concept can possibly be applied to remove the verbosity of the volatile loads and store of peripheral control register operations, as discussed in \autoref{sec:disc:lang-challenges}.

\item[The Standard Library] \hfill \\
  As discussed in \autoref{sec:disc:rust_for_arm}, we found the current basis for writing reusable libraries to be limiting for our platform.
  One of the major problems is the interoperability between standard and non-standard libraries, which limits the amount of third-party libraries that can be utilized for the {\rg} platform.
  We leave it as future work to go into detail on ways to solve this problem, by either applying a different approach or to modify the blocking obstacle in the \lib{std} crate.

\item[\glsdesc{rel}] \hfill \\
  In \autoref{sec:rust-embedded-modules}, we looked at a library that provided higher-level abstractions by using the language features of {\rust}.
  This library can be further developed to include better initialization procedures for the peripherals, as in {\zinc}, and extended to provide abstractions for more peripherals.

\item[Safe Access to Hardware] \hfill \\
  The interrupt-driven programming model that is used to programming of embedded systems did not prove to be as applicable to {\rust} and its ownership-semantics as we had hoped.
  {\rust}'s ownership system helps to rule out data races in programs \cite{web:rust_book}.
  We believe that the ownership system can be utilized in a new framework, based on the one that was described in \autoref{sec:irq-closures} and further discussed in \autoref{sec:avoiding_mutable_aliases_to_hardware}, for initializing and dispatching interrupts.
  It will be interesting to see if {\rust}'s ownership-semantics can help to apply a more dynamic approach to this framework, in order to ensure safe access to hardware peripherals.

\item[Using Rust at the Library Layer] \hfill \\
  Throughout this thesis, we have only worked with the {\rust} language at the application layer language and binding to {\C} for library support.
  It can be interesting to do this the other way around.
  Concretely, parts of the {\emlib} library can be replaced with a {\rust} implementation and still expose the same \gls{api} to the user and be callable from a {\C} application.

\item[Rust-only Solution] \hfill \\
  We chose to base the {\rg} platform on the software libraries that were already available for the {\gecko}.
  This enabled us to quickly shape the platform and explore many different project ideas.
  However, we argued in \autoref{sec:disc:project_evaluations} that to make a more thoroughly comparison of the {\C} and {\rust} languages in this domain, more of the platform have to be implemented in {\rust}.
  A new direction for the platform can be to adapt the {\zinc} project for the {\gecko}, and to port the actual implementations of the {\emlib} library to {\rust}.
  This approach can provide a better basis for an evaluation of {\rust}.
  It is also possible that {\rust}'s \code{unsafe}-semantics can help to uncover more bugs in the {\C}-implementation of {\emlib}.

\end{description}
